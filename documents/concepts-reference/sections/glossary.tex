% Copyright (c) 2021 Eclipse Arrowhead Project
%
% This program and the accompanying materials are made available under the
% terms of the Eclipse Public License 2.0 which is available at
% http://www.eclipse.org/legal/epl-2.0.
%
% SPDX-License-Identifier: EPL-2.0

This section provides an alphabetically sorted list of all significant terms introduced or named in this document.
Each term consisting of more than one word is sorted by its final, or qualified, word.
This means that the definition of \GlossaryHyperRef{protocol-service}{service protocol}, for example, is found at \GlossaryNameRef{protocol-service}.

Many of the definitions are amended with notes and references to IoTA:AF \cite{delsing2017iot}, ISO42010 \cite{iso42010}, SOA-RM \cite{mackenzie2006reference} and RAMI4.0 \cite{adolphs2016reference}, which are always listed after the definition they amend.
Regular notes are numbered, while those making a comment on a definition in IoTA:AF, ISO42010, SOA-RM or RAMI4.0 are introduced with the abbreviations just listed.

{

\newcommand{\GlossaryEntry}[3][]{\subsubsection*{#3\IfStrEq{#1}{}{}{ {\normalfont \textit{#1}}}}\label{sec:glossary:#2}}
\newcommand{\GlossaryNote}[2]{\begin{minipage}[b]{\dimexpr\linewidth-0.5cm\relax}\vspace*{0.33cm}\footnotesize{\textbf{#1}\ #2}\end{minipage}}

\GlossaryEntry{acquirer}{Acquirer}
A \GlossaryHyperRef{stakeholder}{stakeholder} in the process of acquiring, or considering to acquire, a \GlossaryHyperRef{system}{system} or \GlossaryHyperRef{system-of-systems}{system-of-systems} with the intent to operate and/or use it.
See Section \ref{sec:concepts:stakeholder}.

\GlossaryEntry{architect}{Architect}
A \GlossaryHyperRef{stakeholder}{stakeholder} who designs or specifies Arrowhead systems or systems-of-systems, or who extends the \GlossaryHyperRef{framework-arrowhead}{Arrowhead framework} itself, by, for example, writing core documentation or producing architectural \GlossaryHyperRef{description}{descriptions}.
See Section \ref{sec:concepts:stakeholder}.

\GlossaryEntry{architecture}{Architecture}
A \GlossaryHyperRef{model}{model} of a \GlossaryHyperRef{system-of-systems}{system-of-systems} defined in terms of (1) goals, ambitions and other design principles; (2) an environment, either abstract or concrete; (3) as well as significant life-cycle events, such as construction, maintenance or decommissioning.

	\GlossaryNote{ISO42010}{
		defines architecture as ``$<$system$>$ fundamental concepts or properties of a system in its environment embodied in its elements, relationships, and in the principles of its design and evolution''.
		Our definition should be interpreted as being equivalent.
		Note that ISO42010 uses the term ``element'' to refer to what we call a \textit{\GlossaryHyperRef{model}{model} \GlossaryHyperRef{entity}{entity}}.
	}

	\GlossaryNote{SOA-RM}{
		defines software architecture as ``the structure or structures of an information system consisting of entities and their externally visible properties, and the relationships among them''.
		That definition is equivalent to our definition of \GlossaryHyperRef{model}{model}, with the exception that the thing being modeled has to be an information system.
		As our definition is concerned with a model and a system-of-systems, which must be an information system, we regard out definition as compatible but more specific.
	}

	\GlossaryNote{RAMI4.0}{
		defines architecture as the ``combination of elements of a model based on principles and rules for constructing, refining and using it''.
		We consider ``combinations of elements of a model'' to be a ``model of a system-of-systems'' and to be ``based on principles and rules for constructing, refining and using it'' as being concerned with principles, an environment and life-cycle events.
		Our definition should be interpreted as being compatible but more specific.
	}

\GlossaryEntry{architecture-software}{Architecture, Software}
Prefer \GlossaryNameRef{architecture}.

\GlossaryEntry{arrowhead}{Arrowhead}
The name of the initiative part of which this document and the rest of the \GlossaryHyperRef{framework-arrowhead}{Arrowhead framework} is being produced.

\GlossaryEntry{artifact}{Artifact}
A thing or object, tangible or intangible.

\GlossaryEntry{asset}{Asset}
Synonymous to \GlossaryNameRef{resource}.

	\GlossaryNote{RAMI4.0}{
		defines asset as an ``object which has a value for an organization''.
		See \GlossaryNameRef{resource} for a comparable term.
	}

\GlossaryEntry{assumption}{Assumption}
The taking for granted that some fact or statement is true.

	\GlossaryNote{Note 1}{
		Assumptions can be an important method of imposing delimitation to a \GlossaryHyperRef{model}{model} or \GlossaryHyperRef{framework}{framework}.
		If, for example, assuming that a suitable means of secure \GlossaryHyperRef{communication}{communication} will exist, a model of communication can be formulated without having to specify how its \GlossaryHyperRef{message}{messages} are secured.
	}

\GlossaryEntry{attribute}{Attribute}
A name/value pair of \GlossaryHyperRef{data}{data}, associated with either an \GlossaryHyperRef{entity}{entity} or a \GlossaryHyperRef{relationship}{relationship}.

	\GlossaryNote{Note 1}{
		A attribute is a form of \GlossaryHyperRef{metadata}{metadata}.
	}

\GlossaryEntry{automation}{Automation}
The control of a process by a mechanical or electronic apparatus, taking the place of human labor.

\GlossaryEntry{boundary}{Boundary}
A point or border where either two or more \GlossaryHyperRef{artifact}{artifacts} meet or one artifact ends.

\GlossaryEntry{boundary-cloud}{Boundary, Cloud}
A \GlossaryHyperRef{boundary}{boundary} separating the \GlossaryHyperRef{artifact}{artifacts} belonging to a \GlossaryHyperRef{cloud}{cloud} from those not belonging to it.

	\GlossaryNote{Note 1}{
		A cloud boundary can be \GlossaryHyperRef{boundary-local}{local} or \GlossaryHyperRef{boundary-virtual}{virtual}, depending on if the boundary is formed by physical or virtual \GlossaryHyperRef{attribute}{attributes}.
	} 

\GlossaryEntry{boundary-local}{Boundary, Local}
A \GlossaryHyperRef{boundary}{boundary} that exists in the physical world.

	\GlossaryNote{Note 1}{
		Local boundaries can be facilitated by walls, locations of operation, attachment to certain vehicles or power sources, and so on.
	}

\GlossaryEntry{boundary-virtual}{Boundary, Virtual}
A \GlossaryHyperRef{boundary}{boundary} that exists only \GlossaryHyperRef{virtual}{virtually}.

	\GlossaryNote{Note 1}{
		Virtual boundaries can be facilitated by cryptographic secrets, identifiers, ownership statements, contracts, and so on.
	}

\GlossaryEntry{builder}{Builder}
A \GlossaryHyperRef{stakeholder}{stakeholder} constructing \GlossaryHyperRef{arrowhead}{Arrowhead} \GlossaryHyperRef{system-automation}{automation systems} by assembling and preparing \GlossaryHyperRef{device}{devices}, as well as installing \GlossaryHyperRef{system}{systems} on those devices.
See Section \ref{sec:concepts:stakeholder}.

\GlossaryEntry{capability}{Capability}
A task, of any nature, that can be executed by an \GlossaryHyperRef{artifact}{artifact}.

	\GlossaryNote{Note 1}{
		The term must be understood in the most general sense possible.
		It includes the abilities of \GlossaryHyperRef{hosting-system}{hosting systems}, reading from sensors, triggering actuators, among many other possible examples.
	}

	\GlossaryNote{SOA-RM}{
		defines a capability as ``a real-world effect that a service provider is able to provide to a service consumer''.
		Our definition is more general in the sense that not only \GlossaryHyperRef{provider-service}{service providers} are allowed to have capabilities.
		See also \GlossaryNameRef{capability-system}.
	}

\GlossaryEntry{capability-device}{Capability, Device}
A \GlossaryHyperRef{capability}{capability} facilitated by the \GlossaryHyperRef{component-hardware}{hardware components} of a \GlossaryHyperRef{device}{device}.
See Section \ref{sec:concepts:device}.

\GlossaryEntry{capability-emergent}{Capability, Emergent}
A \GlossaryHyperRef{capability}{capability} facilitated by the formation of a \GlossaryHyperRef{system-of-systems}{system-of-systems}.
See Section \ref{sec:concepts:sos}.

\GlossaryEntry{capability-system}{Capability, System}
A \GlossaryHyperRef{capability}{capability} facilitated by the \GlossaryHyperRef{component-software}{software components} of a \GlossaryHyperRef{system}{system}.
See Section \ref{sec:concepts:system}.

\GlossaryEntry{cloud}{Cloud}
A \GlossaryHyperRef{boundary}{bounded} \GlossaryHyperRef{system-of-systems}{system-of-systems} able to independently execute given tasks through the use of a pool of \GlossaryHyperRef{resource}{resources}.
See Section \ref{sec:concepts:cloud}.

\GlossaryEntry{cloud-local}{Cloud, Local}
A \GlossaryHyperRef{cloud}{cloud} \GlossaryHyperRef{boundary-local}{bound to a physical location} due to its acting on or producing \GlossaryHyperRef{resource-local}{local resources}.
See Section \ref{sec:concepts:cloud}.

	\GlossaryNote{IoTA:AF}{
		provides an introduction to the local cloud concept in its second chapter, as well as an architectural definition in its third chapter.
		The following is an excerpt from the introduction:
		\begin{quote}
		The local cloud concept takes the view that specific geographically local automation tasks should be encapsulated and protected.
		These tasks have strong requirements on real time, ease of engineering, operation and maintenance, and system security and safety.
		The local cloud idea is to let the local cloud include the devices and systems required to perform the desired automation tasks, thus providing a local ``room'' which can be protected from outside activities.
		In other words, the cloud will provide a boundary to the open internet, thus aiming to protect the internal of the local cloud from the open internet.
		\end{quote}
		The third chapter contains the following:
		\begin{quote}
		In the Arrowhead Framework context a local cloud is defined as a self-contained network with the three mandatory core systems deployed and at least one application system deployed [...]
		\end{quote}
		Both of these descriptions are practical, in the sense that they emphasize engineering aspects.
		As this document is a reference model, engineering aspects are out of scope.
		The more general terms ``geographically local'', ``room'' and ``boundary'' clearly highlight the physicality of the local cloud itself, while the depiction of ``devices'' performing ``automation tasks'' makes it apparent that some kind of physical activity is involved, such as manufacturing.
		Finally, the local cloud being ``encapsulated'', ``protected'' and ``self-contained'' indicates that it is understood to exhibit a degree of independence with respect to the tasks it is given, which we expect all kinds of clouds to exhibit.
		Our definition should be interpreted as a summation of these characteristics.
	}

\GlossaryEntry{cloud-local-automation}{Cloud, Local Automation}
Prefer \GlossaryNameRef{cloud-local}.
See Section \ref{sec:concepts:cloud}.

\GlossaryEntry{cloud-virtual}{Cloud, Virtual}
A \GlossaryHyperRef{cloud}{cloud} \GlossaryHyperRef{boundary-virtual}{unbound by physical location} by only acting on or producing \GlossaryHyperRef{resource-virtual}{virtual resources}.
See Section \ref{sec:concepts:cloud}.

\GlossaryEntry{communication}{Communication}
The activity of sending and/or receiving \GlossaryHyperRef{message}{messages}.

\GlossaryEntry{communication-service-oriented}{Communication, Service-Oriented}
\GlossaryHyperRef{communication}{Communication} \GlossaryHyperRef{description}{described} in terms of the \GlossaryHyperRef{provider-service}{provision} and \GlossaryHyperRef{consumer-service}{consumption} of \GlossaryHyperRef{service}{services}.

\GlossaryEntry{component}{Component}
An \GlossaryHyperRef{artifact}{artifact} that can be part of another artifact and contribute to it facilitating its \GlossaryHyperRef{capability}{capabilities}.

	\GlossaryNote{Note 1}{
		The term ``component'' should not be used to refer to a system being a constituent of a \GlossaryHyperRef{system-of-systems}{system-of-systems}.
		Such a system should rather be referred to as being a \GlossaryHyperRef{subsystem}{subsystem}.
	}

	\GlossaryNote{RAMI4.0}{
		makes no practical distinction between components and \GlossaryHyperRef{system}{systems}.
		We approach something akin to a practical distinction by only defining \GlossaryHyperRef{device}{devices} and systems as having components.
		We do not, however, forbid other, more generic, uses of the word.
	}

\GlossaryEntry{component-hardware}{Component, Hardware}
A physical \GlossaryHyperRef{component}{component} that may be part of a \GlossaryHyperRef{device}{device}.
See Section \ref{sec:concepts:device}.

\GlossaryEntry{component-software}{Component, Software}
A \GlossaryHyperRef{virtual}{virtual} \GlossaryHyperRef{component}{component} that may be part of a \GlossaryHyperRef{system}{system}.
See Section \ref{sec:concepts:system}.

\GlossaryEntry{component-virtual-hardware}{Component, Virtual Hardware}
A \GlossaryHyperRef{component-software}{software component} that represents what normally would be a \GlossaryHyperRef{component-hardware}{hardware component}.

\GlossaryEntry{compress}{Compress}
To \GlossaryHyperRef{encode}{encode} \GlossaryHyperRef{data}{data} from its original form to a space-efficient form.
See Section \ref{sec:concepts:encoding:compression}.

\GlossaryEntry[(noun)]{compression}{Compression}
An \GlossaryHyperRef{encoding}{encoding} that \GlossaryHyperRef{compress}{compresses} \GlossaryHyperRef{data}{data} into a space-efficient form and \GlossaryHyperRef{decompress}{decompresses} it back to its original form.
See Section \ref{sec:concepts:encoding:compression}.

\GlossaryEntry{concept}{Concept}
The description of a generic or abstract idea.

\GlossaryEntry{configuration}{Configuration}
A set of changeable \GlossaryHyperRef{attribute}{attributes} that directly influence how a \GlossaryHyperRef{system}{system} exercises its \GlossaryHyperRef{capability-system}{capabilities}.

\GlossaryEntry{configure}{Configure}
To update a \GlossaryHyperRef{configuration}{configuration}.

\GlossaryEntry{connection}{Connection}
An active medium through which attached \GlossaryHyperRef{interface}{interfaces} can \GlossaryHyperRef{communication}{communicate}.

\GlossaryEntry{constraint}{Constraint}
A \GlossaryHyperRef{attribute}{attribute} that imposes constraints, or limits, on an \GlossaryHyperRef{entity}{entity} or \GlossaryHyperRef{relationship}{relationship}.

	\GlossaryNote{Note 1}{
		The presence of constraints enable \GlossaryHyperRef{validation}{validation}.
	}

	\GlossaryNote{Note 2}{
		Perhaps a bit counterintuitively, a constraint \textit{adds} information to its target by reducing the ways in which it could be realized.
	}

\GlossaryEntry{constraint-policy}{Constraint, Policy}
A \GlossaryHyperRef{constraint}{constraint} imposed by a \GlossaryHyperRef{policy}{policy}.
See Section \ref{sec:concepts:policy}.

\GlossaryEntry{constraint-profile}{Constraint,Profile}
A \GlossaryHyperRef{constraint}{constraint} imposed by a \GlossaryHyperRef{profile}{profile}.
See Section \ref{sec:concepts:profile}.

\GlossaryEntry{consumer-service}{Consumer, Service}
A \GlossaryHyperRef{system}{system} currently \GlossaryHyperRef{consumption-service}{consuming} a \GlossaryHyperRef{service}{service} by sending a \GlossaryHyperRef{message}{message} to one of its \GlossaryHyperRef{operation}{operations}.

	\GlossaryNote{Note 1}{
		The term may also be used to refer to a \GlossaryHyperRef{stakeholder}{stakeholder}, in which case the stakeholder must be interpreted as if consuming services via systems.
	}

	\GlossaryNote{SOA-RM}{
		defines a service consumer as ``an entity which seeks to satisfy a particular need through the use [of] capabilities offered by means of a service''.
		We require that the one consuming the service is (1) a \GlossaryHyperRef{system}{system} rather than just any \GlossaryHyperRef{entity}{entity}, as well as (2) that the \GlossaryHyperRef{capability-system}{capabilities} of the consumed service be exercised by invoking a function.
	}

\GlossaryEntry{consumption-service}{Consumption, Service}
The act of consuming a \GlossaryHyperRef{service}{service} by sending a \GlossaryHyperRef{message}{message} to one of its \GlossaryHyperRef{operation}{operations}.
See \GlossaryNameRef{consumer-service}.

\GlossaryEntry{data}{Data}
A sequence of \GlossaryHyperRef{datum}{datums} recording a set of \GlossaryHyperRef{description}{descriptions} via the structure superimposed by a \GlossaryHyperRef{type-data}{data type}.

	\GlossaryNote{Note 1}{
		Let us assume that some data is going to be sent to a drilling machine.
		The type associated with the data requires that it always consists of 8 bits, organized such that the first 4 bits indicate the speed of drilling in multiples of 100 rotations per minute, while the latter 4 determine how much to lower the drill in multiples of 5 millimeters.
		A \GlossaryHyperRef{state}{state} that could be expressed with those 8 bits is \texttt{0100 1101}.
		If each of the two sequences of 4 bits is treated as a big-endian integer with base 2, they record $4$ and $13$ in decimal notation.
		This would indicate that the drill should spin at $4 * 100 = 400$ rotations per minute and be lowered $13 * 5 = 65$ millimeters.
	}

	\GlossaryNote{Note 2}{
		Without knowledge of the types and context associated with some data, that data cannot be interpreted.
	}

\GlossaryEntry{datum}{Datum}
A variable expressing one out of a set of possible values.
See also \GlossaryNameRef{state}.

	\GlossaryNote{Note 1}{
		A familiar example of a datum may be the bit, or \textit{binary digit}.
		Its possible set of symbols is $\{0, 1\}$.
	}

\GlossaryEntry{decode}{Decode}
The act of transforming \GlossaryHyperRef{data}{data} from being expressed in a \GlossaryHyperRef{encoding}{encoding} suitable for transmission or storage to another encoding suitable for interpretation.

	\GlossaryNote{Note 1}{
		Decoding is the reverse of \GlossaryHyperRef{encode}{encoding}.
	}

	\GlossaryNote{Note 2}{
		The term can also be used to express the act of a human interpreting data.
	}

\GlossaryEntry{decoder}{Decoder}
An \GlossaryHyperRef{entity}{entity} capable of \GlossaryHyperRef{decode}{decoding} \GlossaryHyperRef{data}{data}.

\GlossaryEntry{decompress}{Decompress}
To \GlossaryHyperRef{decode}{decode} \GlossaryHyperRef{data}{data} from a space-efficient form to its original form.
See Section \ref{sec:concepts:encoding:compression}.

\GlossaryEntry{decrypt}{Decrypt}
To \GlossaryHyperRef{decode}{decode} \GlossaryHyperRef{data}{data} from an undecipherable, or \textit{cipher text}, form to its original form.
See Section \ref{sec:concepts:encoding:encryption}.

\GlossaryEntry{description}{Description}
Facts about an \GlossaryHyperRef{entity}{entity} or \GlossaryHyperRef{entity-class-of}{class of entities}, expressed in the form of a \GlossaryHyperRef{model}{model}, a text, or both.

\GlossaryEntry[(noun)]{design}{Design}
Every document, \GlossaryHyperRef{model}{model} and other record \GlossaryHyperRef{description}{describing} how a certain \GlossaryHyperRef{artifact}{artifact} can be \GlossaryHyperRef{implementation}{implemented}.

\GlossaryEntry[(verb)]{design-verb}{Design}
The activity of producing \GlossaryHyperRef{design}{designs}.

\GlossaryEntry{developer}{Developer}
A \GlossaryHyperRef{stakeholder}{stakeholder} developing the \GlossaryHyperRef{component}{components} that make up \GlossaryHyperRef{device}{devices} and/or \GlossaryHyperRef{system}{systems}.
See Section \ref{sec:concepts:stakeholder}.

\GlossaryEntry{device}{Device}
A physical \GlossaryHyperRef{entity}{entity} made from \GlossaryHyperRef{component-hardware}{hardware components} with the significant \GlossaryHyperRef{capability}{capability} of being able to \GlossaryHyperRef{hosting-system}{host systems}.
See Section \ref{sec:concepts:device}.

	\GlossaryNote{IoTA:AF}{
		defines device as ``a piece of equipment, machine, hardware, etc. with computational, memory and communication capabilities which hosts one or several Arrowhead Framework systems and can be bootstrapped in an Arrowhead local cloud''.
		The definition provided here should be interpreted as being equivalent.
	}

\GlossaryEntry{device-connected}{Device, Connected}
A \GlossaryHyperRef{device}{device} that is \GlossaryHyperRef{connection}{connected} to at least one other device via their \GlossaryHyperRef{interface-network}{network interfaces}, enabling them to \GlossaryHyperRef{communication}{communicate}.

\GlossaryEntry{device-end}{Device, End}
A \GlossaryHyperRef{device-connected}{connected device} being the intended recipient of a \GlossaryHyperRef{message}{message}.

\GlossaryEntry{device-human-interface}{Device, Human Interface}
A \GlossaryHyperRef{device}{device} with sensors and actuators that together make up a \GlossaryHyperRef{interface-human}{human interface}.

\GlossaryEntry{device-intermediary}{Device, Intermediary}
A \GlossaryHyperRef{device-connected}{connected device} that receives and forwards \GlossaryHyperRef{message}{messages} toward \GlossaryHyperRef{device-end}{end devices}.

\GlossaryEntry{device-virtual}{Device, Virtual}
A \GlossaryHyperRef{device}{device} that exists only \GlossaryHyperRef{virtual}{virtually}.
Examples of virtual devices can be application containers, virtual machines or emulated machines.

\GlossaryEntry{domain-problem}{Domain, Problem}
All aspects, known and unknown, that influence the potential for solving a particular problem.

\GlossaryNote{Note 1}{
	When you \textit{address} a problem domain, you provide a solution to its problem that accounts for all of its known aspects.
}

\GlossaryNote{Note 2}{
	The problem domain of the \GlossaryHyperRef{framework-arrowhead}{Arrowhead framework} consists of all aspects, known and unknown, that influence the potential for computer systems to exchange the information they need to execute tasks they have been assigned.
	Addressing its problem domain means that software specifications or implementations are produced that describe or handle such information exchanges, respectively.
}

\GlossaryEntry{encode}{Encode}
The act of transforming \GlossaryHyperRef{data}{data} from being expressed in a \GlossaryHyperRef{encoding}{encoding} suitable for interpretation to another encoding suitable for transmission or storage.

	\GlossaryNote{Note 1}{
		Encoding is the reverse of \GlossaryHyperRef{decode}{decoding}.
	}

	\GlossaryNote{Note 2}{
		The term can also be used to express the act of a human recording data.
	}

\GlossaryEntry{encoder}{Encoder}
An \GlossaryHyperRef{entity}{entity} capable of \GlossaryHyperRef{encode}{encoding} \GlossaryHyperRef{data}{data}.

\GlossaryEntry[(noun)]{encoding}{Encoding}
A \GlossaryHyperRef{type-data}{data type} used to structure and interpret certain \GlossaryHyperRef{data}{data}.
See Section \ref{sec:concepts:encoding}.

\GlossaryEntry{encrypt}{Encrypt}
To \GlossaryHyperRef{encode}{encode} \GlossaryHyperRef{data}{data} from its original form to an undecipherable, or \textit{cipher text}, form.
See Section \ref{sec:concepts:encoding:encryption}.

\GlossaryEntry[(noun)]{encryption}{Encryption}
An \GlossaryHyperRef{encoding}{encoding} that \GlossaryHyperRef{encrypt}{encrypts} \GlossaryHyperRef{data}{data} into an undecipherable, or \textit{cipher text}, form and \GlossaryHyperRef{decrypt}{decrypts} it back to its original, or \textit{clear text}, form.
See Section \ref{sec:concepts:encoding:encryption}.

\GlossaryEntry{entity}{Entity}
An \GlossaryHyperRef{artifact}{artifact} with an \GlossaryHyperRef{identity}{identity}, allowing for it to be distinguished from all other artifacts.
See Section \ref{sec:concepts:entity}.

	\GlossaryNote{Note 1}{
		An entity being uniquely identifiable does not necessarily mean that it is associated with a certificate or \GlossaryHyperRef{identifier}{identifier}.
		It only means that a \GlossaryHyperRef{description}{description} can be rendered that unambiguously refers to the entity in question.
	}

	\GlossaryNote{SOA-RM}{
		mentions the word ``entity'' nine times, but provides no explicit definition.
		We assume their definition to match that of a regular English dictionary, such as ``something that has separate and distinct existence and objective or conceptual reality'' \cite{webster2021entity}.
		Our definition should be interpreted as being equivalent.
	}

	\GlossaryNote{RAMI4.0}{
		defines entity as an ``uniquely identifiable object which is administered in the information world due to its importance''.
		Our definition should be interpreted as being equivalent.
	}

\GlossaryEntry{entity-class-of}{Entity, Class of}
A set of \GlossaryHyperRef{entity}{entities} that share a common \GlossaryHyperRef{attribute}{attribute}.

\GlossaryEntry{framework}{Framework}
A set of ideas and software artifacts that frame and address a problem domain of a certain community of \GlossaryHyperRef{stakeholder}{stakeholders}.
See Section \ref{sec:overview}.

	\GlossaryNote{ISO42010}{
	 	defines architecture framework as ``conventions, principles and practices for the description of architectures established within a specific domain of application and/or community of stakeholders''.
		Our definition of \GlossaryHyperRef{framework-idea}{\textit{framework of ideas}} should be interpreted as being compatible with that of ISO42010.
	}

	\GlossaryNote{SOA-RM}{
		defines framework as ``a set of assumptions, concepts, values, and practices that constitutes a way of viewing the current environment''.
		Our definition of \GlossaryHyperRef{framework-idea}{\textit{framework of ideas}} should be interpreted as being equivalent to that of SOA-RM.
	}

\GlossaryEntry{framework-architecture}{Framework, Architecture}
Prefer \GlossaryNameRef{framework}.

\GlossaryEntry{framework-arrowhead}{Framework, Arrowhead}
Either of the \GlossaryHyperRef{framework}{framework of ideas} and the \GlossaryHyperRef{framework-software}{framework of software} maintained by the \GlossaryHyperRef{project-eclipse-arrowhead}{Arrowhead project}.
See Section \ref{sec:overview}.

\GlossaryEntry{framework-idea}{Framework, Idea}
A set of assumptions, concepts, values and practices that frame a certain problem domain.
See Section \ref{sec:overview}.

\GlossaryEntry{framework-software}{Framework, Software}
A set of software specifications, \GlossaryHyperRef{implementation-software}{implementations} and other \GlossaryHyperRef{artifact}{artifacts} meant to help address the problem domain of a certain \GlossaryHyperRef{framework}{framework}.
See Section \ref{sec:overview}.

\GlossaryEntry{function}{Function}
A conceptual mathematical construct that transforms given input values into output values.

	\GlossaryNote{Note 1}{
		Most functions can be \GlossaryHyperRef{implementation}{implemented} as \GlossaryHyperRef{implementation-software}{software}, in which case they can also be referred to as \GlossaryHyperRef{function-computer}{computer functions}.
	}

\GlossaryEntry{function-computer}{Function, Computer}
A sequence of instructions executed by one or more \GlossaryHyperRef{unit-compute}{compute units} in response to an invocation with a set of arguments.
The result of executing such a function may be that the \GlossaryHyperRef{state}{state} of the computer is updated and/or that the function returns a value.
See \GlossaryNameRef{function}.

\GlossaryEntry{function-predicate}{Function, Predicate}
A \GlossaryHyperRef{function}{function} whose output value must be a \textit{Boolean variable}.
An output value of \textit{true} indicates that the function is \textit{satisfied}, while an output value of false indicates it being \textit{violated}.

	\GlossaryNote{Note 1}{
		A \textit{Boolean variable} can only be either of the two mentioned values, \textit{true} and \textit{false}.
	}

\GlossaryEntry[(adjective)]{hardware-adjective}{Hardware}
The property of being physical, as opposed to being \GlossaryHyperRef{virtual}{virtual}.
See \GlossaryNameRef{software-adjective}.

\GlossaryEntry[(noun)]{hardware}{Hardware}
A physical \GlossaryHyperRef{artifact}{artifact}.
See \GlossaryNameRef{hardware}.

\GlossaryEntry{hid}{HID}
Abbreviation for \GlossaryNameRef{device-human-interface}.

\GlossaryEntry{hosting-system}{Hosting, System}
The act of making a \GlossaryHyperRef{service}{service} available for \GlossaryHyperRef{consumption-service}{consumption} by running its \GlossaryHyperRef{software}{software} and giving that software access to a \GlossaryHyperRef{network}{network}.

\GlossaryEntry{human}{Human}
Prefer \GlossaryNameRef{person}.

\GlossaryEntry{identifiable}{Identifiable}
The property of being possible to distinguish a certain \GlossaryHyperRef{artifact}{artifact} from all other artifacts.

	\GlossaryNote{Note 1}{
		Being identifiable is the same as being an \GlossaryHyperRef{entity}{entity}.
	}

\GlossaryEntry{identification}{Identification}
The process through which an \GlossaryHyperRef{entity}{entity} determines and/or verifies the \GlossaryHyperRef{identity}{identity} of another entity. 

\GlossaryEntry{identifier}{Identifier}
\GlossaryHyperRef{data}{Data} associated with an \GlossaryHyperRef{entity}{entity} that allows for it to be \GlossaryHyperRef{identification}{identified}.

\GlossaryEntry{identity}{Identity}
The aspect or aspects, such as \GlossaryHyperRef{identifier}{identifiers}, that makes an \GlossaryHyperRef{entity}{entity} distinct from all other entities.

\GlossaryEntry{image-software}{Image, Software}
A \GlossaryHyperRef{data}{data} \GlossaryHyperRef{artifact}{artifact} comprised of instructions that could be executed by a compatible \GlossaryHyperRef{unit-compute}{compute unit} or virtual machine.

\GlossaryEntry{implementation}{Implementation}
The realization of a \GlossaryHyperRef{design}{design} as a set of \GlossaryHyperRef{artifact}{artifacts}.

\GlossaryEntry{implementation-software}{Implementation, Software}
An \GlossaryHyperRef{implementation}{implementation} comprised of \GlossaryHyperRef{software}{software} \GlossaryHyperRef{artifact}{artifacts}.

	\GlossaryNote{Note 1}{
		The term may also be used to refer to all software artifacts part of an implementation.
	}

\GlossaryEntry{implementation-hardware}{Implementation, Hardware}
An \GlossaryHyperRef{implementation}{implementation} comprised of \GlossaryHyperRef{hardware}{hardware} \GlossaryHyperRef{artifact}{artifacts}.

	\GlossaryNote{Note 1}{
		The term may also be used to refer to all hardware artifacts part of an implementation.
	}

\GlossaryEntry{industry40}{Industry 4.0}
The fourth industrial paradigm, primarily characterized by high degrees of computerization, digitization and interconnectivity.
See also \cite{adolphs2016reference}.

\GlossaryEntry{instance-software}{Instance, Software}
A \GlossaryHyperRef{image-software}{software image} currently being executed by a \GlossaryHyperRef{unit-compute}{compute unit} or virtual machine.

	\GlossaryNote{Note 1}{
		The same image can be executed any number of times, even in parallel.
		Each execution of that image is its own instance, distinct from all other instances.
	}

\GlossaryEntry{interconnection}{Interconnection}
A \GlossaryHyperRef{connection}{connection} that passes through one or more \GlossaryHyperRef{device-intermediary}{intermediary devices}.

\GlossaryEntry{interface}{Interface}
A \GlossaryHyperRef{boundary}{boundary} where \GlossaryHyperRef{message}{messages} of certain \GlossaryHyperRef{protocol}{protocols} can pass between a \GlossaryHyperRef{connection}{connection} and an \GlossaryHyperRef{entity}{entity}, between two entities, or between an entity and a \GlossaryHyperRef{person}.
See Section \ref{sec:concepts:interface}.

\GlossaryEntry{interface-human}{Interface, Human}
An \GlossaryHyperRef{interface}{interface} through which a \GlossaryHyperRef{person} may send and/or receive \GlossaryHyperRef{message}{messages} to/from an \GlossaryHyperRef{entity}{entity}.

\GlossaryEntry{interface-network}{Interface, Network}
An \GlossaryHyperRef{interface}{interface} through which a \GlossaryHyperRef{device}{device} could communicate with other devices, or with itself, over a \GlossaryHyperRef{network}{network}.

\GlossaryEntry{interface-operation}{Interface, Operation}
An \GlossaryHyperRef{interface}{interface} through which a certain \GlossaryHyperRef{operation}{operation} of some \GlossaryHyperRef{service}{service} can be \GlossaryHyperRef{consumer-service}{consumed}.

\GlossaryEntry{interface-service}{Interface, Service}
An \GlossaryHyperRef{interface}{interface} through which a certain \GlossaryHyperRef{service}{service} can be \GlossaryHyperRef{consumer-service}{consumed}.

	\GlossaryNote{Note 1}{
		Consuming a service requires that \GlossaryHyperRef{message}{messages} be passed from its \GlossaryHyperRef{device}{device} to its \GlossaryHyperRef{system}{system}, and then from its system to the service itself.
		As the \GlossaryHyperRef{software}{software} making up the service is owned by the system, it is the system that is understood to produce any responses.
		Those are passed on via its device.
	}

	\GlossaryNote{SOA-RM}{
		defines service interface as ``the means by which the underlying capabilities of a service are accessed''.
		Our definition should be interpreted as being equivalent.
	}

\GlossaryEntry{interface-system}{Interface, System}
An \GlossaryHyperRef{interface}{interface} through which a \GlossaryHyperRef{system}{system} may send and/or receive \GlossaryHyperRef{message}{messages} via its \GlossaryHyperRef{hosting-system}{hosting} \GlossaryHyperRef{device}{device}.

\GlossaryEntry{kind-model}{Kind, Model}
A \GlossaryHyperRef{description}{description} of how to produce a certain kind of \GlossaryHyperRef{model}{model}.

	\GlossaryNote{ISO42010}{
		defines model kind as ``conventions for a type of modelling''.
		It also provides ``data flow diagrams, class diagrams, Petri nets, balance sheets, organization charts and state transition models'' as examples of what model kinds could establish.
		Our definition must be considered as being either equivalent or incorrect.
	}

\GlossaryEntry{language-architecture-description}{Language, Architecture Description}
A formal language in which \GlossaryHyperRef{architecture}{architectures} can be \GlossaryHyperRef{description}{described}.

\GlossaryEntry{maintainer}{Maintainer}
A \GlossaryHyperRef{stakeholder}{stakeholder} involved in maintaining \GlossaryHyperRef{device}{devices} and \GlossaryHyperRef{system}{systems}, primarily by repairing and upgrading devices and updating system software.
See Section \ref{sec:concepts:stakeholder}.

\GlossaryEntry{message}{Message}
\GlossaryHyperRef{data}{Data} describing how to invoke a certain \GlossaryHyperRef{operation}{service operation}.
See Section \ref{sec:concepts:message}.

\GlossaryEntry{message-error}{Message, Error}
A \GlossaryHyperRef{message}{message} indicating why the request expressed by some other message could not be satisfied.

\GlossaryEntry{message-forbidden}{Message, Forbidden}
A \GlossaryHyperRef{message}{message} that fails to satisfy a \GlossaryHyperRef{policy}{policy} of concern and, therefore, will not be executed.
See Section \ref{sec:concepts:policy}.

\GlossaryEntry{message-inbound}{Message, Inbound}
A \GlossaryHyperRef{message}{message} that it currently being \GlossaryHyperRef{routing}{routed} away from a \GlossaryHyperRef{interface-network}{network interface} and towards an \GlossaryHyperRef{operation}{operation}.
See Section \ref{sec:concepts:interface}.

\GlossaryEntry{message-invalid}{Message, Invalid}
A \GlossaryHyperRef{message}{message} that fails to satisfy a \GlossaryHyperRef{protocol}{protocol} of concern and, therefore, will not be executed.
See Section \ref{sec:concepts:protocol}.

\GlossaryEntry{message-outbound}{Message, Outbound}
A \GlossaryHyperRef{message}{message} that it currently being sent towards a \GlossaryHyperRef{interface-network}{network interface} as part of its journey to an \GlossaryHyperRef{operation}{operation}.
See Section \ref{sec:concepts:interface}.

\GlossaryEntry{message-permitted}{Message, Permitted}
A \GlossaryHyperRef{message}{message} that does satisfy a \GlossaryHyperRef{policy}{policy} of concern and, therefore, will be executed if all other policies are also satisfied.
See Section \ref{sec:concepts:policy}.

\GlossaryEntry{message-valid}{Message, Valid}
A \GlossaryHyperRef{message}{message} that does satisfy a \GlossaryHyperRef{protocol}{protocol} of concern and, therefore, will be executed if it is \GlossaryHyperRef{message-permitted}{permitted}.
See Section \ref{sec:concepts:protocol}.

\GlossaryEntry{metadata}{Metadata}
\GlossaryHyperRef{data}{Data} \GlossaryHyperRef{description}{describing} other data.

\GlossaryEntry{metamodel}{Metamodel}
A basic set of \GlossaryHyperRef{model}{model} constructs that can be extended by other models.

	\GlossaryNote{Note 1}{
		A metamodel can be thought of as a general language in which more specific models can be expressed.
		Just as a given sentence in a human language can be determined to be valid or invalid, a model can also be verified to be correct in relation to its metamodels, if any.
	}

	\GlossaryNote{ISO42010}{
		defines metamodel as what ``presents  the  [architectural description] elements that comprise the vocabulary of a \GlossaryHyperRef{kind-model}{model kind}''.
		It further adds that a ``metamodel should present entities[,] attributes[,] relationships [and] constraints''.
		Our definition must be considered as being either equivalent or incorrect.
	}

\GlossaryEntry{model}{Model}
A representation of facts in the form of a graph, consisting of \GlossaryHyperRef{entity}{entities}, \GlossaryHyperRef{relationship}{relationships} and \GlossaryHyperRef{attribute}{attributes}.

	\GlossaryNote{Note 1}{
		Models can be expressed or recorded in many ways, including as visual diagrams, spoken words, text and binary data.
	}

	\GlossaryNote{Note 2}{
		Models can be human-readable, machine-readable, or both.
	}

\GlossaryEntry{network}{Network}
A set of two or more \GlossaryHyperRef{device-end}{end devices}, \GlossaryHyperRef{connection}{connected} in such a manner that any \GlossaryHyperRef{system}{systems} they \GlossaryHyperRef{hosting-system}{host} are able to \GlossaryHyperRef{communication}{communicate}.
See Section \ref{sec:concepts:network}.

\GlossaryEntry{operation}{Operation}
A \GlossaryHyperRef{component-software}{component} of a \GlossaryHyperRef{service}{service} that handles given \GlossaryHyperRef{message}{messages} by using the \GlossaryHyperRef{capability-system}{capabilities} of its \GlossaryHyperRef{system}{system}.
See Section \ref{sec:concepts:operation}.

\GlossaryEntry{operation-exposed}{Operation, Exposed}
An \GlossaryHyperRef{operation}{operation} part of a \GlossaryHyperRef{service}{service} that is currently being \GlossaryHyperRef{provision-service}{provided} by a \GlossaryHyperRef{system}{system}.
See Section \ref{sec:concepts:operation}.

\GlossaryEntry{operator}{Operator}
A \GlossaryHyperRef{stakeholder}{stakeholder} responsible for the \GlossaryHyperRef{configure}{configuration} and oversight of \GlossaryHyperRef{system}{systems} and the \GlossaryHyperRef{resource}{resources} those systems manage.
See Section \ref{sec:concepts:stakeholder}.

\GlossaryEntry{organization}{Organization}
A \GlossaryHyperRef{stakeholder}{stakeholder} comprised of an organized body of other stakeholders and/or other persons.

\GlossaryEntry{owner}{Owner}
A \GlossaryHyperRef{stakeholder}{stakeholder} that owns significant \GlossaryHyperRef{resource}{resources} and/or other \GlossaryHyperRef{artifact}{artifacts}.
See Section \ref{sec:concepts:stakeholder}.

\GlossaryEntry{person}{Person}
A human being.

\GlossaryEntry{policy}{Policy}
A set of \GlossaryHyperRef{constraint}{constraints}, of any nature, that must be satisfied by all \GlossaryHyperRef{message}{messages} passed on by an \GlossaryHyperRef{interface}{interface}.
See Section \ref{sec:concepts:policy}.

	\GlossaryNote{SOA-RM}{
		defines policy as ``a statement of obligations, constraints or other conditions of use of an owned entity as defined by a participant''.
		Our definition should be interpreted as being equivalent.
	}

\GlossaryEntry{policy-message}{Policy, Message}
Prefer \GlossaryNameRef{policy}.

\GlossaryEntry{practice}{Practice}
A particular way in which some task or activity is carried out.

	\GlossaryNote{Note 1}{
		The \GlossaryHyperRef{framework-arrowhead}{Arrowhead framework} is, among other things, concerned with shaping practices surrounding the development and maintenance of \GlossaryHyperRef{system-of-systems}{systems-of-systems}.
	}

\GlossaryEntry{profile}{Profile}
A set of \GlossaryHyperRef{constraint}{constraints} superimposed on a \GlossaryHyperRef{protocol}{protocol}.
See Section \ref{sec:concepts:profile}.

	\GlossaryNote{Note 1}{
		A profile \textit{never} introduces more \GlossaryHyperRef{message}{messages} or \GlossaryHyperRef{state}{states} to a protocol.
		It adds constraints to existing messages and states.
	}

	\GlossaryNote{Note 2}{
		A profile could, for example, introduce an authentication mechanism to a protocol by requiring that a certain type of token be included in each message.
		It could demand that a certain protocol be extended, or that a particular kind of \GlossaryHyperRef{encoding}{encoding} be used for message bodies, and so on.
	}

\GlossaryEntry{profile-protocol}{Profile, Protocol}
Prefer \GlossaryNameRef{profile}.

\GlossaryEntry{project-eclipse-arrowhead}{Project, Eclipse Arrowhead}
The effort of the \GlossaryHyperRef{arrowhead}{Arrowhead} community to increase the utility of the \GlossaryHyperRef{framework-arrowhead}{Arrowhead framework}.

\GlossaryEntry{protocol}{Protocol}
A \GlossaryHyperRef{model}{model} of communication defined in terms of \GlossaryHyperRef{state}{states} and \GlossaryHyperRef{message}{messages}.
See Section \ref{sec:concepts:protocol}.

	\GlossaryNote{Note 1}{
		The states, if any, dictate the outcomes of sending certain messages.
		For example, let us assume that some state can be either \texttt{BUSY} or \texttt{READY}.
		If the former state would be the active when a certain message is received, the designated response could be an error message.
		If, however, the \texttt{READY} state would have been active, the state could be transitioned to the \texttt{BUSY} value and a success response be provided to the sender.
	}

\GlossaryEntry{protocol-extensible}{Protocol, Extensible}
A \GlossaryHyperRef{protocol}{protocol} allowing for \GlossaryHyperRef{subprotocol}{subprotocols} to be formulated in terms of its \GlossaryHyperRef{message}{messages}.

	\GlossaryNote{Note 1}{
		Every new message introduced by a subprotocol must be a \GlossaryHyperRef{validation}{valid} message of its \GlossaryHyperRef{superprotocol}{superprotocol}.
	}

	\GlossaryNote{Note 2}{
		Many of the currently prevalent protocols are designed with the intent of being extensible.
		For example, HTTP \cite{fielding2014hypertext} provides provisions for an extending protocol to define its own set of directory operations, to simultaneously support multiple \GlossaryHyperRef{encoding}{encodings}, and so on.
	}

	\GlossaryNote{Note 3}{
		As long as a given protocol provides at least one message whose contents can be arbitrary, a subprotocol can be produced.
		This means that even protocols not designed to be extended can, in some contexts, be meaningfully used to define subprotocols.
	}

\GlossaryEntry{protocol-network}{Protocol, Network}
A \GlossaryHyperRef{protocol}{protocol} implemented by an \GlossaryHyperRef{interface-network}{network interface}.
See Section \ref{sec:concepts:protocol}.

\GlossaryEntry{protocol-operation}{Protocol, Operation}
A \GlossaryHyperRef{protocol}{protocol} implemented by an \GlossaryHyperRef{interface-operation}{operation interface}.
See Section \ref{sec:concepts:protocol}.

	\GlossaryNote{Note 1}{
		An operation protocol is always an \GlossaryHyperRef{protocol-extensible}{extension} of a \GlossaryHyperRef{protocol-service}{service protocol}.
	}

\GlossaryEntry{protocol-service}{Protocol, Service}
A \GlossaryHyperRef{protocol}{protocol} implemented by a \GlossaryHyperRef{interface-service}{service interface}.
See Section \ref{sec:concepts:protocol}.

	\GlossaryNote{Note 1}{
		A service protocol is always an \GlossaryHyperRef{protocol-extensible}{extension} of a \GlossaryHyperRef{protocol-system}{system protocol}.
	}

\GlossaryEntry{protocol-system}{Protocol, System}
A \GlossaryHyperRef{protocol}{protocol} implemented by a \GlossaryHyperRef{interface-system}{system interface}.
See Section \ref{sec:concepts:protocol}.

	\GlossaryNote{Note 1}{
		A system protocol is always an \GlossaryHyperRef{protocol-extensible}{extension} of a \GlossaryHyperRef{protocol-network}{network protocol}.
	}

\GlossaryEntry{provider-service}{Provider, Service}
A \GlossaryHyperRef{system}{system} that makes \GlossaryHyperRef{service}{services} available for \GlossaryHyperRef{consumption-service}{consumption} to other systems.

	\GlossaryNote{Note 1}{
		If used to refer to a \GlossaryHyperRef{stakeholder}{stakeholder}, the term must be interpreted as if that stakeholder provides services via systems it controls.
	}

	\GlossaryNote{SOA-RM}{
		defines a service provider as ``an entity (person or organization) that offers the use of capabilities by means of a service''.
		Our definition is equivalent only if referring to a stakeholder as a service provider, as described in Note 1.
	}

\GlossaryEntry{provision-service}{Provision, Service}
The act of making \GlossaryHyperRef{service}{services} available for \GlossaryHyperRef{consumption-service}{consumption}.
See \GlossaryNameRef{provider-service}.

\GlossaryEntry{proxy}{Proxy}
An \GlossaryHyperRef{entity}{entity} representing the agenda or desires of another entity or \GlossaryHyperRef{stakeholder}{stakeholder}.

\GlossaryEntry{relationship}{Relationship}
A named uni-directional association between two \GlossaryHyperRef{model}{model} \GlossaryHyperRef{entity}{entities}.

\GlossaryEntry{researcher}{Researcher}
A \GlossaryHyperRef{stakeholder}{stakeholder} involved in the analysis or development of significant \GlossaryHyperRef{entity}{entities}, particularly with the ambition of facilitating \GlossaryHyperRef{attribute}{attributes} or use cases that cannot be realized without refining, extending or replacing those entities.
See Section \ref{sec:concepts:stakeholder}.

\GlossaryEntry{resource}{Resource}
An \GlossaryHyperRef{artifact}{artifact} that is of value to a \GlossaryHyperRef{stakeholder}{stakeholder} or of use to another artifact.

	\GlossaryNote{Note 1}{
		Any type of artifact can be a resource, which includes everything from \GlossaryHyperRef{resource-local}{local resources}, such as raw materials or \GlossaryHyperRef{device}{devices}, to \GlossaryHyperRef{resource-virtual}{virtual resources}, such as \GlossaryHyperRef{system}{systems} or \GlossaryHyperRef{data}{data}.
	}

	\GlossaryNote{Note 2}{
		An artifact stops be a resource when it is perceived as having no value or use, at which point it may be destroyed, recycled or sold to someone that does perceive it as a resource, for example.
	}

\GlossaryEntry{resource-local}{Resource, Local}
A \GlossaryHyperRef{resource}{resource} whose value or utility is inextricably tied to at least one physical \GlossaryHyperRef{attribute}{attribute}.

	\GlossaryNote{Note 1}{
		Examples of local resources could be raw materials, drills, pumps, power stations, or drones.
	}

\GlossaryEntry{resource-virtual}{Resource, Virtual}
A \GlossaryHyperRef{resource}{resource} whose value or utility is not derived from any physical \GlossaryHyperRef{attribute}{attribute}.

	\GlossaryNote{Note 1}{
		Examples of \GlossaryHyperRef{virtual}{virtual} resources could be compute, storage, or software-defined network utilities.
		While all of these resources are facilitated by physical entities, namely various types of computer equipment, they do not depend on any particular machines.
		They can be moved to different machines without loosing their value or utility.
	}

\GlossaryEntry{role}{Role}
An assignment, objective, or other responsibility, that makes a \GlossaryHyperRef{person}{person} or \GlossaryHyperRef{organization}{organization} into a \GlossaryHyperRef{stakeholder}{stakeholder}. 

\GlossaryEntry{role-stakeholder}{Role, Stakeholder}
Prefer \GlossaryNameRef{role}.

\GlossaryEntry{routing}{Routing}
The act of forwarding a \GlossaryHyperRef{message}{message} towards the \GlossaryHyperRef{operation}{service operation} it targets.

\GlossaryEntry{routing-message}{Routing, Message}
Prefer \GlossaryNameRef{routing}.

\GlossaryEntry{semantics}{Semantics}
A \GlossaryHyperRef{model}{model} used to interpret \GlossaryHyperRef{decode}{decoded} \GlossaryHyperRef{data}{data}, such that it meanings can be acted upon.
See Section \ref{sec:concepts:semantics}.

\GlossaryEntry{semantics-data}{Semantics, Data}
Prefer \GlossaryNameRef{semantics}.

\GlossaryEntry{semantics-message}{Semantics, Message}
The \GlossaryHyperRef{semantics}{semantics} of a \GlossaryHyperRef{message}{message}.

\GlossaryEntry{service}{Service}
A set of \GlossaryHyperRef{operation}{operations} that can be \GlossaryHyperRef{provision-service}{provided} by a \GlossaryHyperRef{system}{system} via one or more \GlossaryHyperRef{interface-service}{service interfaces}.
See Section \ref{sec:concepts:service}.

	\GlossaryNote{IoTA:AF}{
		defines a service as ``what [is] used to exchange information from a providing system to a consuming system''.
		It further adds that ``in a service, capabilities are grouped together if they share the same context''.
		The definition presented here should be interpreted as being compatible but more specific about how information is exchanged and \GlossaryHyperRef{capability}{capabilities} are exercised.
	}

	\GlossaryNote{SOA-RM}{
		defines a service as ``the means by which the needs of a consumer are brought together with the capabilities of a provider''.
		Our definition is more specific about how the \GlossaryHyperRef{capability-system}{capabilities} of a service are made available.
	}

	\GlossaryNote{RAMI4.0}{
		defines a service as ``separate scope of functions offered by an entity or organization via interfaces''.
		Given that our understanding of ``operation'' is compatible with the RAMI4.0 definition of ``function'', our definition of ``service'' should be considered as being equivalent.
	}

\GlossaryEntry[(adjective)]{software-adjective}{Software}
The property of being \GlossaryHyperRef{virtual}{virtual}, as opposed to being physical.
See \GlossaryNameRef{hardware-adjective}.

\GlossaryEntry[(noun)]{software}{Software}
A set of sequences of instructions that can be executed by a \GlossaryHyperRef{unit-compute}{compute unit}.

	\GlossaryNote{Note 1}{
		A software does not necessarily have to be expressed in the instruction set native to the compute unit expected to execute it.
		Virtual machines, interpreters and other utilities may be used to execute instructions, which means that our definition of `software'' may be more open-ended than what initially may seem to be the case.
	}

\GlossaryEntry{specification}{Specification}
A detailed \GlossaryHyperRef{description}{description}, outlining the design of some \GlossaryHyperRef{artifact}{artifact} of concern.

\GlossaryEntry{specification-software}{Specification, Software}
A \GlossaryHyperRef{specification}{specification} concerned only or primarily with \GlossaryHyperRef{software}{software}.

\GlossaryEntry{stake}{Stake}
Any type of engagement or commitment.

\GlossaryEntry{stakeholder}{Stakeholder}
A \GlossaryHyperRef{person}{person} or \GlossaryHyperRef{organization}{organization} with one or more \GlossaryHyperRef{role}{roles}, which gives that stakeholder at least one \GlossaryHyperRef{relationship}{relationship} to one \GlossaryHyperRef{artifact}{artifact}.
See Section \ref{sec:concepts:stakeholder}.

\GlossaryEntry{state}{State}
One out of all possible sequences of values that could be expressed by the \GlossaryHyperRef{datum}{datums} of some \GlossaryHyperRef{data}{data}.

	\GlossaryNote{Note 1}{
		If the data would consist of a sequence of bits, each of which can only have the values 0 and 1, a state becomes a pattern of zeroes and ones those bits could record.
		Given four bits, possible states could, for example, be \texttt{0010} and \texttt{1001}. 
	}

	\GlossaryNote{Note 2}{
		The term is often used as a wildcard for any kind of storage construct, including bit flags, state machines and graph databases.
	}

\GlossaryEntry{state-protocol}{State, Protocol}
The \GlossaryHyperRef{state}{state} of a \GlossaryHyperRef{protocol}{protocol} in active use, determining what \GlossaryHyperRef{message}{messages} it currently deems valid.
See Section \ref{sec:concepts:protocol}.

\GlossaryEntry{stub-service}{Stub, Service}
The means through which a \GlossaryHyperRef{system}{system} \GlossaryHyperRef{consumption-service}{consumes} a \GlossaryHyperRef{service}{service} of another system.
Every service stub has its own \GlossaryHyperRef{interface-operation}{operation interfaces}.
If one of those operation interfaces is provided with a \GlossaryHyperRef{message}{message}, the service stub will attempt to pass on that message to the service represented by the stub.
See Section \ref{sec:concepts:interface}.

\GlossaryEntry{subprotocol}{Subprotocol}
A \GlossaryHyperRef{protocol}{protocol} that is realized as an \GlossaryHyperRef{protocol-extensible}{extension} of another protocol.

\GlossaryEntry{subsystem}{Subsystem}
A \GlossaryHyperRef{system}{system} or \GlossaryHyperRef{system-of-systems}{system-of-systems} being a constituent of a larger system-of-systems.

\GlossaryEntry{superprotocol}{Superprotocol}
A \GlossaryHyperRef{protocol}{protocol} that is \GlossaryHyperRef{protocol-extensible}{extended} by another protocol.

\GlossaryEntry{supplier}{Supplier}
A \GlossaryHyperRef{stakeholder}{stakeholder} in the process of supplying, or considering to supply, \GlossaryHyperRef{artifact}{artifacts}, such as \GlossaryHyperRef{device}{devices} and \GlossaryHyperRef{system}{systems}, to an \GlossaryHyperRef{acquirer}{acquirer}.

\GlossaryEntry{system}{System}
A \GlossaryHyperRef{software}{software} \GlossaryHyperRef{entity}{entity} capable of \GlossaryHyperRef{provider-service}{providing services}, \GlossaryHyperRef{consumer-service}{consuming services}, or both.

	\GlossaryNote{IoTA:AF}{
		defines a system as ``what is providing and/or consuming services''.
		It further adds that ``a system can be the service provider of one or more services and at the same time the service consumer of one or more services''.
		The definition presented here should be interpreted as equivalent.
	}

\GlossaryEntry{system-automation}{System, Automation}
Any kind of system, compatible with the \GlossaryHyperRef{framework-arrowhead}{Arrowhead framework} or not, meant to facilitate some form of \GlossaryHyperRef{automation}{automation}.

\GlossaryEntry{system-of-clouds}{System-of-Clouds (SoCl)}
A set of at least two \GlossaryHyperRef{cloud}{clouds} that collaborate by at least one cloud \GlossaryHyperRef{consumer-service}{consuming} at least one \GlossaryHyperRef{service}{service} \GlossaryHyperRef{provider-service}{provided} by another cloud in the set.
See Section \ref{sec:concepts:soc}.

\GlossaryEntry{system-of-local-clouds}{System-of-Local-Clouds (SoLC)}
A \GlossaryHyperRef{system-of-clouds}{system-of-clouds} where every constituent \GlossaryHyperRef{cloud}{cloud} is a \GlossaryHyperRef{cloud-local}{local cloud}.
See Section \ref{sec:concepts:soc}.

\GlossaryEntry{system-of-systems}{System-of-Systems (SoS)}
A set of at least two \GlossaryHyperRef{system}{systems} that collaborate by at least one system \GlossaryHyperRef{consumer-service}{consuming} at least one \GlossaryHyperRef{service}{service} \GlossaryHyperRef{provider-service}{provided} by another system in the set.
See Section \ref{sec:concepts:sos}.

	\GlossaryNote{IoTA:AF}{
		defines a system-of-systems as ``a set of systems, which [...] exchange information by means of services''.
		Our definition must be interpreted as being equivalent or as being incorrect.
	}

\GlossaryEntry{system-of-virtual-clouds}{System-of-Virtual-Clouds (SoVC)}
A \GlossaryHyperRef{system-of-clouds}{system-of-clouds} where every constituent \GlossaryHyperRef{cloud}{cloud} is a \GlossaryHyperRef{cloud-virtual}{virtual cloud}.
See Section \ref{sec:concepts:soc}.

\GlossaryEntry{system-opaque}{System, Opaque}
A \GlossaryHyperRef{system}{system} that is unable to either \GlossaryHyperRef{provider-service}{provide} or \GlossaryHyperRef{consumer-service}{consume} \GlossaryHyperRef{service}{services}.

\GlossaryEntry{system-supervisory}{System, Supervisory}
A \GlossaryHyperRef{system}{system} that is tasked with managing one or more \GlossaryHyperRef{resource}{resources} beyond its direct control.

	\GlossaryNote{Note 1}{
		All systems are managing the resources provided to them by their hosting \GlossaryHyperRef{device}{devices}, such as primary memory, compute time, and so on.
		This term is meant to capture the systems that are engaged in overseeing and/or managing resources beyond those directly provided.
		Examples of such scenarios could be a single system being responsible for provisioning other devices than its own, or a system using its robot arm to collect and handle raw materials.
	}

\GlossaryEntry{type}{Type}
A \GlossaryHyperRef{description}{description} of how datums are to be arranged to \GlossaryHyperRef{encode}{encode} certain facts.
See also \GlossaryNameRef{data}.

	\GlossaryNote{Note 1}{
		While this definition may seem foreign, it does capture how integer types, classes, enumerators and other types are used in the context of a programming language or \GlossaryHyperRef{encoding}{encoding}.
		In the end, all data are bits or other symbols.
		From our perspective, types serve to group those symbols and assign them meaning.
	}

	\GlossaryNote{Note 2}{
		A type provides only syntactic, or structural, information about data.
		While knowing the type used to code some data is required for its interpretation, contextual knowledge is also needed.
		For example, a type may specify a \texttt{name}, but it will not indicate when or why that name is useful.
		That information would have to be provided via documentation or some other means.
	}

\GlossaryEntry{type-data}{Type, Data}
Prefer \GlossaryNameRef{type}.

\GlossaryEntry{type-message}{Type, Message}
The \GlossaryHyperRef{type}{type} dictating the structure of the \GlossaryHyperRef{data}{data} in a \GlossaryHyperRef{message}{message}.

\GlossaryEntry{type-state}{Type, State}
The \GlossaryHyperRef{type}{type} specifying a set of possible \GlossaryHyperRef{state}{states} and transitions between them.

\GlossaryEntry{unit-compute}{Unit, Compute}
A \GlossaryHyperRef{component-hardware}{hardware component} able to execute \GlossaryHyperRef{software}{software} compatible with a certain instruction set.

\GlossaryEntry{unit-memory}{Unit, Memory}
A \GlossaryHyperRef{component-hardware}{hardware component} maintaining a set of changeable \GlossaryHyperRef{datum}{datums}, which are primarily useful for maintaining \GlossaryHyperRef{state}{states}.

\GlossaryEntry{user}{User}
A \GlossaryHyperRef{stakeholder}{stakeholder} taking, or trying to take, advantage of the end utility of a certain \GlossaryHyperRef{entity}{entity}.
See Section \ref{sec:concepts:stakeholder}.

	\GlossaryNote{Note 1}{
		The activity of \textit{using} an entity is not related to its coming into existence, maintenance, decommissioning, or any other peripheral activity.
		When a stakeholder uses an entity, that entity produces whatever end value it was designed to produce.
	}

\GlossaryEntry{validation}{Validation}
The process through which it is determined if a \GlossaryHyperRef{model}{model} satisfies a \GlossaryHyperRef{constraint}{constraint}.

\GlossaryEntry{value}{Value}
Something, such as principle or quality, \GlossaryHyperRef{assumption}{assumed} to be intrinsically desirable.

\GlossaryNote{Note 1}{
	In the context of a technical \GlossaryHyperRef{framework}{framework}, values capture priorities assumed to be had be the users of the framework.
	Values could concern the importance of being able to meet realtime deadlines, technical complexity, material costs, and so on.
}

\GlossaryEntry{viewpoint-architecture}{Viewpoint, Architecture}
An \GlossaryHyperRef{description}{description} of a problem domain, specifying (1) concerns, (2) conventions and (3) \GlossaryHyperRef{kind-model}{model kinds}.

	\GlossaryNote{ISO42010}{
		defines architecture viewpoint as ``work product establishing the conventions for the construction, interpretation and use of architecture views to frame specific system concerns''.
		Our definition is meant to express this definition and must be considered as being either equivalent or incorrect.
	}

\GlossaryEntry{virtual}{Virtual}
The property of having ones existence maintained by a computer.

}