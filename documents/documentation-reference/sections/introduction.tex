% Copyright (c) 2021 Eclipse Arrowhead Project
%
% This program and the accompanying materials are made available under the
% terms of the Eclipse Public License 2.0 which is available at
% http://www.eclipse.org/legal/epl-2.0.
%
% SPDX-License-Identifier: EPL-2.0

X \cite{palm2021reference}

\subsection{Primary Audiences}
\label{sec:introduction:audiences}

This document is being written and maintained for everyone that needs to read and/or write arrowhead documentation.
We understand this to include the following groups:

\begin{itemize}
\item \textit{System architects, integrators and developers} designing, integrating or developing Arrowhead systems.
\item \textit{Standardization engineers and researchers} seeking to extend, analyze or improve upon Arrowhead.
\item \textit{Decision makers, users and other stakeholders} that need to read and grasp Arrowhead documentation.
\end{itemize}

\subsection{Scope}
\label{sec:introduction:scope}

This document make up a so-called \textit{reference architecture}, which we understand to be a set of TODO.

\subsection{Notational Conventions}
\label{sec:introduction:conventions}

The following conventions regarding diagrams, references and requirements are adhered to throughout this document.
All three of them were selected by virtue of being deemed unsurprising to our primary audiences.

\subsubsection{Diagrams}

A box with a name inside it denotes a named entity.
A named arrow between boxes denotes the relationship implied by the name.
If a named arrow has an associated positive integer or range, the relation is to be considered as extending to the number of distinct entities indicated by that integer or range.
A range is denoted by $x..y$, where $x$ and $y$ are positive integers and $x<y$.
Omitting $y$ when using the range notation (e.g. ``$1..$'') means that the range is infinite from $x$.
Names, integers and ranges are \textit{properties}.
If two or more arrows are combined such that their source or target end is shared, a difference is made if a property is closest to a shared or non-shared arrow part.
If it is closest to a shared part, the property must be understood to apply to all combined arrows.
If it is closes to a non-shared part, the property be understood to only apply to that arrow.
A box with a dotted border represents a group.
The entities explicitly placed within the box may or may not represent all entities that belong to that group.
See Figure TODO for an example of this notation being used.

\subsubsection{References}

Square brackets around numbers (e.g. \cite{palm2021reference}) are references to the reference list in Section \ref{sec:references}.
The number within the brackets of any given reference corresponds to the entry with the same number in the reference list.

\subsubsection{Requirements}

Use of the words \textbf{must}, \textbf{must not}, \textbf{required}, \textbf{should}, \textbf{should not}, \textbf{recommended}, \textbf{may}, and \textbf{optional} are to be interpreted as follows when used in this document: \textbf{must} and \textbf{required} denote absolute requirements that must be adhered to for a described entity to be considered as compliant to this reference model; \textbf{must not} denotes an absolute prohibition; \textbf{should}, \textbf{should not} and \textbf{recommended} denote recommendations that should be deviated from only if special circumstances make it relevant; and, finally, \textbf{may} and \textbf{optional} denote something being truly optional.
These word definitions are derived from and are meant to capture what is outlined in RFC 2119 \cite{bradner1997keywords}.

\subsection{Relationships to Other Documents}
\label{sec:introduction:relationships}

TODO

\subsection{Section Overview}
\label{sec:introduction:sections}

The remaining sections of this document are organized as follows:
\vspace*{2mm}
\begin{itemize}[leftmargin=2cm,rightmargin=0pt,labelwidth=2cm,labelsep=0pt,itemindent=0pt,parsep=0.1cm,topsep=0.1cm,align=left]

\item[Section \ref{sec:introduction}]
This section.

\end{itemize}
