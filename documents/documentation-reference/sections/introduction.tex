% Copyright (c) 2021 Eclipse Arrowhead Project
%
% This program and the accompanying materials are made available under the
% terms of the Eclipse Public License 2.0 which is available at
% http://www.eclipse.org/legal/epl-2.0.
%
% SPDX-License-Identifier: EPL-2.0

In this document, we, the Eclipse Arrowhead project, outline the Arrowhead framework document structure, describe how to format print, interactive and semantic documentation, as well as presenting guidelines regarding document conformance and ratification.
We believe the documentation structure we relate here will facilitate effective and efficient distribution of knowledge within the Arrowhead community, serving to speed up decision-making, integration, decommissioning, and other relevant activities.

\subsection{Primary Audiences}
\label{sec:introduction:audiences}

This document is being written and maintained for everyone that needs to read and/or write arrowhead documentation.
We understand this to include the following groups:

\begin{itemize}
\item \textit{System architects, integrators and developers} designing, integrating or developing Arrowhead systems.
\item \textit{Standardization engineers and researchers} seeking to extend, analyze or improve upon Arrowhead.
\item \textit{Decision makers, users and other stakeholders} that need to read and grasp Arrowhead documentation.
\end{itemize}

\subsection{Scope}
\label{sec:introduction:scope}

This document describes a so-called \textit{reference architecture}, which provides abstract building blocks for more concrete documents or models to build upon.
The abstract building blocks are meant to capture practices we have come to understand as effectively addressing real-world problems, but without mentioning any concrete tools or other technologies.
In contrast to a \textit{reference model}, a reference architecture \textit{does} concern itself with how systems must or should be designed.
In this instance, that concern, or scope, is limited to the documentation of the Arrowhead framework.

\subsection{Notational Conventions}
\label{sec:introduction:conventions}

This document inherits its notational conventions form \textit{Eclipse Arrowhead Reference Model} \cite{palm2021reference}, which outlines how diagrams, references and requirements are presented in this document.

\subsection{Relationships to Other Documents}
\label{sec:introduction:relationships}

This document is an extension of the \textit{Eclipse Arrowhead Reference Model} \cite{palm2021reference}, which outlines all Arrowhead concepts not defined here.
You are, as a reader, assumed to be familiar with the concepts introduced by it, such as \textit{system}, \textit{service}, \textit{interface} and \textit{protocol}.

\subsection{Section Overview}
\label{sec:introduction:sections}

The remaining sections of this document are organized as follows:
\vspace*{2mm}
\begin{itemize}[leftmargin=2cm,rightmargin=0pt,labelwidth=2cm,labelsep=0pt,itemindent=0pt,parsep=0.1cm,topsep=0.1cm,align=left]

\item[Section \ref{sec:introduction}]
This section.

\end{itemize}
