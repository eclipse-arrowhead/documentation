% Copyright (c) 2021 Eclipse Arrowhead Project
%
% This program and the accompanying materials are made available under the
% terms of the Eclipse Public License 2.0 which is available at
% http://www.eclipse.org/legal/epl-2.0.
%
% SPDX-License-Identifier: EPL-2.0

We expect the automation system of today to keep becoming more and more computerized, digitized and interconnected.
By this we mean that more aspects of and surrounding automation machines will be handled by computers, more information will be made available to those computers and, finally, comparatively more such computers will be given the opportunity to collect, communicate and act on that information.
We believe that manufacturing, transportation, energy distribution, medicine, recycling, and all other industrial sectors concerned with automation will be affected by this development to some degree.
It will lead to increased automation efficiency and flexibility, as machines become able to perform more of the work traditionally assigned to humans.
However, it will also lead to new magnitudes of complexity, not the least because of the renewed incentive to use more and more of these highly communicative machines.

The \GlossaryHyperRef{framework-arrowhead}{\textit{Arrowhead framework}} is designed to address this explosion of complexity.
It provides a foundation for \GlossaryHyperRef{communication-service-oriented}{\textit{service-oriented communication}} \cite{mackenzie2006reference} between automation systems, such that interoperability, security, safety, performance, and other major concerns can be addressed efficiently and effectively.
It notably allows for \GlossaryHyperRef{system}{system} \GlossaryHyperRef{capability}{capabilities} to be \GlossaryHyperRef{description}{described}, shared and exploited dynamically by communicating \GlossaryHyperRef{device}{devices}.

In this document, we, the Eclipse Arrowhead project, present an authoritative set of concept definitions, meant to serve as the fundamental language for discussions about and the modeling of Arrowhead-based \GlossaryHyperRef{design}{designs}.
These definitions exist to help mitigate compatibility and consistency issues in \GlossaryHyperRef{software}{software}, tooling, \GlossaryHyperRef{model}{models}, documentation and all other things of relevance to the Arrowhead framework.

\subsection{Primary Audiences}
\label{sec:introduction:audiences}

This document is being written and maintained for all who need precise and rigorous definitions of important Arrowhead concepts, which we understand to likely include the following groups:

\begin{itemize}
\item \GlossaryHyperRef{acquirer}{Acquirers}, \GlossaryHyperRef{owner}{owners} and \GlossaryHyperRef{supplier}{suppliers} of Arrowhead systems.
\item \GlossaryHyperRef{builder}{Builders}, \GlossaryHyperRef{developer}{developers}, \GlossaryHyperRef{maintainer}{maintainers} and \GlossaryHyperRef{operator}{operators} of Arrowhead systems.
\item \GlossaryHyperRef{researcher}{Researchers} concerned with analyzing or refining the Arrowhead framework or Arrowhead systems.
\item Advanced \GlossaryHyperRef{user}{users} of Arrowhead systems.
\end{itemize}

\subsection{Scope}
\label{sec:introduction:scope}

This document is intended to clearly define all technical concepts of fundamental importance to the Arrowhead framework.
It does not specify how its definitions are to be used to design systems.
This makes its purpose analogous to that of a dictionary.
Dictionaries define words.
They may give examples of how certain words may be used, but they do not require that those words be used for any particular purposes.

The concepts presented here are meant to be useful as a resource for advanced Arrowhead framework learners, as well as to serve as foundation for other documentation and modeling efforts.
For those interested in using this document for architectural purposes, a description of how it can be used as a \GlossaryHyperRef{model-meta}{meta model} in the context of an ISO/IEC/IEEE 42010 \GlossaryHyperRef{viewpoint-architectural}{architectural viewpoint} is described in Section \ref{sec:conformance:iso42010}.

\newpage

\subsection{Notational Conventions}
\label{sec:introduction:conventions}

The following conventions regarding diagrams depicting graphs, references and requirements are adhered to throughout this document.
All three of them were selected by virtue of being deemed unsurprising to our primary audiences.

\subsubsection{Graph Diagrams}

A box with a solid border and a name inside it denotes a named \GlossaryHyperRef{entity}{entity}.
A named arrow between boxes denotes the \GlossaryHyperRef{relationship}{relationship} implied by the name.
Relationship names should be defined, or have their definitions referred to, in relation to the figures they are used in.
Exceptions are acceptable if the implications of a given name can be considered obvious.
The following five relationship names must be understood to always be defined:

\begin{enumerate}
\item \textit{refers to}, which means that the origin entity knows of or names the target entity (this relationship is implied if an arrow lacks an associated name);
\item \textit{conforms to}, which means that the origin entity \textit{refers to} the target entity and satisfies all \GlossaryHyperRef{constraint}{constraints} implicitly and/or explicitly associated with that target entity;
\item \textit{extends}, which means that the origin entity \textit{conforms to} the target entity and inherits all of its relationships;
\item \textit{is}, which means that the origin entity \textit{extends} the target entity and is member of a group named after that target entity; and, finally,
\item \textit{has}, which means that the origin entity \textit{refers to} and owns the target entity, where ownership entails access to and control over the owned entity.
\end{enumerate}

If a named arrow has an associated positive integer or range, which we refer to as a \textit{quantifier}, the relation is to be considered as extending to the number of distinct entities indicated by that integer or range.
A range is denoted by $x..y$, where $x$ and $y$ are integers and $0 \leq x < y$.
If $y$ is substituted by $*$, the range must be understood to extend infinitely from $x$ (e.g. ``$1..*$'').

If two or more arrows are combined such that their source or target end is shared, a difference is made if a quantifier is closest to a shared or non-shared arrow part.
If it is closest to a shared part, the quantity must be understood to apply to all arrows part of the combination.
For example, if an arrow extends to or from an entity to three other entities and the quantifier is ``$0..1$'', the relationship extends to only zero or one of the three entities.
If a quantifier is closest to a non-shared part, the quantifier must be understood to only apply to that arrow.
Combined arrows may have quantifiers both at their shared and non-shared parts.

A box with a dotted border represents a group.
The entities explicitly placed within the box may or may not represent all entities that belong to that group.
If a relationship extends to or from a group, rather to any entity inside that group, the relationship must be understood to extend to or from all entities inside that group.

Note that this document does \textit{not} define an Arrowhead profile for SysML \cite{omg2019sysml}, or any other modeling language.
As we cover later in Section \ref{sec:conformance}, however, we do expect all models based on this document not to contradict any of its definitions.

\subsubsection{References}

Square brackets around numbers (e.g. \cite{delsing2017iot}) are references to the reference list in Section \ref{sec:references}.
The number within the brackets of any given reference corresponds to the entry with the same number in the reference list.

References within this document are hyperlinked, which means that those reading it electronically can click the references and immediately be taken to their targets.
Special treatment is given to references targeting Section \ref{sec:glossary}, the \nameref{sec:glossary}.
These are displayed as regular text rendered with blue color.

\subsubsection{Requirements}

Use of the words \textbf{must}, \textbf{must not}, \textbf{required}, \textbf{should}, \textbf{should not}, \textbf{recommended}, \textbf{may}, and \textbf{optional} are to be interpreted as follows when used in this document: \textbf{must} and \textbf{required} denote absolute requirements that must be adhered to for a described entity to be considered as compliant to this reference model; \textbf{must not} denotes an absolute prohibition; \textbf{should}, \textbf{should not} and \textbf{recommended} denote recommendations that should be deviated from only if special circumstances make it relevant; and, finally, \textbf{may} and \textbf{optional} denote something being truly optional.
These word definitions are derived from and are meant to capture what is outlined in RFC 2119 \cite{bradner1997keywords}.

\subsection{Relationships to Other Documents}
\label{sec:introduction:relationships}

When this \GlossaryHyperRef{model-reference}{reference model} was produced, care was taken to reuse or build upon the concepts presented in the following works, in order of precedence:

\begin{enumerate}

\item \textbf{IoT Automation: Arrowhead Framework} (IoTA:AF) \cite{delsing2017iot}, which significantly includes an overview of the \textit{local automation cloud} concept in its second chapter, as well as the \textit{Arrowhead framework architecture} in its third chapter.
The book most significantly represents the state of the Arrowhead framework up until it was written.
Even though the framework has evolved since then, it still represents the most comprehensive view of the framework.
While the strictly architectural aspects of IoTA:AF are outside the scope of this document, the two mentioned chapters contain several definition with a high degree of relevance.

\item \textbf{ISO/IEC/IEEE 42010 Systems and software engineering — Architecture description} (ISO42010) \cite{iso42010}, which outlines a standardized approach to structuring architectural documents and models.
The standard is adhered to in the sense that the definitions of this document are meant to be useful as a meta model part of a so-called architectural viewpoint, as defined by the standard.

\item \textbf{Reference Model for Service Oriented Architecture} (SOA-RM) \cite{mackenzie2006reference}, which provides a standardized definition of Service-Oriented Architecture (SOA).
As communication between \GlossaryHyperRef{system}{systems} of the Arrowhead framework is understood to follow this paradigm, it becomes particularly relevant to consider.

\item \textbf{Reference Architecture Model Industrie 4.0} (RAMI4.0) \cite{adolphs2016reference}, which outlines an ontological and architectural view of \GlossaryHyperRef{industry40}{\textit{Industry 4.0}}.
The document may be seen as a predecessor to, or major influence on, the conceptual aspects of the Arrowhead framework.
In particular, the document describes how to model and design communicating industrial systems such that key Industry 4.0 characteristics can be facilitated, such as high degrees of dynamicity and interoperability.
However, as RAMI4.0 is a reference \textit{architecture} rather than a reference \textit{model}, we have only been concerned with what concepts it defines and what problems it frames.
This delimitation excludes its ``architectural layers'', ``life-cycle \& value-stream'' phases and ``hierarchical levels'', as well as the abstract design of its ``asset administrative shell''.
These excluded aspects are neither condemned nor endorsed by this document.
They are simply outside its scope.

\end{enumerate}

Only conformity with IoTA:AF and ISO42010 is observed strictly, which means that concept definitions presented here may diverge from those of the other two works.
All significant terminology differences are noted in the glossary of Section \ref{sec:glossary}, which briefly defines all concepts of relevance to this document.

\subsection{Section Overview}
\label{sec:introduction:sections}

The remaining sections of this document are organized as follows:
\vspace*{2mm}
\begin{itemize}[leftmargin=2cm,rightmargin=0pt,labelwidth=2cm,labelsep=0pt,itemindent=0pt,parsep=0.1cm,topsep=0.1cm,align=left]

\item[Section \ref{sec:introduction}]
This section.

\item[Section \ref{sec:arrowhead}]
An informal overview of Arrowhead, serving both to provide a workable summary of the framework and to prepare readers for better understanding Section \ref{sec:model}.

\item[Section \ref{sec:model}]
The formal and normative description of Arrowhead.
Each of its subsections is concerned with one major Arrowhead concept, ranging from entities to systems-of-local-clouds.

\item[Section \ref{sec:conformance}]
A brief list of requirements, meant to help determine whether or not a given model or document is conforming to this reference model.

\item[Section \ref{sec:glossary}]
Lists all significant terms and abbreviations presented in this document in alphabetical order.

\item[Section \ref{sec:references}]
Lists references to publications referred to in this document.

\item[Section \ref{sec:revision}]
Records the history of changes made to this document.

\end{itemize}
