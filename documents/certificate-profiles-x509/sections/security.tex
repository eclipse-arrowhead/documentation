% Copyright (c) 2021 Eclipse Arrowhead Project
%
% This program and the accompanying materials are made available under the
% terms of the Eclipse Public License 2.0 which is available at
% http://www.eclipse.org/legal/epl-2.0.
%
% SPDX-License-Identifier: EPL-2.0

WChoosing a signature suite for a certificate poorly can lead to severe security vulnerabilities.
We \textit{recommend} that credible information security institutes, such as NIST, ENSIA or IETF, be consulted for making choices about algorithms, key lengths and other relevant security details.

Given that no relevant breakthroughs are made or expected in quantum computing, you \textit{may} chose to follow RFC 7525, which recommends the following four TLS cipher suites:

\vspace*{0.5cm}
\noindent\begin{tabularx}{\textwidth}{| p{2.5cm} | p{4.5cm} | p{2.5cm} | X |} \hline
\rowcolor{gray!33} Key Exchange & Authentication             & Encryption  & Hash \\ \hline

DHE                             & RSA (2048-bit or 3072-bit) & AES 128 GCM & SHA256 \\ \hline
ECDHE                           & RSA (2048-bit or 3072-bit) & AES 128 GCM & SHA256 \\ \hline
DHE                             & RSA (2048-bit or 3072-bit) & AES 256 GCM & SHA384 \\ \hline
ECDHE                           & RSA (2048-bit or 3072-bit) & AES 256 GCM & SHA384 \\ \hline

\end{tabularx}
\vspace*{0.5cm}

Only the first of the above cipher suites is required to be supported by all TLS 1.3 implementations (RFC 8446, Section 9.1).
Each cipher suite includes a signature suite (the \textit{Authentication} and \textit{Hash} fields).
Adherence to RFC 7525 means that RSA (2048-bit or 3072-bit) with SHA256 or SHA384 is used to sign certificates.
Given that RFC 7525 is trusted, SHA256 and SHA384 \textit{may} be suitable choices for producing certificate identifiers, as discussed in Section 4.

The above recommendations are \textit{general}, in the sense that no particular assumptions are made about the setting in which the device employing the signature or cipher suite is located.
We understand that many Arrowhead installations will involve hardware with limited computational capabilities, which may not be able to handle primitives of the cryptographic strengths we have mentioned.
The Eclipse Arrowhead project will publish summaries of recommendations for such and other settings in the future.