% Copyright (c) 2021 Eclipse Arrowhead Project
%
% This program and the accompanying materials are made available under the
% terms of the Eclipse Public License 2.0 which is available at
% http://www.eclipse.org/legal/epl-2.0.
%
% SPDX-License-Identifier: EPL-2.0

Certificates must be created, distributed, replaced as they expire and, sometimes, revoked before they expire.
If these tasks are handled without care, it can lead to serious security vulnerabilities.
To help making this handling as rigorous as possible, the Eclipse Arrowhead project provides the \textit{Certificate Authority} system, which, through some other helper systems, provides an infrastructure for managing the certificate life-cycle within local clouds.
We \textit{recommend} that the system be used, or a similarly capable replacement, for all Eclipse Arrowhead installations.

Generally, when certificate life-cycles are managed, we \textit{recommend} that the following be observed:

1. Create each private key on the device that will use it.
2. Use CSRs to avoid moving private keys between devices during certificate issuance.
3. Never make backups or other copies of private keys that can be trivially replaced.
4. Store backups of sensitive private keys offline, if possible.
5. Store active private keys in secure hardware elements, such as TPMs (ISO/IEC 11889).
6. Immediately revoke owned certificates whose private keys are suspected to be compromised.
7. Actively look for and act on revocations in the certificate chains of counter-parties.

The above list is \textit{not} to be considered as being exhaustive.
Adhering to it is not a substitute for consulting independent and credible security experts.
The list is likely to be revised as more experience is gained related to the security of Arrowhead installations.