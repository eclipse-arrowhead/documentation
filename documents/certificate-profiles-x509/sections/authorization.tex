% Copyright (c) 2021 Eclipse Arrowhead Project
%
% This program and the accompanying materials are made available under the
% terms of the Eclipse Public License 2.0 which is available at
% http://www.eclipse.org/legal/epl-2.0.
%
% SPDX-License-Identifier: EPL-2.0

While no steps in addition to those in Section 6 of RFC 5280 are mandatory for certificate \textit{validation}, there are additional details made available by our profiles that \textit{should} be used for \textit{authorization} when applicable.
These details are (A) profile identifiers and (B) issuer identities.
For example, if an Arrowhead system provides a service, it \textit{may} be relevant to ensure that any consumer of that service belongs to the same local cloud and is an operator.

We \textit{recommend} that a procedure practically equivalent to the below algorithm, described in pseudo-code, is used when extracting certificate profile identifiers.
The input to the algorithm is an array containing a certificate chain `chain` of a given peer.
The certificate at index 0 of that array is owned by the peer in question, while the remaining certificates represent its issuance hierarchy, in order of issuance from least significant to most.

\begin{verbatim}
function getCertificateProfileIdentifier(chain) {
    if (chain.length == 0) { throw "empty chain"; }

    let dnq = getDNQualifier(chain[0]);
    switch (dnq) {
        case "ma":
            break;
        case "or": case "ga":
            requireProfileSequence(chain[1..], ["ma"]); break;
        case "lo":
            requireProfileSequence(chain[1..], ["or", "ma"]); break;
        case "on": case "de": case "sy": case "op":
            requireProfileSequence(chain[1..], ["lo", "or", "ma"]); break;
        default:
            throw "unexpected DNQualifier";
    }
    return dnq;
}

function requireProfileSequence(chain, identifiers) {
    if (identifiers.length == 0) { return; }
    if (chain.length == 0) { throw "expected additional certificate(s)"; }

    if (getDNQualifier(chain[0]) != identifiers[0]) {
        throw "unexpected certificate profile";
    }

    requireProfileSequence(chain[1..], identifiers[1..]);
}
\end{verbatim}

The `getDNQualifier` function is assumed to extract the `subject` `DN` field, described in Section 2.1.6, of a given certificate.
If several such fields is present in a given certificate, the rightmost (i.e. least significant) must be returned by the function.
The `[1..]` notation is used to describe an array being "sliced" such that a new array is produced containing all but the first element of the original array.
The algorithm asserts that any extracted identifier appears at the expected hierarchical level, apart from extracting the identifier itself.

Testing elements of an issuance hierarchy could be performed by comparing certificate DER representations byte by byte, or by comparing their DER hashes.
The following pseudo-code describes how a service provider could assert that a given peer is a system belonging to a certain local cloud:

\begin{verbatim}
if (getCertificateProfileIdentifier(peer.chain) != "sy" || hash(peer.chain[1]) != hashOfRelevantLocalCloudCertificate) {
    throw "unauthorized";
}
\end{verbatim}

Note that if either of the secure transports TLS or DTLS is used by a given Arrowhead peer, the `post\_handshake\_auth` extension (RFC 8446, Section 4.6.2), must be required in order for both peers of a connection to provide and validate certificates.
The extension \textit{should} always be required when Arrowhead systems communicate.
Use of the `post\_handshake\_auth` extension is referred to as \textit{client authentication} by many software libraries and applications, as the initiating peer, or \textit{client}, is also authenticated by the responding peer, or \textit{server}.