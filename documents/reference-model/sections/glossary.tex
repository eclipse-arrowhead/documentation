% Copyright (c) 2021-10-07 Eclipse Arrowhead Project
%
% This program and the accompanying materials are made available under the
% terms of the Eclipse Public License 2.0 which is available at
% http://www.eclipse.org/legal/epl-2.0.
%
% SPDX-License-Identifier: EPL-2.0

{

\newcommand{\GlossaryEntry}[2]{\paragraph{#2}\label{sec:glossary:#1}\,}
\newcommand{\GlossaryNote}[2]{\begin{minipage}[b]{\dimexpr\linewidth-0.5cm\relax}\vspace*{0.33cm}\footnotesize{\textbf{#1}\ #2}\end{minipage}}

\GlossaryEntry{abstract}{Abstract}
See \GlossaryNameRef{model-abstract}

\GlossaryEntry{administration}{Administration}

\GlossaryEntry{administrator}{Administrator}

\GlossaryEntry{architecture}{Architecture}
A \GlossaryHyperRef{model-concrete}{concrete model} of a \GlossaryHyperRef{system-of-systems}{system-of-systems} defined in terms of certain \GlossaryHyperRef{model-reference}{reference models}, \GlossaryHyperRef{architecture-reference}{reference architectures} and other concrete architectures.
See also Section \ref{sec:introduction:scope}.

	\GlossaryNote{RAMI4.0}{
	    defines architecture as the ``combination of elements of a model based on principles and rules for constructing, refining and using it''.
		We consider ``combinations of elements of a model'' to be a ``model of a system-of-systems'' and to be ``based on principles and rules for constructing, refining and using it'' as building upon reference models and architectures.
		Our definition should be interpreted as being compatible but more specific.
	}

	\GlossaryNote{SOA-RM}{
		defines software architecture as ``the structure or structures of an information system consisting of entities and their externally visible properties, and the relationships among them''.
		That definition is equivalent to our definition of \GlossaryHyperRef{model}{model}, with the exception that the thing being modelled has to be an information system.
		As our definition is concerned with a model and a system-of-systems, which must be an information system, we regard out definition as compatible but more specific.
	}

\GlossaryEntry{architecture-reference}{Architecture, Reference}
A significantly useful \GlossaryHyperRef{model-abstract}{abstract model} of a \GlossaryHyperRef{system-of-systems}{system-of-systems} defined in terms of certain \GlossaryHyperRef{model-reference}{reference models} and other reference architectures.
See also Section \ref{sec:introduction:scope}.

	\GlossaryNote{RAMI4.0}{
		defines reference architecture as a ``model for an architecture description (for I[ndustry ]4.0) which is generally used and recognized as being suitable (has reference character)''.
		We consider a ``model for an architecture description'' to be an ``abstract model of a system-of-systems''.
		Our definition should be interpreted as being compatible but more specific.
	}

	\GlossaryNote{SOA-RM}{
	    defines reference architecture as ``an architectural design pattern that indicates how an abstract set of mechanisms and relationships realizes a predetermined set of requirements''.
	    While we let the part about requirements be implicit, our definition should be interpreted as being compatible but more specific.
	}

\GlossaryEntry{arrowhead}{Arrowhead}
See \GlossaryNameRef{framework-arrowhead}.

\GlossaryEntry{artifact}{Artifact}
A thing or object, tangible or intangible.

\GlossaryEntry{asset}{Asset}
An \GlossaryHyperRef{artifact}{artifact} that is of value to an \GlossaryHyperRef{organization}{organization}.

	\GlossaryNote{RAMI4.0}{
		defines asset as an ``object which has a value for an organization''.
		Our definition should be interpreted as being equivalent.
	}

\GlossaryEntry{authentication}{Authentication}

\GlossaryEntry{authorization}{Authorization}

\GlossaryEntry{capability}{Capability}
A task, of any nature, that can be performed by a \GlossaryHyperRef{provider-service}{service provider}.

	\GlossaryNote{SOA-RM}{
		defines a capability as ``a real-world effect that a service provider is able to provide to a service consumer''.
		Our definition should be interpreted as being equivalent.
	}

\GlossaryEntry{certificate}{Certificate}

\GlossaryEntry{cloud}{Cloud}

\GlossaryEntry{cloud-compute}{Cloud, Compute}

\GlossaryEntry{cloud-local}{Cloud, Local}

\GlossaryEntry{cloud-storage}{Cloud, Storage}

\GlossaryEntry{cloud-virtual}{Cloud, Virtual}

\GlossaryEntry{codec}{Codec}
A \GlossaryHyperRef{model-concrete}{concrete} \GlossaryHyperRef{type-data}{data type} used to structure \GlossaryHyperRef{data}{data} for transmission, storage and/or interpretation.

\GlossaryEntry{coding}{Coding}
Transforming \GlossaryHyperRef{data}{data} from being expressed in one \GlossaryHyperRef{codec}{codec} into another.
See also \GlossaryHyperRef{decoding}{decoding} and \GlossaryHyperRef{encoding}{encoding}.

\GlossaryEntry{coding-a}{Coding, A}
Synonymous to \GlossaryNameRef{codec}.

\GlossaryEntry{component}{Component}
A part of a \GlossaryHyperRef{system}{system}, contributing to it facilitating its \GlossaryHyperRef{capability}{capabilities}.

	\GlossaryNote{Note 1}{
		While it is also correct that a system can be regarded as a component of a \GlossaryHyperRef{system-of-systems}{system-of-systems}, we associate the word ``component'' primarily with plain systems.
		When referring to the systems of a system-of-systems, we recommend using the words ``system'' and ``subsystem'' to avoid confusion.
	}

	\GlossaryNote{Note 2}{
		A component is practically distinct from a system by being unable to \GlossaryHyperRef{provider-service}{provide} or \GlossaryHyperRef{consumer-service}{consume} \GlossaryHyperRef{service}{services} independently.
	}

\GlossaryEntry{concrete}{Concrete}
See \GlossaryNameRef{model-concrete}.

\GlossaryEntry{concretization}{Concretization}
Making an \GlossaryHyperRef{model-abstract}{abstract model} less abstract by specifying details required to realize it.

\GlossaryEntry{configuration}{Configuration}

\GlossaryEntry{constraint}{Constraint}
A \GlossaryHyperRef{property}{property} that imposes constraints, or limits, on an \GlossaryHyperRef{entity}{entity} or \GlossaryHyperRef{relationship}{relationship}.

	\GlossaryNote{Note 1}{
		The presence of constraints enable \GlossaryHyperRef{validation}{validation}.
	}

	\GlossaryNote{Note 2}{
		Perhaps a bit counterintuitively, a constraint \textit{adds} information to its target by reducing the ways in which it could be realized.
	}

\GlossaryEntry{consumer-service}{Consumer, Service}
A \GlossaryHyperRef{system}{system} currently \GlossaryHyperRef{invocation-function}{invoking} a \GlossaryHyperRef{function}{function} \GlossaryHyperRef{provider-service}{provided} via a \GlossaryHyperRef{service}{service}.
See also TODO.

	\GlossaryNote{SOA-RM}{
		defines a service consumer as ``an entity which seeks to satisfy a particular need through the use [of] capabilities offered by means of a service''.
		We require that the one consuming the service is (1) a \GlossaryHyperRef{system}{system} rather than just any \GlossaryHyperRef{entity}{entity}, (2) that the \GlossaryHyperRef{capability}{capabilities} of the consumed service be exercised by invoking a function, as well as (3) that the invocation satisfies a service consumption policy.
	}

\GlossaryEntry{data}{Data}
A sequence of datums forming a set of \GlossaryHyperRef{description}{descriptions} via the structure imposed by a \GlossaryHyperRef{type-data}{data type}.

	\GlossaryNote{Note 1}{
		Without complete knowledge of the types associated with some data, that data cannot be interpreted.
	}

\GlossaryEntry{decoding}{Decoding}
The process through which \GlossaryHyperRef{data}{data} is transformed from being expressed in a \GlossaryHyperRef{codec}{codec} suitable for transmission or storage to another codec suitable for interpretation.

	\GlossaryNote{Note 1}{
		The operation is the reverse of \GlossaryHyperRef{encoding}{encoding}.
	}

	\GlossaryNote{Note 2}{
		The term can also be used to express the act of a human interpreting data.
	}


\GlossaryEntry{description}{Description}
Facts about an \GlossaryHyperRef{entity}{entity} or \GlossaryHyperRef{entity-class-of}{class of entities}, expressed in the form of a \GlossaryHyperRef{model}{model}, a text, or both.

\GlossaryEntry{description-interface-design}{Description, Interface Design}

\GlossaryEntry{design}{Design}

\GlossaryEntry{design-interface}{Design, Interface}

\GlossaryEntry{device}{Device}

\GlossaryEntry{device-human-interface}{Device, Human Interface (HID)}

\GlossaryEntry{encoding}{Encoding}
The process through which \GlossaryHyperRef{data}{data} is transformed from being expressed in a \GlossaryHyperRef{codec}{codec} suitable for interpretation to another codec suitable for transmission or storage.

	\GlossaryNote{Note 1}{
		The operation is the reverse of \GlossaryHyperRef{decoding}{decoding}.
	}

	\GlossaryNote{Note 2}{
		The term can also be used to express the act of a human recording data.
	}

\GlossaryEntry{encoding-an}{Encoding, An}
Synonymous to \GlossaryNameRef{codec}.

\GlossaryEntry{entity}{Entity}
An object, tangible or intangible, that is uniquely \GlossaryHyperRef{identity}{identifiable}.

	\GlossaryNote{Note 1}{
		An entity being uniquely identifiable does not necessarily mean that it is associated with a \GlossaryHyperRef{certificate}{certificate} or \GlossaryHyperRef{identifier}{identifier}.
		It only means that a \GlossaryHyperRef{description}{description} can be rendered that unambigously refers to the entity in question.
	}

	\GlossaryNote{RAMI4.0}{
		defines entity as an ``uniquely identifiable object which is administered in the information world due to its importance''.
		Our definition should be interpreted as being equivalent.
	}

	\GlossaryNote{SOA-RM}{
	    mentions the word ``entity'' nine times, but provides no explicit definition.
	    We assume their definition to match that of a regular English dictionary, such as ``something that has separate and distinct existence and objective or conceptual reality'' \cite{webster2021entity}.
		Our definition should be interpreted as being equivalent.
	}

\GlossaryEntry{entity-class-of}{Entity, Class of}
A set of \GlossaryHyperRef{entity}{entities} that share a common \GlossaryHyperRef{property}{property}.

\GlossaryEntry{framework}{Framework}
A set of assumptions, concepts, values and practices that frame a certain problem domain.

	\GlossaryNote{SOA-RM}{
		defines framework as ``a set of assumptions, concepts, values, and practices that constitutes a way of viewing the current environment''.
		Our definition should be interpreted as being equivalent.
	}

\GlossaryEntry{framework-arrowhead}{Framework, Arrowhead}
Either of the \GlossaryHyperRef{framework}{framework of ideas} and the \GlossaryHyperRef{framework-software}{framework of software} maintained by the Arrowhead project.

\GlossaryEntry{framework-software}{Framework, Software}
A set of software specifications, \GlossaryHyperRef{implementation-software}{implementations} and other \GlossaryHyperRef{artifact}{artifacts} meant to help address the problem domain of a certain \GlossaryHyperRef{framework}{framework}.

\GlossaryEntry{function}{Function}
The realization of the \GlossaryHyperRef{protocol}{protocol} established by a certain \GlossaryHyperRef{signature-function}{function signature}.
See also TODO.

\GlossaryEntry{hid}{HID} See \GlossaryNameRef{device-human-interface}.

\GlossaryEntry{industry40}{Industry 4.0}

\GlossaryEntry{identification}{Identification}
The process through which an \GlossaryHyperRef{entity}{entity} collects and verifies the \GlossaryHyperRef{identity}{identity} of another entity. 

\GlossaryEntry{identifier}{Identifier}
\GlossaryHyperRef{data}{Data} associated with an \GlossaryHyperRef{entity}{entity} that allows for it to be \GlossaryHyperRef{identification}{identified}.

\GlossaryEntry{identity}{Identity}
The aspect or aspects, such as \GlossaryHyperRef{identifier}{identifiers}, that makes an \GlossaryHyperRef{entity}{entity} distinct from all other entities.

\GlossaryEntry{implementation}{Implementation}
See \GlossaryNameRef{implementation-software}.

\GlossaryEntry{implementation-software}{Implementation, Software}

\GlossaryEntry{interface}{Interface}
A means of moving \GlossaryHyperRef{message}{messages} between a \GlossaryHyperRef{medium}{medium} and a \GlossaryHyperRef{system}{system}.

\GlossaryEntry{interface-administrative}{Interface, Administrative}

\GlossaryEntry{interface-management}{Interface, Management}

\GlossaryEntry{interface-operator}{Interface, Operator}

\GlossaryEntry{invocation-function}{Invocation, Function}
The attempt to excercise \GlossaryHyperRef{capability}{capabilities} associated with a \GlossaryHyperRef{provider-service}{service provider} by sending a \GlossaryHyperRef{message}{message} to one of its \GlossaryHyperRef{function}{functions}.
See also TODO.

\GlossaryEntry{manager}{Manager}

\GlossaryEntry{management}{Management}

\GlossaryEntry{medium}{Medium}
A means of connecting \GlossaryHyperRef{interface}{interfaces} such that \GlossaryHyperRef{message}{messages} of a certain \GlossaryHyperRef{protocol}{protocol} can be transmitted between them.

\GlossaryEntry{message}{Message}

\GlossaryEntry{metadata}{Metadata}

\GlossaryEntry{model}{Model}
A representation of facts in the form of a graph, consisting of \GlossaryHyperRef{entity}{entities}, \GlossaryHyperRef{relationship}{relationships} and \GlossaryHyperRef{property}{properties}.

	\GlossaryNote{Note 1}{
		Models can be expressed or recorded in many ways, including as visual diagrams, spoken words, text and binary data.
	}

	\GlossaryNote{Note 2}{
		Models can be human-readable, machine-readable, or both.
	}

\GlossaryEntry{model-abstract}{Model, Abstract}
A \GlossaryHyperRef{model}{model} that is \textit{insufficiently} specified to be possible to realize as the \GlossaryHyperRef{artifact}{artifact} it represents.

	\GlossaryNote{Note 1}{
		Abstract models can be referred to by other models, serving as a form of \GlossaryHyperRef{constraint}{constraint}.
		They are commonly used to enforce a degree of uniformity across multiple other models.
	}

\GlossaryEntry{model-concrete}{Model, Concrete}
A \GlossaryHyperRef{model}{model} that is \textit{sufficiently} specified to be possible to realize as the \GlossaryHyperRef{artifact}{artifact} it represents.

	\GlossaryNote{Note 1}{
		Two examples of artifacts that could be produced from a concrete model are concrete \GlossaryHyperRef{protocol}{protocols}, the \GlossaryHyperRef{message}{messages} of which can be practically \GlossaryHyperRef{coding}{coded}, and \GlossaryHyperRef{implementation-software}{software implementations}.
	}

\GlossaryEntry{model-reference}{Model, Reference}
An \GlossaryHyperRef{model-abstract}{abstract model} defining technical concepts of fundamental importance to a specific problem domain.
See also Section \ref{sec:introduction:scope}.

	\GlossaryNote{RAMI4.0}{
		defines reference model as a ``model that is generally used and recognized as being suitable (has recommendation character) for deriving specific models''.
		We understand their use of the word ``specific'' to be equivalent to how we use ``concrete''.
		Even though our definition clarifies that the model in question must be abstract, it should be interpreted as being equivalent.
	}

	\GlossaryNote{SOA-RM}{
		defines reference model as ``an abstract framework for understanding significant relationships among the entities of some environment that enables the development of specific architectures using consistent standards or specifications supporting that environment''.
		It further clarifies that a ``reference model consists of a minimal set of unifying concepts, axioms and relationships within a particular problem domain, and is independent of specific standards, technologies, implementations, or other concrete details''.
		Our definition should be interpreted as being equivalent.
	}

\GlossaryEntry{operator}{Operator}

\GlossaryEntry{organization}{Organization}

\GlossaryEntry{policy}{Policy}
A set of obligations, \GlossaryHyperRef{constraint}{constraints} and/or other conditions under which some activity is permitted.

	\GlossaryNote{SOA-RM}{
		defines policy as ``a statement of obligations, constraints or other conditions of use of an owned entity as defined by a participant''.
		Our definition should be interpreted as being equivalent.
	}

\GlossaryEntry{policy-service-consumption}{Policy, Service Consumption}
The \GlossaryHyperRef{policy}{policy} under which a \GlossaryHyperRef{provider-service}{provided service} is permitted to be \GlossaryHyperRef{consumer-service}{consumed}.

\GlossaryEntry{profile}{Profile}
A \GlossaryHyperRef{model}{model} imposing constraints on a \GlossaryHyperRef{protocol}{protocol}.
A profile could specify a \GlossaryHyperRef{stack-protocol}{protocol stack}, certain message semantics, how authentication and authorization are to be carried out, among many other possible examples.

\GlossaryEntry{property}{Property}
A name/value pair of \GlossaryHyperRef{data}{data}, associated with either an \GlossaryHyperRef{entity}{entity} or a \GlossaryHyperRef{relationship}{relationship}.

\GlossaryEntry{protocol}{Protocol}
A \GlossaryHyperRef{model}{model} of communication defined in terms of \GlossaryHyperRef{message}{messages}.
See also \GlossaryNameRef{stack-protocol}.

\GlossaryEntry{provider-service}{Provider, Service}
A \GlossaryHyperRef{system}{system} that makes \GlossaryHyperRef{service}{services} available for \GlossaryHyperRef{consumer-service}{consumption} by any systems able to satisfy its \GlossaryHyperRef{policy-service-consumption}{consumption policies}.
See also TODO.

	\GlossaryNote{SOA-RM}{
		defines a service provider as ``an entity (person or organization) that offers the use of capabilities by means of a service''.
		Our definition is more specific in that it requires the \GlossaryHyperRef{entity}{entity} be a system.
	}

\GlossaryEntry{relationship}{Relationship}
A uni-directional association of two \GlossaryHyperRef{entity}{entities}, possibly with an associated \GlossaryHyperRef{data}{data} name.

\GlossaryEntry{service}{Service}
A set of \GlossaryHyperRef{function}{functions} that can be \GlossaryHyperRef{provider-service}{provided} via an \GlossaryHyperRef{interface}{interface}.
See also TODO.

	\GlossaryNote{RAMI4.0}{
		defines a service as ``separate scope of functions offered by an entity or organization via interfaces''.
		Our definition restricts service provision to systems.
	}

	\GlossaryNote{SOA-RM}{
		defines a service as ``the means by which the needs of a consumer are brought together with the capabilities of a provider''.
		Our definition is more specific about how the \GlossaryHyperRef{capability}{capabilities} of a service are made available.
	}

\GlossaryEntry{session}{Session}

\GlossaryEntry{shell-administrative}{Shell, Administrative}

\GlossaryEntry{signature-function}{Signature, Function}
A \GlossaryHyperRef{model}{model} specifying the \GlossaryHyperRef{type-data}{type} of the \GlossaryHyperRef{message}{message} a given \GlossaryHyperRef{function}{function} accepts when \GlossaryHyperRef{invocation-function}{invoked}, as well as the types of any messages it could return in response.
See also TODO.

	\GlossaryNote{Note 1}{
		This means that a function signature establishes a \GlossaryHyperRef{protocol}{protocol} for a certain \GlossaryHyperRef{service}{service} function.
	}

\GlossaryEntry{software}{Software}
See \GlossaryNameRef{implementation-software}.

\GlossaryEntry{solc}{SoLC} See \GlossaryNameRef{system-of-local-clouds}.

\GlossaryEntry{sos}{SoS} See \GlossaryNameRef{system-of-systems}.

\GlossaryEntry{stack-protocol}{Stack, Protocol}

\GlossaryEntry{standard}{Standard}

\GlossaryEntry{stakeholder}{Stakeholder}

\GlossaryEntry{subsystem}{Subsystem} See \GlossaryNameRef{component}.

\GlossaryEntry{system}{System}
An \GlossaryHyperRef{entity}{entity} capable of \GlossaryHyperRef{provider-service}{providing services}, \GlossaryHyperRef{consumer-service}{consuming services}, or both.

	\GlossaryNote{Note 1}{
		The word ``system'' is more generally understood to be very inclusive, expressing the larger idea of connected \GlossaryHyperRef{component}{components} facilitating one or more \GlossaryHyperRef{capability}{capabilities}.
		From the perspective of Arrowhead, however, capabilities can only be \GlossaryHyperRef{invocation-function}{invoked} through \GlossaryHyperRef{service}{services}, which means that a system unable to provide or consume services can only be described as a component of another system.
	}

	\GlossaryNote{Note 2}{
		A system is practically distinct from a \GlossaryHyperRef{system-of-systems}{system-of-systems} by being represented only by a single \GlossaryHyperRef{identity}{identity}.
	}

\GlossaryEntry{system-of-local-clouds}{System-of-Local-Clouds (SoLC)}
A set of \GlossaryHyperRef{cloud-local}{local clouds} that \GlossaryHyperRef{consumer-service}{consume} each other's \GlossaryHyperRef{service}{services} in order to facilitate a \GlossaryHyperRef{capability}{capability} none of the constituent local clouds could \GlossaryHyperRef{provider-service}{provide} on its own.

\GlossaryEntry{system-of-systems}{System-of-Systems (SoS)}
A set of \GlossaryHyperRef{system}{systems} that \GlossaryHyperRef{consumer-service}{consume} each other's \GlossaryHyperRef{service}{services} in order to facilitate a \GlossaryHyperRef{capability}{capability} none of the constituent systems could \GlossaryHyperRef{provider-service}{provide} on its own.

	\GlossaryNote{Note 1}{
		The system-of-systems concept is not restricted to the boundaries imposed by local clouds, organizations and systems-of-local-clouds, even though a local cloud could, for example, be regarded as a type system-of-systems.
	}

\GlossaryEntry{token}{Token} See \GlossaryNameRef{token-authentication}.

\GlossaryEntry{token-authentication}{Token, Authentication}

\GlossaryEntry{type-data}{Type, Data}

\GlossaryEntry{type-enumerating}{Type, Enumerating}

\GlossaryEntry{type-primitive}{Type, Primitive}

\GlossaryEntry{type-structured}{Type, Structured}

\GlossaryEntry{type-union}{Type, Union}

\GlossaryEntry{user}{User}

\GlossaryEntry{validation}{Validation}
The process through which it is determined if a \GlossaryHyperRef{model}{model} satisfies all of its \GlossaryHyperRef{constraint}{constraints}.

}