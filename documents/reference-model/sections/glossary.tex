% Copyright (c) 2021-10-07 Eclipse Arrowhead Project
%
% This program and the accompanying materials are made available under the
% terms of the Eclipse Public License 2.0 which is available at
% http://www.eclipse.org/legal/epl-2.0.
%
% SPDX-License-Identifier: EPL-2.0

{

\newcommand{\GlossaryEntry}[2]{\paragraph{#1}\label{sec:glossary:#2}\,}
\newcommand{\GlossaryNote}[2]{\begin{minipage}[b]{\dimexpr\linewidth-0.5cm\relax}\vspace*{0.33cm}\footnotesize{\textbf{#1}\ #2}\end{minipage}}

\GlossaryEntry{Abstract}{abstract}
See \GlossaryNameRef{model-abstract}

\GlossaryEntry{Administration}{administration}

\GlossaryEntry{Administrator}{administrator}

\GlossaryEntry{ANI}{ani}
See \GlossaryNameRef{interface-application-network}.

\GlossaryEntry{API}{api}
See \GlossaryNameRef{interface-application-programming}.

\GlossaryEntry{Application}{application}

\GlossaryEntry{Architecture}{architecture}
A \GlossaryHyperRef{concrete model}{model-concrete} of a \GlossaryHyperRef{system-of-systems}{system-of-systems} defined in terms of certain \GlossaryHyperRef{reference models}{model-reference}, \GlossaryHyperRef{reference architectures}{architecture-reference}, \GlossaryHyperRef{protocols}{protocol}, \GlossaryHyperRef{profiles}{profile} and \GlossaryHyperRef{specifications}{specification}.
See also Section \ref{sec:introduction:scope}.

	\GlossaryNote{Note 1}{
		An \GlossaryHyperRef{architecture}{architecture} can be extended by another architecture or realized as a concrete system-of-systems.
	}

	\GlossaryNote{RAMI4.0}{
	    defines architecture as the ``combination of elements of a model based on principles and rules for constructing, refining and using it''.
		We consider ``combinations of elements of a model'' to be a ``model of a system-of-systems'' and to be ``based on principles and rules for constructing, refining and using it'' as building upon the artifacts we list above.
		Our definition should be interpreted as being compatible but more specific.
	}

	\GlossaryNote{SOA-RM}{
		defines software architecture as ``the structure or structures of an information system consisting of entities and their externally visible properties, and the relationships among them''.
		That definition is equivalent to our definition of \GlossaryHyperRef{model}{model}, with the exception that the thing being modelled has to be an information system.
		As our definition is concerned with a model and a system-of-systems, which must be an information system, we regard out definition as compatible but more specific.
	}

\GlossaryEntry{Architecture, Reference}{architecture-reference}
An \GlossaryHyperRef{abstract model}{model-abstract} of a \GlossaryHyperRef{system-of-systems}{system-of-systems} defined in terms of certain \GlossaryHyperRef{reference models}{model-reference}, abstract \GlossaryHyperRef{protocols}{protocol}, abstract \GlossaryHyperRef{profiles}{profile} and abstract \GlossaryHyperRef{specifications}{specification}.

	\GlossaryNote{Note 1}{
		A reference architecture can be extended by another reference architecture or be \GlossaryHyperRef{concretized}{concretization-model} by a regular \GlossaryHyperRef{architecture}{architecture}.
	}

	\GlossaryNote{RAMI4.0}{
		defines reference architecture as a ``model for an architecture description (for I[ndustry ]4.0) which is generally used and recognized as being suitable (has reference character)''.
		We consider ``model for an architecture description'' to be a ``model of a system-of-systems'' and assume that there is no need to explicitly mention that a given reference architecture must be suitable for meeting real-world business objectives or be useful as foundation for other architectures.
		ur definition should be interpreted as being compatible but more specific.
	}

	\GlossaryNote{SOA-RM}{
	    defines reference architecture as ``an architectural design pattern that indicates how an abstract set of mechanisms and relationships realizes a predetermined set of requirements''.
	    It complements its definition with explanations in its first section, which also make it clear that a reference architecture is defined in terms of \GlossaryHyperRef{protocols}{protocol}, \GlossaryHyperRef{profiles}{profile}, \GlossaryHyperRef{specifications}{specification} and standards, the latter of which we understand to be any other the earlier three in standardized form.
	    While we let the part about requirements be implicit, our definition should be interpreted as being compatible but more specific.
	}

\GlossaryEntry{Arrowhead}{arrowhead}

\GlossaryEntry{Asset}{asset}
An object, tangible or intangible, that is of value to an \GlossaryHyperRef{organization}{organization}.

	\GlossaryNote{RAMI4.0}{
		defines asset as an ``object which has a value for an organization''.
		Our definition should be interpreted as being equivalent.
	}

\GlossaryEntry{Authentication}{authentication}

\GlossaryEntry{Authorization}{authorization}

\GlossaryEntry{Capability}{capability}
An action that can be performed by a \GlossaryHyperRef{service provider}{provider-service}.

	\GlossaryNote{SOA-RM}{
		defines a capability as ``a real-world effect that a service provider is able to provide to a service consumer''.
		To stress that a ``real-world effect'' also could be related to a purely digital event, such as data being sent or altered, we replace the term with ``action''.
		Our definition should be interpreted as being equivalent.
	}

\GlossaryEntry{Certificate}{certificate}

\GlossaryEntry{Cloud}{cloud}

\GlossaryEntry{Cloud, Compute}{cloud-compute}

\GlossaryEntry{Cloud, Local}{cloud-local}

\GlossaryEntry{Cloud, Storage}{cloud-storage}

\GlossaryEntry{Cloud, Virtual}{cloud-virtual}

\GlossaryEntry{Codec}{codec}
%A \GlossaryHyperRef{model}{model} specifying a fundamental set of \GlossaryHyperRef{primitive}{type-primitive} and \GlossaryHyperRef{structured}{type-structured} \GlossaryHyperRef{data}{data} \GlossaryHyperRef{types}{type-data} useful for \GlossaryHyperRef{coding}{coding} data.
%A codec can be either \GlossaryHyperRef{abstract}{codec-abstract} or \GlossaryHyperRef{concrete}{codec-concrete}.

%\GlossaryEntry{Codec, Abstract}{codec-abstract}
%A \GlossaryHyperRef{codec}{codec} specified only in terms of \GlossaryHyperRef{abstract data types}{type-abstract-data}.
%Such a codec can be used to \GlossaryHyperRef{model}{model} \GlossaryHyperRef{abstract messages}{message-abstract}, serve as \GlossaryHyperRef{component}{component} of a larger abstract codec, or \GlossaryHyperRef{constrain}{constraint-model} a \GlossaryHyperRef{concrete codec}{codec-concrete}.

%\GlossaryEntry{Codec, Concrete}{codec-concrete}
%A \GlossaryHyperRef{codec}{codec} specified only in terms of \GlossaryHyperRef{concrete data types}{type-concrete-data}.
%Such a codec can be used to \GlossaryHyperRef{encode}{encoding} and \GlossaryHyperRef{decode}{decoding} \GlossaryHyperRef{concrete messages}{message-concrete}, or serve as \GlossaryHyperRef{component}{component} of a larger concrete codec.

\GlossaryEntry{Coding}{coding}
Transforming \GlossaryHyperRef{data}{data} from being expressed in one \GlossaryHyperRef{codec}{codec} into another.
See also \GlossaryHyperRef{decoding}{decoding} and \GlossaryHyperRef{encoding}{encoding}.

\GlossaryEntry{Coding, A}{coding-a}
Synonymous to \GlossaryNameRef{codec}.

\GlossaryEntry{Component}{component}
A part of a \GlossaryHyperRef{system}{system}, contributing to it facilitating its \GlossaryHyperRef{capabilities}{capability}.

	\GlossaryNote{Note 1}{
		While it is also correct that a system can be regarded as a component of a \GlossaryHyperRef{system-of-systems}{system-of-systems}, we associate the word ``component'' primarily with plain systems.
		When referring to the systems of a system-of-systems, we recommend using the words ``system'' and ``subsystem'' to avoid confusion.
	}

	\GlossaryNote{Note 2}{
		A component is practically distinct from a system by being unable to \GlossaryHyperRef{provide}{provider-service} or \GlossaryHyperRef{consume}{consumer-service} \GlossaryHyperRef{services}{service} independently.
	}

\GlossaryEntry{Concrete}{concrete}
See \GlossaryNameRef{model-concrete}.

\GlossaryEntry{Concretization, Model}{concretization-model}
Making an \GlossaryHyperRef{abstract model}{model-abstract} less abstract by \GlossaryHyperRef{specifying}{specification} details required to realize it.

\GlossaryEntry{Configuration}{configuration}

\GlossaryEntry{Constraint, Model}{constraint-model}
A \GlossaryHyperRef{model relation}{relation} where one \GlossaryHyperRef{entity}{entity-model} imposes constraints, or limits, on the other.

	\GlossaryNote{Note 1}{
		The presence of explicit model constraints enable \GlossaryHyperRef{model validation}{validation-model}.
	}

	\GlossaryNote{Note 2}{
		Perhaps a bit counterintuitively, a constraint \textit{adds} information to the constrained entity by narrowing down the the ways in which it could be realized.
		See also \GlossaryNameRef{property-model} and \GlossaryNameRef{specification}.
	}

\GlossaryEntry{Consumer, Service}{consumer-service}
A \GlossaryHyperRef{system}{system} currently \GlossaryHyperRef{invoking}{invocation-function} a \GlossaryHyperRef{function}{function} \GlossaryHyperRef{provided}{provider-service} via a \GlossaryHyperRef{service}{service}.

	\GlossaryNote{SOA-RM}{
		defines a service consumer as ``an entity which seeks to satisfy a particular need through the use [of] capabilities offered by means of a service''.
		We require that the one consuming the service is (1) a \GlossaryHyperRef{system}{system} rather than just any \GlossaryHyperRef{entity}{entity}, (2) that the \GlossaryHyperRef{capabilities}{capability} of the consumed service be exercised by invoking a function, as well as (3) that the invocation satisfies a service consumption policy.
	}

%\GlossaryEntry{Context}{context}
%A \GlossaryHyperRef{model}{model} of an \GlossaryHyperRef{abstract}{context-abstract} or \GlossaryHyperRef{concrete}{context-concrete} environment in which a \GlossaryHyperRef{model entity}{entity-model} can be situated.
%The entity in question is \GlossaryHyperRef{constrained}{constraint-model} by, or limited to, the \GlossaryHyperRef{data}{data} and other resources the context provides.

%\GlossaryEntry{Context, Abstract}{context-abstract}
%A \GlossaryHyperRef{model}{model} of an abstract environment in which an abstract \GlossaryHyperRef{entity}{entity} can be situated.
%It can serve as \GlossaryHyperRef{reference}{model-reference} for or \GlossaryHyperRef{component}{component} of another \GlossaryHyperRef{context}{context}, or as a \GlossaryHyperRef{constraint}{constraint-model} for an \GlossaryHyperRef{abstract architecture}{architecture-abstract}.

%\GlossaryEntry{Context, Concrete}{context-concrete}
%A \GlossaryHyperRef{model}{model} of a concrete environment in which a concrete \GlossaryHyperRef{entity}{entity} can be situated.
%It can serve as \GlossaryHyperRef{reference}{model-reference} for or \GlossaryHyperRef{component}{component} of another \GlossaryHyperRef{concrete context}{context-concrete}, or as a \GlossaryHyperRef{constraint}{constraint-model} for a \GlossaryHyperRef{concrete architecture}{architecture-concrete}.

\GlossaryEntry{Data}{data}
A sequence of datums, such as binary digits or other symbols, expressing a set of \GlossaryHyperRef{descriptions}{description}.

	\GlossaryNote{Note 1}{
		The descriptions can only be interpreted if the \GlossaryHyperRef{data types}{type-data} used to organize the datums are known to the interpreter.
	}

\GlossaryEntry{Decoding}{decoding}
The process through which \GlossaryHyperRef{data}{data} is transformed from being expressed in a \GlossaryHyperRef{codec}{codec} suitable for transmission or storage to another codec suitable for interpretation.

	\GlossaryNote{Note 1}{
		The operation is the reverse of \GlossaryHyperRef{encoding}{encoding}.
	}

	\GlossaryNote{Note 2}{
		The term can also be used to express the act of a human interpreting data.
	}


\GlossaryEntry{Description}{description}
\GlossaryHyperRef{Data}{data} about a \GlossaryHyperRef{entity}{entity}, concretely represented by a \GlossaryHyperRef{model}{model}, a human-readable text, or both.

\GlossaryEntry{Description, Interface Design}{description-interface-design}

\GlossaryEntry{Design}{design}

\GlossaryEntry{Design, Interface}{design-interface}

\GlossaryEntry{Device}{device}

\GlossaryEntry{Device, Human Interface (HID)}{device-human-interface}

\GlossaryEntry{Encoding}{encoding}
The process through which \GlossaryHyperRef{data}{data} is transformed from being expressed in a \GlossaryHyperRef{codec}{codec} suitable for interpretation to another codec suitable for transmission or storage.

	\GlossaryNote{Note 1}{
		The operation is the reverse of \GlossaryHyperRef{decoding}{decoding}.
	}

	\GlossaryNote{Note 2}{
		The term can also be used to express the act of a human recording data.
	}

\GlossaryEntry{Encoding, An}{encoding-an}
Synonymous to \GlossaryNameRef{codec}.

\GlossaryEntry{Entity}{entity}
An object, tangible or intangible, that is uniquely \GlossaryHyperRef{identifiable}{identity}.

	\GlossaryNote{Note 1}{
		An entity being uniquely identifiable does not necessarily mean that it is associated with a \GlossaryHyperRef{certificate}{certificate} or \GlossaryHyperRef{identifier}{identifier}.
		It only means that a \GlossaryHyperRef{description}{description} can be rendered that unambigously refers to the entity in question.
	}

	\GlossaryNote{RAMI4.0}{
		defines entity as an ``uniquely identifiable object which is administered in the information world due to its importance''.
		Our definition should be interpreted as being equivalent.
	}

	\GlossaryNote{SOA-RM}{
	    mentions the word ``entity'' nine times, but provides no explicit definition.
	    We assume their definition to match that of a regular English dictionary, such as ``something that has separate and distinct existence and objective or conceptual reality'' \cite{webster2021entity}.
		Our definition should be interpreted as being equivalent.
	}

\GlossaryEntry{Entity, Model}{entity-model}
An \GlossaryHyperRef{entity}{entity} represented as part of a \GlossaryHyperRef{model}{model}.

\GlossaryEntry{Framework}{framework}

\GlossaryEntry{Function}{function}
An \GlossaryHyperRef{invocable}{invocation-function} \GlossaryHyperRef{subroutine}{subroutine} adhering to the \GlossaryHyperRef{protocol}{protocol} established by a certain \GlossaryHyperRef{signature}{signature-functon}.

\GlossaryEntry{HID}{hid} See \GlossaryNameRef{device-human-interface}.

\GlossaryEntry{Industry 4.0}{industry40}

\GlossaryEntry{Identification}{identification}
The process through which an \GlossaryHyperRef{entity}{entity} collects and verifies the \GlossaryHyperRef{identity}{identity} of another entity. 

\GlossaryEntry{Identifier}{identifier}
\GlossaryHyperRef{Data}{data} associated with an \GlossaryHyperRef{entity}{entity} that allows for it to be \GlossaryHyperRef{identified}{identification}.

\GlossaryEntry{Identity}{identity}
The aspect or aspects, such as \GlossaryHyperRef{identifiers}{identifier}, that makes an \GlossaryHyperRef{entity}{entity} distinct from all other entities.

\GlossaryEntry{Interface}{interface}

\GlossaryEntry{Interface, Administrative}{interface-administrative}

\GlossaryEntry{Interface, Application Network (ANI)}{interface-application-network}

\GlossaryEntry{Interface, Application Programming (API)}{interface-application-programming}

\GlossaryEntry{Interface, Management}{interface-management}

\GlossaryEntry{Interface, Network}{interface-network}

\GlossaryEntry{Interface, Operator}{interface-operator}

\GlossaryEntry{Invocation, Function}{invocation-function}
The attempt to trigger the \GlossaryHyperRef{capabilities}{capability} associated with a \GlossaryHyperRef{function}{function} by sending a \GlossaryHyperRef{message}{message} to the \GlossaryHyperRef{service}{service} through which the function is \GlossaryHyperRef{provided}{provider-service}.
See also \GlossaryNameRef{signature-function}. TODO: Make use of subroutine definition to simplify this.

\GlossaryEntry{Manager}{manager}

\GlossaryEntry{Management}{management}

\GlossaryEntry{Message}{message}

\GlossaryEntry{Metadata}{metadata}

\GlossaryEntry{Model}{model}
A representation of facts in the form of a graph, consisting of \GlossaryHyperRef{entities}{entity-model}, \GlossaryHyperRef{relations}{relation-model} and \GlossaryHyperRef{properties}{property-model}.
Models can be expressed as visual diagrams, text or binary data.
They can be human-readable, machine-readable, or both.
They can be either \GlossaryHyperRef{abstract}{model-abstract} or \GlossaryHyperRef{concrete}{model-concrete}.
See also \GlossaryNameRef{model-reference}.

\GlossaryEntry{Model, Abstract}{model-abstract}
A \GlossaryHyperRef{model}{model} that is insufficiently specified to be possible to realize as a concrete artifact.
Abstract models can be used for imposing \GlossaryHyperRef{constraints}{constraint-model} on other models, which creates room for producing \GlossaryHyperRef{reference arhictecures}{architecture-reference}, abstract \GlossaryHyperRef{service descriptions}{service}, and so on.

\GlossaryEntry{Model, Concrete}{model-concrete}
A \GlossaryHyperRef{model}{model} that is sufficiently specified to be possible to realize as a concrete artifact, such as a \GlossaryHyperRef{protocol}{protocol} or \GlossaryHyperRef{software}{software}.

\GlossaryEntry{Model, Reference}{model-reference}
An \GlossaryHyperRef{abstract model}{model-abstract} defining technical concepts of fundamental importance to a specific problem domain.
See also Section \ref{sec:introduction:scope}.

	\GlossaryNote{RAMI4.0}{
		defines reference model as a ``model that is generally used and recognized as being suitable (has recommendation character) for deriving
specific models.
		We adds that the model must be abstract and define fundamental concepts for a problem domain, setting it apart as more significant than other abstract models.
	}

	\GlossaryNote{SOA-RM}{
		defines reference model as ``an abstract framework for understanding significant relationships among the entities of some environment that enables the development of specific architectures using consistent standards or specifications supporting that environment''.
		It further clarifies that a ``reference model consists of a minimal set of unifying concepts, axioms and relationships within a particular problem domain, and is independent of specific standards, technologies, implementations, or other concrete details''.
		Our definition should be interpreted as being equivalent.
	}

\GlossaryEntry{Model, Specific}{model-specific}
See \GlossaryNameRef{model-concrete}.

\GlossaryEntry{Operator}{operator}

\GlossaryEntry{Organization}{organization}

\GlossaryEntry{Policy}{policy}

\GlossaryEntry{Policy, Service Consumption}{policy-service-consumption}

\GlossaryEntry{Process, Application}{process-application}

\GlossaryEntry{Profile}{profile}
A \GlossaryHyperRef{model}{model} imposing constraints on a \GlossaryHyperRef{protocol}{protocol}.
A profile could specify a \GlossaryHyperRef{protocol stack}{stack-protocol}, certain message semantics, how authentication and authorization are to be carried out, among many other possible examples.

\GlossaryEntry{Property, Model}{property-model}
Named \GlossaryHyperRef{data}{data} associated with either a model \GlossaryHyperRef{entity}{entity-model} or \GlossaryHyperRef{relation}{relation-model}.

\GlossaryEntry{Protocol}{protocol}
A \GlossaryHyperRef{model}{model} of communication defined in terms of \GlossaryHyperRef{messages}{message}.
See also \GlossaryNameRef{stack-protocol}.

\GlossaryEntry{Provider, Service}{provider-service}
A \GlossaryHyperRef{system}{system} that makes \GlossaryHyperRef{services}{service} available for \GlossaryHyperRef{consumption}{consumer-service} by any systems able to satisfy its \GlossaryHyperRef{consumption policies}{policy-service-consumption}.

	\GlossaryNote{SOA-RM}{
		defines a service provider as ``an entity (person or organization) that offers the use of capabilities by means of a service''.
		Our definition is more specific in that it requires the \GlossaryHyperRef{entity}{entity} be a system.
	}

\GlossaryEntry{Relation, Model}{relation-model}
A uni-directional association of two \GlossaryHyperRef{entities}{entity-model} part of the same \GlossaryHyperRef{model}{model}.

\GlossaryEntry{Service}{service}
A set of \GlossaryHyperRef{functions}{function} that can be \GlossaryHyperRef{provided}{provider-service} via an \GlossaryHyperRef{interface}{interface}.

	\GlossaryNote{RAMI4.0}{
		defines a service as ``separate scope of functions offered by an entity or organization via interfaces''.
		Our definition restricts service provision to systems.
	}

	\GlossaryNote{SOA-RM}{
		defines a service as ``the means by which the needs of a consumer are brought together with the capabilities of a provider''.
		Our definition is more specific about how the \GlossaryHyperRef{capabilities}{capability} of a service are made available.
	}

\GlossaryEntry{Session}{session}

\GlossaryEntry{Shell, Administrative}{shell-administrative}

\GlossaryEntry{Signature, Function}{signature-function}
A \GlossaryHyperRef{model}{model} specifying the \GlossaryHyperRef{type}{type-data} of the \GlossaryHyperRef{message}{message} a given \GlossaryHyperRef{function}{function} accepts when \GlossaryHyperRef{invoked}{invocation-function}, as well as the types of any messages it could return in response.

	\GlossaryNote{Note 1}{
		This means that a function signature establishes a \GlossaryHyperRef{protocol}{protocol} for a certain \GlossaryHyperRef{service}{service} function.
	}

\GlossaryEntry{Software}{software}

\GlossaryEntry{SoLC}{solc} See \GlossaryNameRef{system-of-local-clouds}.

\GlossaryEntry{SoS}{sos} See \GlossaryNameRef{system-of-systems}.

\GlossaryEntry{Specification}{specification}
A \GlossaryHyperRef{model}{model} of \GlossaryHyperRef{properties}{property-model}, constituting \GlossaryHyperRef{constraints}{constraint-model} a target model must satisfy.
A specification may refer to \GlossaryHyperRef{prototols}{protocol}, \GlossaryHyperRef{profiles}{profile}, \GlossaryHyperRef{services}{service}, as well as other things the entity in question must conform to or provide.

\GlossaryEntry{Stack, Protocol}{stack-protocol}

\GlossaryEntry{Standard}{standard}

\GlossaryEntry{Stakeholder}{stakeholder}

\GlossaryEntry{Subroutine}{subroutine}
A component of a \GlossaryHyperRef{software artifact}{software} that may exercise one or more \GlossaryHyperRef{service}{service} \GlossaryHyperRef{capabilities}{capability} if \GlossaryHyperRef{invoked}{invocation-function} via a \GlossaryHyperRef{function}{function}.

\GlossaryEntry{Subsystem}{subsystem} See \GlossaryNameRef{component}.

\GlossaryEntry{System}{system}
An \GlossaryHyperRef{entity}{entity} capable of \GlossaryHyperRef{providing services}{provider-service}, \GlossaryHyperRef{consuming services}{consumer-service}, or both.

	\GlossaryNote{Note 1}{
		The word ``system'' is more generally understood to be very inclusive, expressing the larger idea of connected \GlossaryHyperRef{components}{component} facilitating one or more \GlossaryHyperRef{capabilities}{capability}.
		From the perspective of Arrowhead, however, capabilities can only be \GlossaryHyperRef{invoked}{invocation-function} through \GlossaryHyperRef{services}{service}, which means that a system unable to provide or consume services can only be described as a component of another system.
	}

	\GlossaryNote{Note 2}{
		A system is practically distinct from a \GlossaryHyperRef{system-of-systems}{system-of-systems} by being represented only by a single \GlossaryHyperRef{identity}{identity}.
	}

\GlossaryEntry{System-of-Local-Clouds (SoLC)}{system-of-local-clouds}
A set of \GlossaryHyperRef{local clouds}{cloud-local} that \GlossaryHyperRef{consume}{consumer-service} each other's \GlossaryHyperRef{services}{service} in order to facilitate a \GlossaryHyperRef{capability}{capability} none of the constituent local clouds could \GlossaryHyperRef{provide}{provider-service} on its own.

\GlossaryEntry{System-of-Systems (SoS)}{system-of-systems}
A set of \GlossaryHyperRef{systems}{system} that \GlossaryHyperRef{consume}{consumer-service} each other's \GlossaryHyperRef{services}{service} in order to facilitate a \GlossaryHyperRef{capability}{capability} none of the constituent systems could \GlossaryHyperRef{provide}{provider-service} on its own.

\GlossaryEntry{Token}{token} See \GlossaryNameRef{token-authentication}.

\GlossaryEntry{Token, Authentication}{token-authentication}

\GlossaryEntry{Type, Data}{type-data}

\GlossaryEntry{Type, Enumerating}{type-enumerating}

\GlossaryEntry{Type, Primitive}{type-primitive}

\GlossaryEntry{Type, Structured}{type-structured}

\GlossaryEntry{Type, Union}{type-union}

\GlossaryEntry{User}{user}

\GlossaryEntry{Validation, Model}{validation-model}
The process through which it is determined if a \GlossaryHyperRef{model}{model} satisfies all of its \GlossaryHyperRef{constraints}{constraint-model}.

}