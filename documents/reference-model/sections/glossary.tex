% Copyright (c) 2021-10-07 Eclipse Arrowhead Project
%
% This program and the accompanying materials are made available under the
% terms of the Eclipse Public License 2.0 which is available at
% http://www.eclipse.org/legal/epl-2.0.
%
% SPDX-License-Identifier: EPL-2.0

This section provides an alphabetically sorted list of all significant terms introduced or named in this document.
Each term consisting of more than one word is sorted by its final, or qualified, word.
This means that the definition of \GlossaryHyperRef{description-service}{service description}, for example, is found at \GlossaryNameRef{description-service}.

Many of the definitions are amended with notes and references to RAMI4.0 \cite{adolphs2016reference}, SOA-RM \cite{mackenzie2006reference} and IoTA:AF \cite{delsing2017iot}, which are always listed after the definition they amend.
Regular notes are numbered, while those making a comment on a definition in RAMI4.0, SOA-RM or IoTA:AF are introduced with the three abbreviations just listed.

{

\newcommand{\GlossaryEntry}[3][]{\subsubsection*{#3\IfStrEq{#1}{}{}{ {\normalfont \textit{#1}}}}\label{sec:glossary:#2}}
\newcommand{\GlossaryNote}[2]{\begin{minipage}[b]{\dimexpr\linewidth-0.5cm\relax}\vspace*{0.33cm}\footnotesize{\textbf{#1}\ #2}\end{minipage}}

\GlossaryEntry{abstract}{Abstract}
See \GlossaryNameRef{model-abstract}.

\GlossaryEntry{architecture}{Architecture}
A \GlossaryHyperRef{model-concrete}{concrete model} of a \GlossaryHyperRef{system-of-systems}{system-of-systems} defined in terms of certain \GlossaryHyperRef{model-reference}{reference models}, \GlossaryHyperRef{architecture-reference}{reference architectures} and other concrete architectures.
See also Section \ref{sec:introduction:scope}.

	\GlossaryNote{RAMI4.0}{
	    defines architecture as the ``combination of elements of a model based on principles and rules for constructing, refining and using it''.
		We consider ``combinations of elements of a model'' to be a ``model of a system-of-systems'' and to be ``based on principles and rules for constructing, refining and using it'' as building upon reference models and architectures.
		Our definition should be interpreted as being compatible but more specific.
	}

	\GlossaryNote{SOA-RM}{
		defines software architecture as ``the structure or structures of an information system consisting of entities and their externally visible properties, and the relationships among them''.
		That definition is equivalent to our definition of \GlossaryHyperRef{model}{model}, with the exception that the thing being modeled has to be an information system.
		As our definition is concerned with a model and a system-of-systems, which must be an information system, we regard out definition as compatible but more specific.
	}

\GlossaryEntry{architecture-reference}{Architecture, Reference}
A significantly useful \GlossaryHyperRef{model-abstract}{abstract model} of a \GlossaryHyperRef{system-of-systems}{system-of-systems} defined in terms of certain \GlossaryHyperRef{model-reference}{reference models} and other reference architectures.
See also Section \ref{sec:introduction:scope}.

	\GlossaryNote{RAMI4.0}{
		defines reference architecture as a ``model for an architecture description (for I[ndustry ]4.0) which is generally used and recognized as being suitable (has reference character)''.
		We consider a ``model for an architecture description'' to be an ``abstract model of a system-of-systems''.
		Our definition should be interpreted as being compatible but more specific.
	}

	\GlossaryNote{SOA-RM}{
	    defines reference architecture as ``an architectural design pattern that indicates how an abstract set of mechanisms and relationships realizes a predetermined set of requirements''.
	    While we let the part about requirements be implicit, our definition should be interpreted as being compatible but more specific.
	}

\GlossaryEntry{architecture-service-oriented}{Architecture, Service-Oriented (SOA)}
An \GlossaryHyperRef{architecture}{architecture} concerned with \GlossaryHyperRef{service}{service} \GlossaryHyperRef{provider-service}{provision} and \GlossaryHyperRef{consumer-service}{consumption}.

	\GlossaryNote{Note 1}{
		Any architecture building upon the \GlossaryHyperRef{model-reference}{reference model} of this document will become service-oriented.
		See also Section \ref{sec:introduction}.
	}

\GlossaryEntry{arrowhead}{Arrowhead}
See \GlossaryNameRef{framework-arrowhead}.

\GlossaryEntry{artifact}{Artifact}
A thing or object, tangible or intangible.

\GlossaryEntry{asset}{Asset}
An \GlossaryHyperRef{artifact}{artifact} that is of value to a person or \GlossaryHyperRef{organization}{organization}.

	\GlossaryNote{RAMI4.0}{
		defines asset as an ``object which has a value for an organization''.
		Our definition should be interpreted as being equivalent.
	}

\GlossaryEntry{capability}{Capability}
A task, of any nature, that can be performed by a \GlossaryHyperRef{provider-service}{service provider}.

	\GlossaryNote{Note 1}{
		The term must be interpreted as being very general.
		It includes the abilities of consuming and providing services, reading from sensors, triggering actuators, among many other possible examples.
	}

	\GlossaryNote{SOA-RM}{
		defines a capability as ``a real-world effect that a service provider is able to provide to a service consumer''.
		Our definition should be interpreted as being equivalent.
	}

\GlossaryEntry{cloud}{Cloud}
A \GlossaryHyperRef{system-of-systems}{system-of-systems} able independently execute given tasks through the use of pools of owned resources.

	\GlossaryNote{Note 1}{
		When the term ``cloud'' is used elsewhere, it often refers to clouds with only virtual resources, such as compute, storage and software-defined network utilities.
		Here, we refer to such clouds as \GlossaryHyperRef{cloud-virtual}{virtual clouds}.
		By making the unqualified word ``cloud'' less specific, it becomes more clear how our \GlossaryHyperRef{cloud-local}{local cloud} concept shares similarities with other types of clouds.
	}

\GlossaryEntry{cloud-local}{Cloud, Local}
A \GlossaryHyperRef{cloud}{cloud} bound to a physical location due to its acting on or producing physical \GlossaryHyperRef{artifact}{artifacts}.

	\GlossaryNote{IoTA:AF}{
		provides an introduction to the local cloud concept in its second chapter, as well as an architectural definition in its third chapter.
		The following is an excerpt from the introduction:
		\begin{quote}
		The local cloud concept takes the view that specific geographically local automation tasks should be encapsulated and protected.
		These tasks have strong requirements on real time, ease of engineering, operation and maintenance, and system security and safety.
		The local cloud idea is to let the local cloud include the devices and systems required to perform the desired automation tasks, thus providing a local ``room'' which can be protected from outside activities.
		In other words, the cloud will provide a boundary to the open internet, thus aiming to protect the internal of the local cloud from the open internet.
		\end{quote}
		The third chapter contains the following:
		\begin{quote}
		In the Arrowhead Framework context a local cloud is defined as a self-contained network with the three mandatory core systems deployed and at least one application system deployed [...]
		\end{quote}
		Both of these descriptions are practical, in the sense that they emphasize engineering aspects.
		However, as this document is a reference model, engineering aspects are out of scope.
		To be able to use the above descriptions, we look for indications of what a local cloud is, rather than what requirements such should fulfill.
		The terms ``geographically local'', ``room'' and ``boundary'' clearly highlight the physicality of the local cloud itself, while the depiction of ``devices'' performing ``automation tasks'' makes it apparent that some kind of physical activity is involved, such as manufacturing.
		Finally, the local cloud being ``encapsulated'', ``protected'' and ``self-contained'' indicates that it is understood to exhibit a degree of independence with respect to the tasks it is given, which we expect all kinds of clouds to exhibit.
		Our definition should be interpreted as a summation of these characteristics.
	}

\GlossaryEntry{cloud-local-automation}{Cloud, Local Automation}
See \GlossaryNameRef{cloud-local}.

\GlossaryEntry{cloud-virtual}{Cloud, Virtual}
A \GlossaryHyperRef{cloud}{cloud} unbound by physical location by only acting on or producing virtual \GlossaryHyperRef{artifact}{artifacts}.

	\GlossaryNote{Note 1}{
		Examples of virtual resources could be compute, storage and software-defined network utilities.
	}

\GlossaryEntry{codec}{Codec}
A \GlossaryHyperRef{model-concrete}{concrete} \GlossaryHyperRef{type-data}{data type} used to structure \GlossaryHyperRef{data}{data} for transmission, storage and/or interpretation.

\GlossaryEntry[(verb)]{code}{Code}
Transforming \GlossaryHyperRef{data}{data} from being expressed in one \GlossaryHyperRef{codec}{codec} into another.
See also \GlossaryHyperRef{decode}{decode} and \GlossaryHyperRef{encode}{encode}.

\GlossaryEntry{coding}{Coding}
Synonymous to \GlossaryNameRef{codec}.

\GlossaryEntry{component}{Component}
A part of a \GlossaryHyperRef{system}{system}, contributing to it facilitating its \GlossaryHyperRef{capability}{capabilities}.
Compare with \GlossaryNameRef{subsystem}.

	\GlossaryNote{Note 1}{
		While ``component'' generally is a rather open-ended term, in the context of Arrowhead it gains the rather specific meaning of being a constituent of a system.
		If referring to a constituent of a \GlossaryHyperRef{system-of-systems}{system-of-systems}, prefer use of the words ``system'' and ``subsystem''.
	}

	\GlossaryNote{Note 2}{
		A component is practically distinct from a system by being unable to \GlossaryHyperRef{provider-service}{provide} or \GlossaryHyperRef{consumer-service}{consume} \GlossaryHyperRef{service}{services} independently.
	}

	\GlossaryNote{RAMI4.0}{
		makes no practical distinction between components and systems, as is done here.
		See \GlossaryNameRef{system} for more details.
	}

\GlossaryEntry{component-industry40}{Component, Industry 4.0}
See \GlossaryNameRef{system-industry40}.

	\GlossaryNote{RAMI4.0}{
		introduces this concept.
		To make it align better with our definitions, we refer to such components as Industry 4.0 \textit{systems}.
	}

\GlossaryEntry{concrete}{Concrete}
See \GlossaryNameRef{model-concrete}.

\GlossaryEntry{concretization}{Concretization}
Making an \GlossaryHyperRef{model-abstract}{abstract model} less abstract by specifying details required to realize it.

\GlossaryEntry{configuration}{Configuration}
A set of changeable \GlossaryHyperRef{property}{properties} that directly influence the \GlossaryHyperRef{capability}{capabilities} of a \GlossaryHyperRef{provider-service}{service provider}.

\GlossaryEntry{configure}{Configure}
To change a \GlossaryHyperRef{configuration}{configuration}.

\GlossaryEntry{constraint}{Constraint}
A \GlossaryHyperRef{property}{property} that imposes constraints, or limits, on an \GlossaryHyperRef{entity}{entity} or \GlossaryHyperRef{relationship}{relationship}.

	\GlossaryNote{Note 1}{
		The presence of constraints enable \GlossaryHyperRef{validation}{validation}.
	}

	\GlossaryNote{Note 2}{
		Perhaps a bit counterintuitively, a constraint \textit{adds} information to its target by reducing the ways in which it could be realized.
	}

\GlossaryEntry{consumer-service}{Consumer, Service}
A \GlossaryHyperRef{system}{system} currently \GlossaryHyperRef{invocation-function}{invoking} a \GlossaryHyperRef{function}{function} \GlossaryHyperRef{provider-service}{provided} via a \GlossaryHyperRef{service}{service}.

	\GlossaryNote{SOA-RM}{
		defines a service consumer as ``an entity which seeks to satisfy a particular need through the use [of] capabilities offered by means of a service''.
		We require that the one consuming the service is (1) a \GlossaryHyperRef{system}{system} rather than just any \GlossaryHyperRef{entity}{entity}, as well as (2) that the \GlossaryHyperRef{capability}{capabilities} of the consumed service be exercised by invoking a function.
	}

\GlossaryEntry{data}{Data}
A sequence of \GlossaryHyperRef{datum}{datums} recording a set of \GlossaryHyperRef{description}{descriptions} via the structure superimposed by a \GlossaryHyperRef{type-data}{data type}.

	\GlossaryNote{Note 1}{
		Let us assume that some data is going to be sent to a drilling machine.
		The type associated with the data requires that it always consists of 8 bits, organized such that the first 4 bits indicate the speed of drilling in multiples of 100 rotations per minute, while the latter 4 determine how much to lower the drill in multiples of 5 millimeters.
		A \GlossaryHyperRef{state}{state} that could be expressed with those 8 bits is \texttt{0100 1101}.
		If each of the two sequences of 4 bits is treated as a big-endian integer with base 2, they record $4$ and $13$ in decimal notation.
		This would indicate that the drill should spin at $4 * 100 = 400$ rotations per minute and be lowered $13 * 5 = 65$ millimeters.
	}

	\GlossaryNote{Note 2}{
		Without knowledge of the types and context associated with some data, that data cannot be interpreted.
	}

\GlossaryEntry{datum}{Datum}
A variable expressing one out of a set of possible values.
See also \GlossaryNameRef{state}.

	\GlossaryNote{Note 1}{
		A familiar example of a datum may be the bit, or binary digit.
		Its possible set of symbols is $\{0, 1\}$.
	}

\GlossaryEntry{decode}{Decode}
The act of transforming \GlossaryHyperRef{data}{data} from being expressed in a \GlossaryHyperRef{codec}{codec} suitable for transmission or storage to another codec suitable for interpretation.

	\GlossaryNote{Note 1}{
		Decoding is the reverse of \GlossaryHyperRef{encode}{encoding}.
	}

	\GlossaryNote{Note 2}{
		The term can also be used to express the act of a human interpreting data.
	}

\GlossaryEntry{description}{Description}
Facts about an \GlossaryHyperRef{entity}{entity} or \GlossaryHyperRef{entity-class-of}{class of entities}, expressed in the form of a \GlossaryHyperRef{model}{model}, a text, or both.

\GlossaryEntry{description-design}{Description, Design}
A \GlossaryHyperRef{description}{description} of a \GlossaryHyperRef{design}{design}. TODO

	\GlossaryNote{Note 1}{
		Since a design also is a form of description, a design description becomes a kind of \GlossaryHyperRef{metadata}{metadata}.
	}

\GlossaryEntry{description-interface-design}{Description, Interface Design}
A \GlossaryHyperRef{description-design}{design description} of an \GlossaryHyperRef{interface}{interface}.

	\GlossaryNote{Note 1}{
		In the context of Arrowhead, the term is only used to refer to design descriptions of \GlossaryHyperRef{interface-service}{interfaces}.
	}

\GlossaryEntry{description-service}{Description, Service}
A \GlossaryHyperRef{description}{description} of a \GlossaryHyperRef{service}{service}, especially in terms of its utility and its \GlossaryHyperRef{interface-service}{interfaces}.

	\GlossaryNote{SOA-RM}{
		defines service description as ``the information needed in order to use, or consider using, a service''.
		Our definition should be interpreted as being equivalent.
	}

\GlossaryEntry[(noun)]{design}{Design}
Every document, \GlossaryHyperRef{model}{model} and other record \GlossaryHyperRef{description}{describing} how a certain \GlossaryHyperRef{artifact}{artifact} can be \GlossaryHyperRef{implementation}{implemented} or realized.

\GlossaryEntry{design-interface}{Design, Interface}
The \GlossaryHyperRef{design}{design} of an \GlossaryHyperRef{interface}{interface}, which must specify how it enables communication and under what conditions.

\GlossaryEntry{device}{Device}
A physical \GlossaryHyperRef{entity}{entity} capable of hosting \GlossaryHyperRef{system}{systems}.

	\GlossaryNote{IoTA:AF}{
		provides the following architectural definition of device in its third chapter:
		\begin{quote}
		An Arrowhead compliant device is a piece of equipment, machine, hardware, etc. with computational, memory and communication capabilities which hosts one or several Arrowhead Framework systems and can be bootstrapped in an Arrowhead local cloud [...] Any other device, equipment, machine, hardware component etc. is non-Arrowhead compliant.
		\end{quote}
		The definition provided here should be interpreted as being equivalent.
	}

\GlossaryEntry{device-human-interface}{Device, Human Interface (HID)}
A \GlossaryHyperRef{device}{device} that provides sensors and actuators that together make up an \GlossaryHyperRef{interface}{interface} through which a human can exchange messages with one or more \GlossaryHyperRef{system}{systems}.

\GlossaryEntry{encode}{Encode}
The act of transforming \GlossaryHyperRef{data}{data} from being expressed in a \GlossaryHyperRef{codec}{codec} suitable for interpretation to another codec suitable for transmission or storage.

	\GlossaryNote{Note 1}{
		Encoding is the reverse of \GlossaryHyperRef{decode}{decoding}.
	}

	\GlossaryNote{Note 2}{
		The term can also be used to express the act of a human recording data.
	}

\GlossaryEntry{encoding}{Encoding}
Synonymous to \GlossaryNameRef{codec}.

\GlossaryEntry{entity}{Entity}
An \GlossaryHyperRef{artifact}{artifact} that is uniquely \GlossaryHyperRef{identity}{identifiable}.

	\GlossaryNote{Note 1}{
		An entity being uniquely identifiable does not necessarily mean that it is associated with a certificate or \GlossaryHyperRef{identifier}{identifier}.
		It only means that a \GlossaryHyperRef{description}{description} can be rendered that unambigously refers to the entity in question.
	}

	\GlossaryNote{RAMI4.0}{
		defines entity as an ``uniquely identifiable object which is administered in the information world due to its importance''.
		Our definition should be interpreted as being equivalent.
	}

	\GlossaryNote{SOA-RM}{
	    mentions the word ``entity'' nine times, but provides no explicit definition.
	    We assume their definition to match that of a regular English dictionary, such as ``something that has separate and distinct existence and objective or conceptual reality'' \cite{webster2021entity}.
		Our definition should be interpreted as being equivalent.
	}

\GlossaryEntry{entity-class-of}{Entity, Class of}
A set of \GlossaryHyperRef{entity}{entities} that share a common \GlossaryHyperRef{property}{property}.

\GlossaryEntry{entity-systemic}{Entity, Systemic}
An \GlossaryHyperRef{entity}{entity} being either a \GlossaryHyperRef{component}{component}, a \GlossaryHyperRef{system}{system}, a \GlossaryHyperRef{system-of-systems}{system-of-systems}, a \GlossaryHyperRef{cloud}{cloud}, or any kind of \GlossaryHyperRef{system-of-clouds}{system-of-clouds}.

\GlossaryEntry{framework}{Framework}
A set of assumptions, concepts, values and practices that frame a certain problem domain.

	\GlossaryNote{SOA-RM}{
		defines framework as ``a set of assumptions, concepts, values, and practices that constitutes a way of viewing the current environment''.
		Our definition should be interpreted as being equivalent.
	}

\GlossaryEntry{framework-arrowhead}{Framework, Arrowhead}
Either of the \GlossaryHyperRef{framework}{framework of ideas} and the \GlossaryHyperRef{framework-software}{framework of software} maintained by the Arrowhead project.

\GlossaryEntry{framework-software}{Framework, Software}
A set of software specifications, \GlossaryHyperRef{implementation-software}{implementations} and other \GlossaryHyperRef{artifact}{artifacts} meant to help address the problem domain of a certain \GlossaryHyperRef{framework}{framework}.

\GlossaryEntry{function}{Function}
The realization of the \GlossaryHyperRef{protocol}{protocol} established by a certain \GlossaryHyperRef{signature-function}{function signature}.

\GlossaryEntry{hid}{HID}
See \GlossaryNameRef{device-human-interface}.

\GlossaryEntry{industry40}{Industry 4.0}
The fourth industrial paradigm, primarily characterized by high degrees of computerization, digitalization and interconnectivity.
See also Section \ref{sec:arrowhead:industry40}.

\GlossaryEntry{identification}{Identification}
The process through which an \GlossaryHyperRef{entity}{entity} collects and verifies the \GlossaryHyperRef{identity}{identity} of another entity. 

\GlossaryEntry{identifier}{Identifier}
\GlossaryHyperRef{data}{Data} associated with an \GlossaryHyperRef{entity}{entity} that allows for it to be \GlossaryHyperRef{identification}{identified}.

\GlossaryEntry{identity}{Identity}
The aspect or aspects, such as \GlossaryHyperRef{identifier}{identifiers}, that makes an \GlossaryHyperRef{entity}{entity} distinct from all other entities.

\GlossaryEntry{implementation}{Implementation}
The realization of a \GlossaryHyperRef{design}{design} as a set of \GlossaryHyperRef{artifact}{artifacts}.

\GlossaryEntry{implementation-software}{Implementation, Software}
An \GlossaryHyperRef{implementation}{implementation} comprised of executable software \GlossaryHyperRef{artifact}{artifacts}.

\GlossaryEntry{interface}{Interface}
An \GlossaryHyperRef{entity}{entity} able to mediate communications that satisfy a certain set of \GlossaryHyperRef{constraint}{constraints}.

	\GlossaryNote{Note 1}{
		The are many possible constraints, or prerequisites, to communication via an interface.
		Physical links may have to exist and be online, certain \GlossaryHyperRef{protocol}{protocols} may have to used, the receiver of a communication may have to be ready, and so on.
	}

\GlossaryEntry{interface-service}{Interface, Service}
An \GlossaryHyperRef{interface}{interface} through which a certain \GlossaryHyperRef{service}{service} can be \GlossaryHyperRef{consumer-service}{consumed}.

	\GlossaryNote{SOA-RM}{
		defines service interface as ``the means by which the underlying capabilities of a service are accessed''.
		Our definition should be interpreted as being equivalent.
	}

\GlossaryEntry{interface-system-management}{Interface, System Management}
An \GlossaryHyperRef{interface}{interface} through which a certain \GlossaryHyperRef{system}{system} can be \GlossaryHyperRef{management-system}{managed}.

	\GlossaryNote{RAMI4.0}{
		introduces the ``component manager'', which is an ``organizer of autonomous administration and access to resources of the relevant I[ndustry ]4.0 component, such as the I[ndustry ]4.0 component itself, object, technical functionality, virtual representation''.
		Due to the difference in interpretation of the terms ``component'' and ``system'', as well as to make it more explicit that the ``manager'' in question is not a human, we use the term ``system management interface'' to refer to the same idea.
		Our definition should be interpreted as being equivalent to that of the RAMI4.0 component manager, with the difference that we do not require that such a manager be associated with an \GlossaryHyperRef{system-industry40}{Industry 4.0 system}.
	}

\GlossaryEntry{invocation-function}{Invocation, Function}
The attempt to exercise the \GlossaryHyperRef{capability}{capabilities} of a \GlossaryHyperRef{provider-service}{service provider} by sending a \GlossaryHyperRef{message}{message} to one of its \GlossaryHyperRef{function}{functions}.

\GlossaryEntry{manager-component}{Manager, Component}
See \GlossaryNameRef{interface-system-management}.

	\GlossaryNote{RAMI4.0}{
		introduces this concept.
		We refer to such managers as \textit{system management interfaces}.
	}

\GlossaryEntry{management-system}{Management, System}
The activity of \GlossaryHyperRef{configure}{configuring} or retrieving restricted \GlossaryHyperRef{data}{data} from a \GlossaryHyperRef{system}{system}.

	\GlossaryNote{Note 1}{
		Both of these activities may be performed via \GlossaryHyperRef{service}{services} \GlossaryHyperRef{provider-service}{provided} by the managed system in question.
		The definition does, however, open up for management to happen some other way, such as by physically interacting with a \GlossaryHyperRef{device}{device} hosting systems.
	}

\GlossaryEntry{manifest}{Manifest}
A set of \GlossaryHyperRef{metadata}{metadata} about relevant aspects of a particular \GlossaryHyperRef{entity}{entity}.

	\GlossaryNote{RAMI4.0}{
		defines manifest as an ``externally accessible, defined set of meta-information on the functional and non-functional properties of the relevant [Industry ]4.0 component''.
		Our definition is more inclusive, as it allows for any entity to be the subject of the manifest, not only \GlossaryHyperRef{component-industry40}{Industry 4.0 components}.
	}

\GlossaryEntry{manifest-system}{Manifest, System}
A \GlossaryHyperRef{manifest}{manifest} \GlossaryHyperRef{description}{describing} a \GlossaryHyperRef{system}{system}.

\GlossaryEntry[(noun)]{message}{Message}
\GlossaryHyperRef{data}{Data} sent or received via a \GlossaryHyperRef{interface-service}{service interface}.

	\GlossaryNote{Note 1}{
		In the context of Arrowhead, messages are only sent to \GlossaryHyperRef{invocation-function}{invoke} the \GlossaryHyperRef{function}{functions} of \GlossaryHyperRef{service}{services} \GlossaryHyperRef{provider-service}{provided} by \GlossaryHyperRef{system}{systems}.
	}

\GlossaryEntry{metadata}{Metadata}
\GlossaryHyperRef{data}{Data} \GlossaryHyperRef{description}{describing} other data.

\GlossaryEntry{model}{Model}
A representation of facts in the form of a graph, consisting of \GlossaryHyperRef{entity}{entities}, \GlossaryHyperRef{relationship}{relationships} and \GlossaryHyperRef{property}{properties}.

	\GlossaryNote{Note 1}{
		Models can be expressed or recorded in many ways, including as visual diagrams, spoken words, text and binary data.
	}

	\GlossaryNote{Note 2}{
		Models can be human-readable, machine-readable, or both.
	}

\GlossaryEntry{model-abstract}{Model, Abstract}
A \GlossaryHyperRef{model}{model} that is \textit{insufficiently} specified to be possible to realize as the \GlossaryHyperRef{artifact}{artifact} it represents.

	\GlossaryNote{Note 1}{
		Abstract models can be referred to by other models, serving as a form of \GlossaryHyperRef{constraint}{constraint}.
		They are commonly used to enforce a degree of uniformity across multiple other models.
	}

\GlossaryEntry{model-concrete}{Model, Concrete}
A \GlossaryHyperRef{model}{model} that is \textit{sufficiently} specified to be possible to realize as the \GlossaryHyperRef{artifact}{artifact} it represents.

	\GlossaryNote{Note 1}{
		Two examples of artifacts that could be produced from a concrete model are concrete \GlossaryHyperRef{protocol}{protocols}, the \GlossaryHyperRef{message}{messages} of which can be practically \GlossaryHyperRef{code}{coded}, and \GlossaryHyperRef{implementation-software}{software implementations}.
	}

\GlossaryEntry{model-reference}{Model, Reference}
An \GlossaryHyperRef{model-abstract}{abstract model} defining technical concepts of fundamental importance to a specific problem domain.
See also Section \ref{sec:introduction:scope}.

	\GlossaryNote{RAMI4.0}{
		defines reference model as a ``model that is generally used and recognized as being suitable (has recommendation character) for deriving specific models''.
		We understand their use of the word ``specific'' to be equivalent to how we use ``concrete''.
		Even though our definition clarifies that the model in question must be abstract, it should be interpreted as being equivalent.
	}

	\GlossaryNote{SOA-RM}{
		defines reference model as ``an abstract framework for understanding significant relationships among the entities of some environment that enables the development of specific architectures using consistent standards or specifications supporting that environment''.
		It further clarifies that a ``reference model consists of a minimal set of unifying concepts, axioms and relationships within a particular problem domain, and is independent of specific standards, technologies, implementations, or other concrete details''.
		Our definition should be interpreted as being equivalent.
	}

\GlossaryEntry{operator}{Operator}
A person or \GlossaryHyperRef{entity}{entity} tasked with the \GlossaryHyperRef{management-system}{management} of a set of \GlossaryHyperRef{system}{systems}.

\GlossaryEntry{organization}{Organization}
A \GlossaryHyperRef{stakeholder}{stakeholder} comprised of an organized body of other stakeholders.

\GlossaryEntry{policy}{Policy}
A set of obligations, \GlossaryHyperRef{constraint}{constraints} and/or other conditions under which some activity is permitted.

	\GlossaryNote{SOA-RM}{
		defines policy as ``a statement of obligations, constraints or other conditions of use of an owned entity as defined by a participant''.
		Our definition should be interpreted as being equivalent.
	}

\GlossaryEntry{policy-service-consumption}{Policy, Service Consumption}
The \GlossaryHyperRef{policy}{policy} under which a \GlossaryHyperRef{provider-service}{provided service} is permitted to be \GlossaryHyperRef{consumer-service}{consumed}.

\GlossaryEntry{profile}{Profile}
A set of \GlossaryHyperRef{constraint}{constraints} superimposed on a \GlossaryHyperRef{protocol}{protocol}.

	\GlossaryNote{Note 1}{
		A profile \textit{never} introduces more \GlossaryHyperRef{message}{messages} to a protocol.
		It adds constraints to the existing messages of a protocol.
	}

	\GlossaryNote{Note 2}{
		A profile could, for example, introduce an authentication mechanism to a protocol by requiring that a certain type of token be included in each message.
		It could demand that a certain protocol be extended, or that a particular kind of \GlossaryHyperRef{codec}{codec} be used for message bodies, and so on.
	}

\GlossaryEntry{property}{Property}
A name/value pair of \GlossaryHyperRef{data}{data}, associated with either an \GlossaryHyperRef{entity}{entity} or a \GlossaryHyperRef{relationship}{relationship}.

	\GlossaryNote{Note 1}{
		A property is a form of \GlossaryHyperRef{metadata}{metadata}.
	}

\GlossaryEntry{protocol}{Protocol}
A \GlossaryHyperRef{model}{model} of communication defined in terms of \GlossaryHyperRef{state}{states} and \GlossaryHyperRef{message}{messages}.
See also \GlossaryNameRef{protocol-extensible}.

	\GlossaryNote{Note 1}{
		The states, if any, dictate the outcomes of sending certain messages.
		For example, let us assume that some state can be either \texttt{BUSY} or \texttt{READY}.
		If the former state would be the active when a certain message is received, the designated response could be an error message.
		If, however, the \texttt{READY} state would have been active, the state could be transitioned to the \texttt{BUSY} value and a success response be provided to the sender.
	}

\GlossaryEntry{protocol-extensible}{Protocol, Extensible}
A \GlossaryHyperRef{protocol}{protocol} allowing for \GlossaryHyperRef{subprotocol}{subprotocols} to be formulated in terms of its \GlossaryHyperRef{message}{messages}.
See also \GlossaryNameRef{stack-protocol}.

	\GlossaryNote{Note 1}{
		Every new message introduced by a subprotocol must be a \GlossaryHyperRef{validation}{valid} message of its \GlossaryHyperRef{superprotocol}{superprotocol}.
	}

	\GlossaryNote{Note 2}{
		Many of the currently prevalent protocols are designed with the intent of being extensible.
		For example, HTTP \cite{fielding2014hypertext} provides provisions for an extending protocol to define its own set of directory operations, to simultaneously support multiple \GlossaryHyperRef{codec}{codecs}, and so on.
	}

	\GlossaryNote{Note 3}{
		As long as a given protocol provides at least one message whose contents can be arbitrary, a subprotocol can be produced.
		This means that even protocols not designed to be extended can, in some context, be meaningfully used to define subprotocols.
	}

\GlossaryEntry{provider-service}{Provider, Service}
A \GlossaryHyperRef{system}{system} that makes \GlossaryHyperRef{service}{services} available for \GlossaryHyperRef{consumer-service}{consumption} by any systems able to satisfy its \GlossaryHyperRef{policy-service-consumption}{consumption policies}.

	\GlossaryNote{SOA-RM}{
		defines a service provider as ``an entity (person or organization) that offers the use of capabilities by means of a service''.
		Our definition is more specific in that it requires the \GlossaryHyperRef{entity}{entity} be a system.
	}

\GlossaryEntry{qos}{QoS}
See \GlossaryNameRef{service-quality-of}.

\GlossaryEntry{relationship}{Relationship}
A uni-directional association of two \GlossaryHyperRef{entity}{entities}, possibly with an associated \GlossaryHyperRef{data}{data} name.

\GlossaryEntry{routing-message}{Routing, Message}
The act of forwarding a \GlossaryHyperRef{message}{message} towards the \GlossaryHyperRef{function}{function} it is meant to \GlossaryHyperRef{invocation-function}{invoke}.

\GlossaryEntry{service}{Service}
A set of \GlossaryHyperRef{function}{functions} that can be \GlossaryHyperRef{provider-service}{provided} via a \GlossaryHyperRef{interface-service}{service interface}.

	\GlossaryNote{Note 1}{
		As each function of a service represents its own \GlossaryHyperRef{protocol}{protocol}, the service itself becomes the union of all those protocols.
		This introduces the need to be able to determine how to forward \GlossaryHyperRef{message}{messages} received by the service to its functions.
		This is accomplished through \GlossaryHyperRef{routing-message}{message routing}.
	}

	\GlossaryNote{RAMI4.0}{
		defines a service as ``separate scope of functions offered by an entity or organization via interfaces''.
		Our definition implicitly restricts service provision to systems, as only \GlossaryHyperRef{system}{systems} can provide services.
	}

	\GlossaryNote{SOA-RM}{
		defines a service as ``the means by which the needs of a consumer are brought together with the capabilities of a provider''.
		Our definition is more specific about how the \GlossaryHyperRef{capability}{capabilities} of a service are made available.
	}

	\GlossaryNote{IoTA:AF}{
		defines a service as ``what [is] used to exchange information from a providing system to a consuming system''.
		It further adds that ``in a service, capabilities are grouped together if they share the same context''.
		The definition presented here should be interpreted as being compatible but more specific about how information is exchanged and capabilities are \GlossaryHyperRef{invocation-function}{invoked}.
	}

\GlossaryEntry{service-quality-of}{Service, Quality of (QoS)}
The degree of performance at which a given \GlossaryHyperRef{service}{service} is \GlossaryHyperRef{provider-service}{provided}. 

\GlossaryEntry{shell-administrative}{Shell, Administrative}
A \GlossaryHyperRef{manifest-system}{manifest} and a \GlossaryHyperRef{interface-system-management}{management interface} associated with a certain \GlossaryHyperRef{system}{system}.

	\GlossaryNote{Note 1}{
		The availability of both a manifest and a management interface allows for both static and dynamic details about a system to be queried, as well as providing for the \GlossaryHyperRef{configuration}{configuration} of the system to be changed.
	}

	\GlossaryNote{RAMI4.0}{
		defines administration shell as a ``virtual digital and active representation of an I[ndustry ]4.0 component in the I[nustry ]4.0 system''.
		It also adds that ``an administration shell contains the manifest and the component manager'' and depicts the shell as wrapping, or being associated with, an \GlossaryHyperRef{asset}{asset}.
		Apart from our referring to \GlossaryHyperRef{component-industry40}{Industry 4.0 components} as \GlossaryHyperRef{system-industry40}{Industry 4.0 systems}, as well as the added qualification that the asset in question be a system, our definitions should be interpreted as being equivalent.
	}

\GlossaryEntry{signature-function}{Signature, Function}
A \GlossaryHyperRef{model}{model} specifying the \GlossaryHyperRef{type-data}{type} of the \GlossaryHyperRef{message}{message} a given \GlossaryHyperRef{function}{function} accepts when \GlossaryHyperRef{invocation-function}{invoked}, as well as the types of any messages it could return in response.

	\GlossaryNote{Note 1}{
		This means that a function signature establishes a \GlossaryHyperRef{protocol}{protocol} for a certain \GlossaryHyperRef{service}{service} function.
	}

\GlossaryEntry{soa}{SOA}
See \GlossaryNameRef{architecture-service-oriented}.

\GlossaryEntry{soc}{SoC}
See \GlossaryNameRef{system-of-clouds}.

\GlossaryEntry{software}{Software}
See \GlossaryNameRef{implementation-software}.

\GlossaryEntry{solc}{SoLC}
See \GlossaryNameRef{system-of-local-clouds}.

\GlossaryEntry{sos}{SoS}
See \GlossaryNameRef{system-of-systems}.

\GlossaryEntry{stack-extensible-protocol}{Stack, Extensible Protocol}
A \GlossaryHyperRef{stack-protocol}{protocol stack} whose topmost \GlossaryHyperRef{protocol}{protocol} is \GlossaryHyperRef{protocol-extensible}{extensible}.

\GlossaryEntry{stack-protocol}{Stack, Protocol}
A stack with an \GlossaryHyperRef{protocol-extensible}{extensible protocol} as base and $n > 0$ \GlossaryHyperRef{subprotocol}{subprotocols} layered on top of it.

	\GlossaryNote{Note 1}{
		Every protocol part of a protocol stack, with the exception of the topmost, must be extensible.
	}

	\GlossaryNote{Note 2}{
		An example of a notable protocol stack is that of HTTP \cite{fielding2014hypertext}.
		It is defined as an extension of the TCP protocol, which in turn extends the IP protocol, which can work as a subprotocol of several other lower-level protocols.
		HTTP is an \GlossaryHyperRef{stack-extensible-protocol}{extensible protocol stack}, which allows for an engineer to define an application-specific protocol on top of its stack.
	}

\GlossaryEntry{stakeholder}{Stakeholder}
A person or group of persons who are involved in or affected by certain \GlossaryHyperRef{entity}{entities} or enterprises.

\GlossaryEntry[(noun)]{state}{State}
One out of all possible sequences of values that could be expressed by the \GlossaryHyperRef{datum}{datums} of some \GlossaryHyperRef{data}{data}.

	\GlossaryNote{Note 1}{
		If the data would consist of a sequence bits, which can only have the values 0 and 1, a state becomes a pattern of zeroes and ones those bits can record.
		Given four bits, possible states could, for example, be \texttt{0010} or \texttt{1001}. 
	}

	\GlossaryNote{Note 2}{
		The term is often used as a wildcard for any kind of storage construct, including bit flags, state machines and graph databases.
	}

\GlossaryEntry{subprotocol}{Subprotocol}
A \GlossaryHyperRef{protocol}{protocol} that is realized as an \GlossaryHyperRef{protocol-extensible}{extension} of another protocol.

\GlossaryEntry{subsystem}{Subsystem}
An \GlossaryHyperRef{entity-systemic}{systemic entity} being a part of than another such. Compare with \GlossaryNameRef{component}.

\GlossaryEntry{superprotocol}{Superprotocol}
A \GlossaryHyperRef{protocol}{protocol} that is \GlossaryHyperRef{protocol-extensible}{extended} by another protocol.

\GlossaryEntry{system}{System}
An \GlossaryHyperRef{entity}{entity} capable of \GlossaryHyperRef{provider-service}{providing services}, \GlossaryHyperRef{consumer-service}{consuming services}, or both.

	\GlossaryNote{Note 1}{
		The word ``system'' is more generally understood to be very inclusive, expressing the larger idea of connected \GlossaryHyperRef{component}{components} facilitating one or more \GlossaryHyperRef{capability}{capabilities}.
		From the perspective of Arrowhead, however, capabilities can only be \GlossaryHyperRef{invocation-function}{invoked} through \GlossaryHyperRef{service}{services}, which means that a system unable to provide or consume services can only be described as a component of another system.
	}

	\GlossaryNote{Note 2}{
		A system is practically distinct from a \GlossaryHyperRef{system-of-systems}{system-of-systems} by being represented only by a single \GlossaryHyperRef{identity}{identity}.
		In contrast, a system-of-systems does either not have its own identity, or has both its own identity and another identity for each of its \GlossaryHyperRef{subsystem}{subsystems}.
	}

	\GlossaryNote{IoTA:AF}{
		defines a system as ``what is providing and/or consuming services''.
		It further adds that ``a system can be the service provider of one or more services and at the same time the service consumer of one or more services''.
		The definition presented here should be interpreted as equivalent.
	}

\GlossaryEntry{system-industry40}{System, Industry 4.0}
A \GlossaryHyperRef{system}{system} with an associated \GlossaryHyperRef{shell-administrative}{administrative shell}.

	\GlossaryNote{Note 1}{
		A system being an Industry 4.0 system is not necessarily the same as it being Industry 4.0 \textit{compliant}.
		Industry 4.0 compliance is outside the scope of this reference model.
	}

	\GlossaryNote{RAMI4.0}{
		introduces the concept of ``Industry 4.0 components'', which ``are globally and uniquely identifiable participants capable of communication, and consist of the administration shell and the asset [...] with a digital connection within an I[ndustry ]4.0 system [...], and offer services there with defined quality of service [...] properties''.
		Given that you have read our definitions for system and \GlossaryHyperRef{component}{component}, you will notice that our understanding of the system aligns more closely with the above quotation than our understanding of the component.
		For this reason, we adopt the RAMI4.0 concept with the modified name, Industry 4.0 \textit{system}.
		Our definition should be interpreted as being equivalent to that of the RAMI4.0 Industry 4.0 component.
	}

\GlossaryEntry{system-of-clouds}{System-of-Clouds (SoC)}
A set of \GlossaryHyperRef{cloud}{clouds} that \GlossaryHyperRef{consumer-service}{consume} each other's \GlossaryHyperRef{service}{services} in order to facilitate a \GlossaryHyperRef{capability}{capability} none of the constituent local clouds could \GlossaryHyperRef{provider-service}{provide} on its own.

\GlossaryEntry{system-of-local-clouds}{System-of-Local-Clouds (SoLC)}
A \GlossaryHyperRef{system-of-clouds}{system-of-clouds} where every \GlossaryHyperRef{cloud}{cloud} is a \GlossaryHyperRef{cloud-local}{local cloud}.

\GlossaryEntry{system-of-systems}{System-of-Systems (SoS)}
A set of \GlossaryHyperRef{system}{systems} that \GlossaryHyperRef{consumer-service}{consume} each other's \GlossaryHyperRef{service}{services} in order to facilitate a \GlossaryHyperRef{capability}{capability} none of the constituent systems could \GlossaryHyperRef{provider-service}{provide} on its own.

	\GlossaryNote{Note 1}{
		The system-of-systems concept is not restricted to the boundaries imposed by \GlossaryHyperRef{organization}{organizations}, \GlossaryHyperRef{cloud}{clouds} and \GlossaryHyperRef{system-of-clouds}{systems-of-clouds}, even though a cloud, for example, could be regarded as a type of system-of-systems.
	}

	\GlossaryNote{IoTA:AF}{
		defines a system-of-systems as ``a set of system, which [...] exchange information by means of services''.
		It further adds that ``when Arrowhead compliant systems collaborate, they become a System of Systems in the Arrowhead Framework's definition''.
		While we clarify here that the desired outcome of collaboration is the facilitation of new capabilities, the definitions should be interpreted as being equivalent.
	}

\GlossaryEntry{type-data}{Type, Data}
A \GlossaryHyperRef{description}{description} of how datums are to be arranged to \GlossaryHyperRef{code}{code} certain facts.
See also \GlossaryNameRef{data}.

	\GlossaryNote{Note 1}{
		While this definition may seem foreign, it does capture how integer types, classes, enumerators and other general data type are used in the context of a programming language or \GlossaryHyperRef{codec}{codec}.
		In the end, all data are bits or other symbols.
		From our perspective, types serve to group those symbols and assign them meaning.
	}

	\GlossaryNote{Note 2}{
		A data type provides only syntactic, or structural, information about data.
		While knowing the data type used to code some data is required for its interpretation, contextual knowledge is also needed.
		For example, a data type may specify a \texttt{name}, but it will not indicate when or why that name is useful.
		That information would have to be provided via documentation or some other means.
	}

\GlossaryEntry{user}{User}
A \GlossaryHyperRef{stakeholder}{stakeholder} associated with the usage of a certain \GlossaryHyperRef{entity}{entity}.

\GlossaryEntry{validation}{Validation}
The process through which it is determined if a \GlossaryHyperRef{model}{model} satisfies a \GlossaryHyperRef{constraint}{constraint}.

}