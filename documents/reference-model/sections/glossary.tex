% Copyright (c) 2021-10-07 Eclipse Arrowhead Project
%
% This program and the accompanying materials are made available under the
% terms of the Eclipse Public License 2.0 which is available at
% http://www.eclipse.org/legal/epl-2.0.
%
% SPDX-License-Identifier: EPL-2.0

This section provides an alphabetically sorted list of all significant terms introduced or named in this document.
Each term consisting of more than one word is sorted by its final, or qualified, word.
This means that the definition of \GlossaryHyperRef{protocol-service}{service protocol}, for example, is found at \GlossaryNameRef{protocol-service}.

Many of the definitions are amended with notes and references to RAMI4.0 \cite{adolphs2016reference}, SOA-RM \cite{mackenzie2006reference} and IoTA:AF \cite{delsing2017iot}, which are always listed after the definition they amend.
Regular notes are numbered, while those making a comment on a definition in RAMI4.0, SOA-RM or IoTA:AF are introduced with the three abbreviations just listed.

{

\newcommand{\GlossaryEntry}[3][]{\subsubsection*{#3\IfStrEq{#1}{}{}{ {\normalfont \textit{#1}}}}\label{sec:glossary:#2}}
\newcommand{\GlossaryNote}[2]{\begin{minipage}[b]{\dimexpr\linewidth-0.5cm\relax}\vspace*{0.33cm}\footnotesize{\textbf{#1}\ #2}\end{minipage}}

\GlossaryEntry{abstract}{Abstract}
See \GlossaryNameRef{model-abstract}.

\GlossaryEntry{architecture}{Architecture}
A \GlossaryHyperRef{model-concrete}{concrete model} of a \GlossaryHyperRef{system-of-systems}{system-of-systems} defined in terms of certain \GlossaryHyperRef{model-reference}{reference models}, \GlossaryHyperRef{architecture-reference}{reference architectures} and other concrete architectures.
See Section \ref{sec:introduction:scope}.

	\GlossaryNote{RAMI4.0}{
	    defines architecture as the ``combination of elements of a model based on principles and rules for constructing, refining and using it''.
		We consider ``combinations of elements of a model'' to be a ``model of a system-of-systems'' and to be ``based on principles and rules for constructing, refining and using it'' as building upon reference models and architectures.
		Our definition should be interpreted as being compatible but more specific.
	}

	\GlossaryNote{SOA-RM}{
		defines software architecture as ``the structure or structures of an information system consisting of entities and their externally visible properties, and the relationships among them''.
		That definition is equivalent to our definition of \GlossaryHyperRef{model}{model}, with the exception that the thing being modeled has to be an information system.
		As our definition is concerned with a model and a system-of-systems, which must be an information system, we regard out definition as compatible but more specific.
	}

\GlossaryEntry{architecture-reference}{Architecture, Reference}
A significantly useful \GlossaryHyperRef{model-abstract}{abstract model} of a \GlossaryHyperRef{system-of-systems}{system-of-systems} defined in terms of certain \GlossaryHyperRef{model-reference}{reference models} and other reference architectures.
See Section \ref{sec:introduction:scope}.

	\GlossaryNote{RAMI4.0}{
		defines reference architecture as a ``model for an architecture description (for I[ndustry ]4.0) which is generally used and recognized as being suitable (has reference character)''.
		We consider a ``model for an architecture description'' to be an ``abstract model of a system-of-systems''.
		Our definition should be interpreted as being compatible but more specific.
	}

	\GlossaryNote{SOA-RM}{
	    defines reference architecture as ``an architectural design pattern that indicates how an abstract set of mechanisms and relationships realizes a predetermined set of requirements''.
	    While we let the part about requirements be implicit, our definition should be interpreted as being compatible but more specific.
	}

\GlossaryEntry{architecture-service-oriented}{Architecture, Service-Oriented (SOA)}
An \GlossaryHyperRef{architecture}{architecture} concerned with \GlossaryHyperRef{service}{service} \GlossaryHyperRef{provider-service}{provision} and \GlossaryHyperRef{consumer-service}{consumption}.

	\GlossaryNote{Note 1}{
		Any architecture building upon the \GlossaryHyperRef{model-reference}{reference model} of this document will become service-oriented.
		See also Section \ref{sec:introduction}.
	}

\GlossaryEntry{arrowhead}{Arrowhead}
See \GlossaryNameRef{framework-arrowhead}.

\GlossaryEntry{artifact}{Artifact}
A thing or object, tangible or intangible.

\GlossaryEntry{asset}{Asset}
Synonymous to \GlossaryNameRef{resource}.

	\GlossaryNote{RAMI4.0}{
		defines asset as an ``object which has a value for an organization''.
		See \GlossaryNameRef{resource} for a comparable term.
	}

\GlossaryEntry{boundary}{Boundary}
A point or border where either two or more \GlossaryHyperRef{artifact}{artifacts} meet or one artifact ends.

\GlossaryEntry{boundary-cloud}{Boundary, Cloud}
A \GlossaryHyperRef{boundary}{boundary} separating the \GlossaryHyperRef{artifact}{artifacts} belonging to a \GlossaryHyperRef{cloud}{cloud} from those not belonging to it.

	\GlossaryNote{Note 1}{
		A cloud boundary can be \GlossaryHyperRef{boundary-local}{local} or \GlossaryHyperRef{boundary-virtual}{virtual}, depending on if the boundary is formed by physical or virtual \GlossaryHyperRef{property}{properties}.
	} 

\GlossaryEntry{boundary-local}{Boundary, Local}
A \GlossaryHyperRef{boundary}{boundary} that exists in the physical world.

	\GlossaryNote{Note 1}{
		Local boundaries can be facilitated by walls, locations of operation, attachment to certain vehicles or power sources, and so on.
	}

\GlossaryEntry{boundary-virtual}{Boundary, Virtual}
A \GlossaryHyperRef{boundary}{boundary} that exists only virtually.

	\GlossaryNote{Note 1}{
		Virtual boundaries can be facilitated by cryptographic secrets, identifiers, ownership statements, contracts, and so on.
	}

\GlossaryEntry{capability}{Capability}
A task, of any nature, that can be performed by a \GlossaryHyperRef{device}{device}.

	\GlossaryNote{Note 1}{
		The term must be understood in the most general sense possible.
		It includes the abilities of hosting \GlossaryHyperRef{system}{systems}, reading from sensors, triggering actuators, among many other possible examples.
	}

	\GlossaryNote{SOA-RM}{
		defines a capability as ``a real-world effect that a service provider is able to provide to a service consumer''.
		To leave room for devices to be described as doing other things that \GlossaryHyperRef{provider-service}{providing} or \GlossaryHyperRef{consumer-service}{consuming} services, we made our definition more general.
		See also \GlossaryNameRef{capability-system}.
	}

\GlossaryEntry{capability-system}{Capability, System}
A \GlossaryHyperRef{capability}{capability} a \GlossaryHyperRef{system}{system} can trigger via its hosting \GlossaryHyperRef{device}{device}.

\GlossaryEntry{cloud}{Cloud}
A \GlossaryHyperRef{boundary}{bounded} \GlossaryHyperRef{system-of-systems}{system-of-systems} able independently execute given tasks through the use of a pool of \GlossaryHyperRef{resource}{resources}.

	\GlossaryNote{Note 1}{
		When the term ``cloud'' is used elsewhere, it often refers to clouds with only virtual resources, such as compute, storage and software-defined network utilities.
		Here, we refer to such clouds as \GlossaryHyperRef{cloud-virtual}{virtual clouds}.
		By making the unqualified word ``cloud'' less specific, it becomes more clear how our \GlossaryHyperRef{cloud-local}{local cloud} concept shares similarities with other types of clouds.
	}

\GlossaryEntry{cloud-local}{Cloud, Local}
A \GlossaryHyperRef{cloud}{cloud} \GlossaryHyperRef{boundary-local}{bound to a physical location} due to its acting on or producing \GlossaryHyperRef{resource-local}{local resources}.
See Section \ref{sec:reference-model:system-of-systems:local-cloud}.

	\GlossaryNote{IoTA:AF}{
		provides an introduction to the local cloud concept in its second chapter, as well as an architectural definition in its third chapter.
		The following is an excerpt from the introduction:
		\begin{quote}
		The local cloud concept takes the view that specific geographically local automation tasks should be encapsulated and protected.
		These tasks have strong requirements on real time, ease of engineering, operation and maintenance, and system security and safety.
		The local cloud idea is to let the local cloud include the devices and systems required to perform the desired automation tasks, thus providing a local ``room'' which can be protected from outside activities.
		In other words, the cloud will provide a boundary to the open internet, thus aiming to protect the internal of the local cloud from the open internet.
		\end{quote}
		The third chapter contains the following:
		\begin{quote}
		In the Arrowhead Framework context a local cloud is defined as a self-contained network with the three mandatory core systems deployed and at least one application system deployed [...]
		\end{quote}
		Both of these descriptions are practical, in the sense that they emphasize engineering aspects.
		As this document is a reference model, engineering aspects are out of scope.
		The more general terms ``geographically local'', ``room'' and ``boundary'' clearly highlight the physicality of the local cloud itself, while the depiction of ``devices'' performing ``automation tasks'' makes it apparent that some kind of physical activity is involved, such as manufacturing.
		Finally, the local cloud being ``encapsulated'', ``protected'' and ``self-contained'' indicates that it is understood to exhibit a degree of independence with respect to the tasks it is given, which we expect all kinds of clouds to exhibit.
		Our definition should be interpreted as a summation of these characteristics.
	}

\GlossaryEntry{cloud-local-automation}{Cloud, Local Automation}
See \GlossaryNameRef{cloud-local}.

\GlossaryEntry{cloud-virtual}{Cloud, Virtual}
A \GlossaryHyperRef{cloud}{cloud} \GlossaryHyperRef{boundary-virtual}{unbound by physical location} by only acting on or producing \GlossaryHyperRef{resource-virtual}{virtual resources}.

\GlossaryEntry[(verb)]{code}{Code}
Transforming \GlossaryHyperRef{data}{data} from being expressed in one \GlossaryHyperRef{encoding}{encoding} into another.
See also \GlossaryHyperRef{decode}{decode} and \GlossaryHyperRef{encode}{encode}.

\GlossaryEntry[(noun)]{coding}{Coding}
Synonymous to \GlossaryNameRef{encoding}.

\GlossaryEntry{component}{Component}
An \GlossaryHyperRef{entity}{entity} that can be part of a \GlossaryHyperRef{device}{device} or \GlossaryHyperRef{system}{system} and contribute to it facilitating its \GlossaryHyperRef{capability}{capabilities}.

	\GlossaryNote{Note 1}{
		The word ``component'' should only be used to refer to the constituents of devices and systems.
		It should never be used to refer to a system being a constituent of a \GlossaryHyperRef{system-of-systems}{system-of-systems}.
		Such a system should be referred to as being a \GlossaryHyperRef{subsystem}{subsystem}.
	}

	\GlossaryNote{RAMI4.0}{
		makes no practical distinction between components and systems, as is done here.
		See \GlossaryNameRef{system} for more details.
	}

\GlossaryEntry{component-hardware}{Component, Hardware}
A physical \GlossaryHyperRef{component}{component} that can only be part of a \GlossaryHyperRef{device}{device}.
See Section \ref{sec:reference-model:device}.

\GlossaryEntry{component-software}{Component, Software}
A virtual \GlossaryHyperRef{component}{component} that can only be part of a \GlossaryHyperRef{system}{system}.
See Section \ref{sec:reference-model:system}.

\GlossaryEntry{communication}{Communication}
The activity of sending and/or receiving \GlossaryHyperRef{message}{messages}.

\GlossaryEntry{communication-service-oriented}{Communication, Service-Oriented}
\GlossaryHyperRef{communication}{Communication} \GlossaryHyperRef{description}{described} in terms of the \GlossaryHyperRef{provider-service}{provision} and \GlossaryHyperRef{consumer-service}{consumption} of \GlossaryHyperRef{service}{services}.

\GlossaryEntry{concrete}{Concrete}
See \GlossaryNameRef{model-concrete}.

\GlossaryEntry{concretization}{Concretization}
Making an \GlossaryHyperRef{model-abstract}{abstract model} less abstract by specifying some or all details required to realize it.

\GlossaryEntry{configuration}{Configuration}
A set of changeable \GlossaryHyperRef{property}{properties} that directly influence how a \GlossaryHyperRef{system}{system} exercises its \GlossaryHyperRef{capability-system}{capabilities}.

\GlossaryEntry{configure}{Configure}
To update a \GlossaryHyperRef{configuration}{configuration}.

\GlossaryEntry{connection}{Connection}
An active medium through which attached \GlossaryHyperRef{interface}{interfaces} can \GlossaryHyperRef{communication}{communicate}.

\GlossaryEntry{constraint}{Constraint}
A \GlossaryHyperRef{property}{property} that imposes constraints, or limits, on an \GlossaryHyperRef{entity}{entity} or \GlossaryHyperRef{relationship}{relationship}.

	\GlossaryNote{Note 1}{
		The presence of constraints enable \GlossaryHyperRef{validation}{validation}.
	}

	\GlossaryNote{Note 2}{
		Perhaps a bit counterintuitively, a constraint \textit{adds} information to its target by reducing the ways in which it could be realized.
	}

\GlossaryEntry{consumer-service}{Consumer, Service}
A \GlossaryHyperRef{system}{system} currently \GlossaryHyperRef{invocation-function}{invoking} a \GlossaryHyperRef{function-service}{function} \GlossaryHyperRef{provider-service}{provided} via a \GlossaryHyperRef{service}{service}.

	\GlossaryNote{Note 1}{
		If used to refer to a \GlossaryHyperRef{stakeholder}{stakeholder}, the term must be interpreted as if that stakeholder consumes services via systems.
	}

	\GlossaryNote{SOA-RM}{
		defines a service consumer as ``an entity which seeks to satisfy a particular need through the use [of] capabilities offered by means of a service''.
		We require that the one consuming the service is (1) a \GlossaryHyperRef{system}{system} rather than just any \GlossaryHyperRef{entity}{entity}, as well as (2) that the \GlossaryHyperRef{capability-system}{capabilities} of the consumed service be exercised by invoking a function.
	}

\GlossaryEntry{data}{Data}
A sequence of \GlossaryHyperRef{datum}{datums} recording a set of \GlossaryHyperRef{description}{descriptions} via the structure superimposed by a \GlossaryHyperRef{type-data}{data type}.

	\GlossaryNote{Note 1}{
		Let us assume that some data is going to be sent to a drilling machine.
		The type associated with the data requires that it always consists of 8 bits, organized such that the first 4 bits indicate the speed of drilling in multiples of 100 rotations per minute, while the latter 4 determine how much to lower the drill in multiples of 5 millimeters.
		A \GlossaryHyperRef{state}{state} that could be expressed with those 8 bits is \texttt{0100 1101}.
		If each of the two sequences of 4 bits is treated as a big-endian integer with base 2, they record $4$ and $13$ in decimal notation.
		This would indicate that the drill should spin at $4 * 100 = 400$ rotations per minute and be lowered $13 * 5 = 65$ millimeters.
	}

	\GlossaryNote{Note 2}{
		Without knowledge of the types and context associated with some data, that data cannot be interpreted.
	}

\GlossaryEntry{datum}{Datum}
A variable expressing one out of a set of possible values.
See also \GlossaryNameRef{state}.

	\GlossaryNote{Note 1}{
		A familiar example of a datum may be the bit, or binary digit.
		Its possible set of symbols is $\{0, 1\}$.
	}

\GlossaryEntry{decode}{Decode}
The act of transforming \GlossaryHyperRef{data}{data} from being expressed in a \GlossaryHyperRef{encoding}{encoding} suitable for transmission or storage to another encoding suitable for interpretation.

	\GlossaryNote{Note 1}{
		Decoding is the reverse of \GlossaryHyperRef{encode}{encoding}.
	}

	\GlossaryNote{Note 2}{
		The term can also be used to express the act of a human interpreting data.
	}

\GlossaryEntry{description}{Description}
Facts about an \GlossaryHyperRef{entity}{entity} or \GlossaryHyperRef{entity-class-of}{class of entities}, expressed in the form of a \GlossaryHyperRef{model}{model}, a text, or both.

\GlossaryEntry[(noun)]{design}{Design}
Every document, \GlossaryHyperRef{model}{model} and other record \GlossaryHyperRef{description}{describing} how a certain \GlossaryHyperRef{artifact}{artifact} can be \GlossaryHyperRef{implementation}{implemented}.

\GlossaryEntry[(verb)]{design-verb}{Design}
The activity of producing \GlossaryHyperRef{design}{designs}.

\GlossaryEntry{designer}{Designer}
A \GlossaryHyperRef{stakeholder}{stakeholder} involved in the \GlossaryHyperRef{design-verb}{design} of \GlossaryHyperRef{artifact}{artifacts}.
See Section \ref{sec:reference-model:stakeholder}.

\GlossaryEntry{developer}{Developer}
A \GlossaryHyperRef{stakeholder}{stakeholder} developing the \GlossaryHyperRef{component}{components} that make up \GlossaryHyperRef{device}{devices} and/or \GlossaryHyperRef{system}{systems}.
See Section \ref{sec:reference-model:stakeholder}.

\GlossaryEntry{device}{Device}
A physical \GlossaryHyperRef{entity}{entity} made from \GlossaryHyperRef{component-hardware}{hardware components} with the significant \GlossaryHyperRef{capability}{capability} of being able to host \GlossaryHyperRef{system}{systems}.
See Section \ref{sec:reference-model:device}.

	\GlossaryNote{IoTA:AF}{
		defines device as ``a piece of equipment, machine, hardware, etc. with computational, memory and communication capabilities which hosts one or several Arrowhead Framework systems and can be bootstrapped in an Arrowhead local cloud''.
		The definition provided here should be interpreted as being equivalent.
	}

\GlossaryEntry{device-connected}{Device, Connected}
A \GlossaryHyperRef{device}{device} that is physically attached to at least one other device via their \GlossaryHyperRef{interface-device}{interfaces}, enabling them to \GlossaryHyperRef{communication}{communicate}.

\GlossaryEntry{device-end}{Device, End}
A \GlossaryHyperRef{device-connected}{connected device} being the intended recipient of a \GlossaryHyperRef{message}{message}.

\GlossaryEntry{device-human-interface}{Device, Human Interface (HID)}
A \GlossaryHyperRef{device}{device} that provides sensors and actuators that together make up an \GlossaryHyperRef{interface}{interface} through which a human can exchange messages with one or more \GlossaryHyperRef{system}{systems}.

\GlossaryEntry{device-intermediary}{Device, Intermediary}
A \GlossaryHyperRef{device-connected}{connected device} that receives and forwards \GlossaryHyperRef{message}{messages} toward \GlossaryHyperRef{device-end}{end devices}.

\GlossaryEntry{encode}{Encode}
The act of transforming \GlossaryHyperRef{data}{data} from being expressed in a \GlossaryHyperRef{encoding}{encoding} suitable for interpretation to another encoding suitable for transmission or storage.

	\GlossaryNote{Note 1}{
		Encoding is the reverse of \GlossaryHyperRef{decode}{decoding}.
	}

	\GlossaryNote{Note 2}{
		The term can also be used to express the act of a human recording data.
	}

\GlossaryEntry[(noun)]{encoding}{Encoding}
A \GlossaryHyperRef{model-concrete}{concrete} \GlossaryHyperRef{type-data}{data type} used to structure \GlossaryHyperRef{data}{data} for transmission, storage and/or interpretation.

\GlossaryEntry{entity}{Entity}
An \GlossaryHyperRef{artifact}{artifact} with an \GlossaryHyperRef{identity}{identity}, allowing for it to be distinguished from all other artifacts.
See Section \ref{sec:reference-model:entity}.

	\GlossaryNote{Note 1}{
		An entity being uniquely identifiable does not necessarily mean that it is associated with a certificate or \GlossaryHyperRef{identifier}{identifier}.
		It only means that a \GlossaryHyperRef{description}{description} can be rendered that unambiguously refers to the entity in question.
	}

	\GlossaryNote{RAMI4.0}{
		defines entity as an ``uniquely identifiable object which is administered in the information world due to its importance''.
		Our definition should be interpreted as being equivalent.
	}

	\GlossaryNote{SOA-RM}{
	    mentions the word ``entity'' nine times, but provides no explicit definition.
	    We assume their definition to match that of a regular English dictionary, such as ``something that has separate and distinct existence and objective or conceptual reality'' \cite{webster2021entity}.
		Our definition should be interpreted as being equivalent.
	}

\GlossaryEntry{entity-class-of}{Entity, Class of}
A set of \GlossaryHyperRef{entity}{entities} that share a common \GlossaryHyperRef{property}{property}.

\GlossaryEntry{framework}{Framework}
A set of assumptions, concepts, values and practices that frame a certain problem domain.

	\GlossaryNote{SOA-RM}{
		defines framework as ``a set of assumptions, concepts, values, and practices that constitutes a way of viewing the current environment''.
		Our definition should be interpreted as being equivalent.
	}

\GlossaryEntry{framework-arrowhead}{Framework, Arrowhead}
Either of the \GlossaryHyperRef{framework}{framework of ideas} and the \GlossaryHyperRef{framework-software}{framework of software} maintained by the Arrowhead project.

\GlossaryEntry{framework-software}{Framework, Software}
A set of software specifications, \GlossaryHyperRef{implementation-software}{implementations} and other \GlossaryHyperRef{artifact}{artifacts} meant to help address the problem domain of a certain \GlossaryHyperRef{framework}{framework}.

\GlossaryEntry{function}{Function}
Typically synonymous to \GlossaryNameRef{function-service}.
See Section \ref{sec:reference-model:service}.

	\GlossaryNote{Note 1}{
		The term may be used to refer to both \GlossaryHyperRef{function-program}{program} and \GlossaryHyperRef{function-service}{service} functions, if the context makes it clear that both kinds of functions are relevant.
	}

\GlossaryEntry{function-program}{Function, Program}
Synonymous to \GlossaryNameRef{procedure-software}.

\GlossaryEntry{function-service}{Function, Service}
A \GlossaryHyperRef{procedure-software}{software procedure} that handles \GlossaryHyperRef{message}{messages} it receives from a \GlossaryHyperRef{interface-service}{service interface}.
See Section \ref{sec:reference-model:service}.

\GlossaryEntry{hid}{HID}
See \GlossaryNameRef{device-human-interface}.

\GlossaryEntry{industry40}{Industry 4.0}
The fourth industrial paradigm, primarily characterized by high degrees of computerization, digitization and interconnectivity.
See also \cite{adolphs2016reference}.

\GlossaryEntry{identification}{Identification}
The process through which an \GlossaryHyperRef{entity}{entity} verifies the \GlossaryHyperRef{identity}{identity} of another entity. 

\GlossaryEntry{identifier}{Identifier}
\GlossaryHyperRef{data}{Data} associated with an \GlossaryHyperRef{entity}{entity} that allows for it to be \GlossaryHyperRef{identification}{identified}.

\GlossaryEntry{identity}{Identity}
The aspect or aspects, such as \GlossaryHyperRef{identifier}{identifiers}, that makes an \GlossaryHyperRef{entity}{entity} distinct from all other entities.

\GlossaryEntry{implementation}{Implementation}
The realization of a \GlossaryHyperRef{design}{design} as a set of \GlossaryHyperRef{artifact}{artifacts}.

\GlossaryEntry{implementation-software}{Implementation, Software}
An \GlossaryHyperRef{implementation}{implementation} comprised of executable \GlossaryHyperRef{software}{software} \GlossaryHyperRef{artifact}{artifacts}.

\GlossaryEntry{instance-software}{Instance, Software}
A \GlossaryHyperRef{software}{software artifact} currently being executed by a \GlossaryHyperRef{unit-compute}{compute unit}.

\GlossaryEntry{interconnection}{Interconnection}
A \GlossaryHyperRef{connection}{connection} that passes through one or more \GlossaryHyperRef{device-intermediary}{intermediary devices}.

\GlossaryEntry{interface}{Interface}
A \GlossaryHyperRef{boundary}{boundary} where \GlossaryHyperRef{message}{messages} of certain \GlossaryHyperRef{protocol}{protocols} can pass between a \GlossaryHyperRef{connection}{connection} and an \GlossaryHyperRef{entity}{entity}, between two entities, or between an entity and a person.
See Section \ref{sec:reference-model:interface}.

\GlossaryEntry{interface-device}{Interface, Device}
An \GlossaryHyperRef{interface}{interface} through which a \GlossaryHyperRef{device}{device} may send and/or receive \GlossaryHyperRef{message}{messages} to/from a \GlossaryHyperRef{connection}{connection}.

\GlossaryEntry{interface-human}{Interface, Human}
An \GlossaryHyperRef{interface}{interface} through which a person may send and/or receive \GlossaryHyperRef{message}{messages} to/from an \GlossaryHyperRef{entity}{entity}.

\GlossaryEntry{interface-network}{interface, Network}
An \GlossaryHyperRef{interface}{interface} of an \GlossaryHyperRef{device-intermediary}{intermediary device}, primarily intended to be used for passing on \GlossaryHyperRef{message}{messages} toward their intended \GlossaryHyperRef{device-end}{end devices}.

\GlossaryEntry{interface-service}{Interface, Service}
An \GlossaryHyperRef{interface}{interface} through which a certain \GlossaryHyperRef{service}{service} can be \GlossaryHyperRef{consumer-service}{consumed}.

	\GlossaryNote{Note 1}{
		Consuming a service requires that \GlossaryHyperRef{message}{messages} be passed from its \GlossaryHyperRef{device}{device} to its \GlossaryHyperRef{system}{system}, and then from its system to the service itself.
		As the \GlossaryHyperRef{software}{software} making up the service is owned by the system, it is the system that is understood to produce any responses.
		Those are passed on via its device.
	}

	\GlossaryNote{SOA-RM}{
		defines service interface as ``the means by which the underlying capabilities of a service are accessed''.
		Our definition should be interpreted as being equivalent.
	}

\GlossaryEntry{interface-system}{Interface, System}
An \GlossaryHyperRef{interface}{interface} through which a \GlossaryHyperRef{system}{system} may send and/or receive \GlossaryHyperRef{message}{messages} to/from its hosting \GlossaryHyperRef{device}{device}.

	\GlossaryNote{Note 1}{
		The device may either send or respond to messages by its own accord, or pass on messages it receives to and/or from any of its \GlossaryHyperRef{interface-device}{device interfaces}.
	}

\GlossaryEntry{invocation-function}{Invocation, Function}
The attempt to exercise the \GlossaryHyperRef{capability-system}{capabilities} of a \GlossaryHyperRef{system}{system} by sending a \GlossaryHyperRef{message}{message} to one of its \GlossaryHyperRef{function}{functions}.

\GlossaryEntry[(noun)]{message}{Message}
\GlossaryHyperRef{data}{Data} sent or received via a \GlossaryHyperRef{interface-service}{service interface}.

	\GlossaryNote{Note 1}{
		In the context of Arrowhead, messages are only sent to \GlossaryHyperRef{invocation-function}{invoke} the \GlossaryHyperRef{function}{functions} of \GlossaryHyperRef{service}{services} \GlossaryHyperRef{provider-service}{provided} by \GlossaryHyperRef{system}{systems}.
	}

\GlossaryEntry{metadata}{Metadata}
\GlossaryHyperRef{data}{Data} \GlossaryHyperRef{description}{describing} other data.

\GlossaryEntry{model}{Model}
A representation of facts in the form of a graph, consisting of \GlossaryHyperRef{entity}{entities}, \GlossaryHyperRef{relationship}{relationships} and \GlossaryHyperRef{property}{properties}.

	\GlossaryNote{Note 1}{
		Models can be expressed or recorded in many ways, including as visual diagrams, spoken words, text and binary data.
	}

	\GlossaryNote{Note 2}{
		Models can be human-readable, machine-readable, or both.
	}

\GlossaryEntry{model-abstract}{Model, Abstract}
A \GlossaryHyperRef{model}{model} that is \textit{insufficiently} specified to be possible to realize as the \GlossaryHyperRef{artifact}{artifact} it represents.

	\GlossaryNote{Note 1}{
		Abstract models can be referred to by other models, serving as a form of \GlossaryHyperRef{constraint}{constraint}.
		They are commonly used to enforce a degree of uniformity across multiple other models.
	}

\GlossaryEntry{model-concrete}{Model, Concrete}
A \GlossaryHyperRef{model}{model} that is \textit{sufficiently} specified to be possible to realize as the \GlossaryHyperRef{artifact}{artifact} it represents.

	\GlossaryNote{Note 1}{
		Two examples of artifacts that could be produced from a concrete model are concrete \GlossaryHyperRef{protocol}{protocols}, the \GlossaryHyperRef{message}{messages} of which can be practically \GlossaryHyperRef{code}{coded}, and \GlossaryHyperRef{implementation-software}{software implementations}.
	}

\GlossaryEntry{model-information}{Model, Information}
A \GlossaryHyperRef{model}{model} consisting of related \GlossaryHyperRef{type-data}{data types}, \GlossaryHyperRef{message}{messages} and/or other information \GlossaryHyperRef{artifact}{artifacts}.

	\GlossaryNote{Note 1}{
		For example, all data types used by a certain \GlossaryHyperRef{service}{service} make up the information model of that service.
		In other words, the concept represents a pool of information artifacts that can are useful to consider as belonging to the same group.
	}

\GlossaryEntry{model-reference}{Model, Reference}
An \GlossaryHyperRef{model-abstract}{abstract model} defining technical concepts of fundamental importance to a specific problem domain.
See also Section \ref{sec:introduction:scope}.

	\GlossaryNote{RAMI4.0}{
		defines reference model as a ``model that is generally used and recognized as being suitable (has recommendation character) for deriving specific models''.
		We understand their use of the word ``specific'' to be equivalent to how we use ``concrete''.
		Even though our definition clarifies that the model in question must be abstract, it should be interpreted as being equivalent.
	}

	\GlossaryNote{SOA-RM}{
		defines reference model as ``an abstract framework for understanding significant relationships among the entities of some environment that enables the development of specific architectures using consistent standards or specifications supporting that environment''.
		It further clarifies that a ``reference model consists of a minimal set of unifying concepts, axioms and relationships within a particular problem domain, and is independent of specific standards, technologies, implementations, or other concrete details''.
		Our definition should be interpreted as being equivalent.
	}

\GlossaryEntry{network}{Network}
A set of two or more \GlossaryHyperRef{device-end}{end devices}, \GlossaryHyperRef{connection}{connected} in such a manner that any \GlossaryHyperRef{system}{systems} they host are able to \GlossaryHyperRef{communication}{communicate}.
See Section \ref{sec:reference-model:network}.

\GlossaryEntry{operator}{Operator}
A \GlossaryHyperRef{stakeholder}{stakeholder} responsible for the \GlossaryHyperRef{configure}{configuration} and oversight of \GlossaryHyperRef{system}{systems} and the \GlossaryHyperRef{resource}{resources} those systems manage.
See Section \ref{sec:reference-model:stakeholder}.

\GlossaryEntry{owner}{Owner}
A \GlossaryHyperRef{stakeholder}{stakeholder} that owns significant \GlossaryHyperRef{resource}{resources} and/or other \GlossaryHyperRef{artifact}{artifacts}.
See Section \ref{sec:reference-model:stakeholder}.

\GlossaryEntry{organization}{Organization}
A \GlossaryHyperRef{stakeholder}{stakeholder} comprised of an organized body of other stakeholders and/or other persons.

\GlossaryEntry{policy}{Policy}
A set of \GlossaryHyperRef{constraint}{constraints}, of any nature, that must be satisfied for a certain activity to be permitted.
See Section \ref{sec:reference-model:policy}.

	\GlossaryNote{SOA-RM}{
		defines policy as ``a statement of obligations, constraints or other conditions of use of an owned entity as defined by a participant''.
		Our definition should be interpreted as being equivalent.
	}

\GlossaryEntry{policy-function}{Policy, Function}
A \GlossaryHyperRef{policy}{policy} that must be satisfied to be permitted to \GlossaryHyperRef{invocation-function}{invoke} a certain \GlossaryHyperRef{function}{function}.

\GlossaryEntry{policy-service}{Policy, Service}
A \GlossaryHyperRef{policy}{policy} that must be satisfied to be permitted to \GlossaryHyperRef{invocation-function}{invoke} any \GlossaryHyperRef{function-service}{function} part of a certain \GlossaryHyperRef{service}{service}.

\GlossaryEntry{procedure}{Procedure}
See \GlossaryNameRef{procedure-software}.

\GlossaryEntry{procedure-software}{Procedure, Software}
A segment of instructions, part of a \GlossaryHyperRef{software}{software artifact}, that perform some activity if executed.

\GlossaryEntry{profile}{Profile}
See \GlossaryNameRef{profile-protocol}.

\GlossaryEntry{profile-protocol}{Profile, Protocol}
A set of \GlossaryHyperRef{constraint}{constraints} superimposed on a \GlossaryHyperRef{protocol}{protocol}.

	\GlossaryNote{Note 1}{
		A profile \textit{never} introduces more \GlossaryHyperRef{message}{messages} to a protocol.
		It adds constraints to the existing messages of a protocol.
	}

	\GlossaryNote{Note 2}{
		A profile could, for example, introduce an authentication mechanism to a protocol by requiring that a certain type of token be included in each message.
		It could demand that a certain protocol be extended, or that a particular kind of \GlossaryHyperRef{encoding}{encoding} be used for message bodies, and so on.
	}

\GlossaryEntry{property}{Property}
A name/value pair of \GlossaryHyperRef{data}{data}, associated with either an \GlossaryHyperRef{entity}{entity} or a \GlossaryHyperRef{relationship}{relationship}.

	\GlossaryNote{Note 1}{
		A property is a form of \GlossaryHyperRef{metadata}{metadata}.
	}

\GlossaryEntry{protocol}{Protocol}
A \GlossaryHyperRef{model}{model} of communication defined in terms of \GlossaryHyperRef{state}{states} and \GlossaryHyperRef{message}{messages}.
See Section \ref{sec:reference-model:protocol}.

	\GlossaryNote{Note 1}{
		The states, if any, dictate the outcomes of sending certain messages.
		For example, let us assume that some state can be either \texttt{BUSY} or \texttt{READY}.
		If the former state would be the active when a certain message is received, the designated response could be an error message.
		If, however, the \texttt{READY} state would have been active, the state could be transitioned to the \texttt{BUSY} value and a success response be provided to the sender.
	}

\GlossaryEntry{protocol-device}{Protocol, Device}
A \GlossaryHyperRef{protocol}{protocol} implemented by a \GlossaryHyperRef{device}{device}.

\GlossaryEntry{protocol-extensible}{Protocol, Extensible}
A \GlossaryHyperRef{protocol}{protocol} allowing for \GlossaryHyperRef{subprotocol}{subprotocols} to be formulated in terms of its \GlossaryHyperRef{message}{messages}.
See also \GlossaryNameRef{stack-protocol}.

	\GlossaryNote{Note 1}{
		Every new message introduced by a subprotocol must be a \GlossaryHyperRef{validation}{valid} message of its \GlossaryHyperRef{superprotocol}{superprotocol}.
	}

	\GlossaryNote{Note 2}{
		Many of the currently prevalent protocols are designed with the intent of being extensible.
		For example, HTTP \cite{fielding2014hypertext} provides provisions for an extending protocol to define its own set of directory operations, to simultaneously support multiple \GlossaryHyperRef{encoding}{codecs}, and so on.
	}

	\GlossaryNote{Note 3}{
		As long as a given protocol provides at least one message whose contents can be arbitrary, a subprotocol can be produced.
		This means that even protocols not designed to be extended can, in some context, be meaningfully used to define subprotocols.
	}

\GlossaryEntry{protocol-function}{Protocol, Function}
A \GlossaryHyperRef{protocol}{protocol} implemented by a \GlossaryHyperRef{function-service}{service function}.

	\GlossaryNote{Note 1}{
		A function protocol is always an \GlossaryHyperRef{protocol-extensible}{extension} of a \GlossaryHyperRef{protocol-service}{service protocol}.
		See Section \ref{sec:reference-model:protocol} for more details.
	}

\GlossaryEntry{protocol-service}{Protocol, Service}
A \GlossaryHyperRef{protocol}{protocol} implemented by a \GlossaryHyperRef{service}{service}.

	\GlossaryNote{Note 1}{
		A service protocol is always an \GlossaryHyperRef{protocol-extensible}{extension} of a \GlossaryHyperRef{protocol-system}{system protocol}.
		See Section \ref{sec:reference-model:protocol} for more details.
	}

\GlossaryEntry{protocol-system}{Protocol, System}
A \GlossaryHyperRef{protocol}{protocol} implemented by a \GlossaryHyperRef{system}{system}.

	\GlossaryNote{Note 1}{
		A system protocol is always an \GlossaryHyperRef{protocol-extensible}{extension} of a \GlossaryHyperRef{protocol-device}{device protocol}.
		See Section \ref{sec:reference-model:protocol} for more details.
	}

\GlossaryEntry{provider-service}{Provider, Service}
A \GlossaryHyperRef{system}{system} that makes \GlossaryHyperRef{service}{services} available for \GlossaryHyperRef{consumer-service}{consumption} to other systems.

	\GlossaryNote{Note 1}{
		If used to refer to a \GlossaryHyperRef{stakeholder}{stakeholder}, the term must be interpreted as if that stakeholder provides services via systems.
	}

	\GlossaryNote{SOA-RM}{
		defines a service provider as ``an entity (person or organization) that offers the use of capabilities by means of a service''.
		Our definition is more specific in that it requires the \GlossaryHyperRef{entity}{entity} be a system.
	}

\GlossaryEntry{qos}{QoS}
See \GlossaryNameRef{service-quality-of}.

\GlossaryEntry{relationship}{Relationship}
A uni-directional association of two \GlossaryHyperRef{entity}{entities}, possibly with an associated \GlossaryHyperRef{data}{data} name.

\GlossaryEntry{resource}{Resource}
An \GlossaryHyperRef{artifact}{artifact} that is of value to a \GlossaryHyperRef{stakeholder}{stakeholder}.

	\GlossaryNote{Note 1}{
		Any type of artifact can be a resource, which includes everything from \GlossaryHyperRef{resource-local}{local resources}, such as raw materials on \GlossaryHyperRef{device}{devices}, to \GlossaryHyperRef{resource-virtual}{virtual resources}, such as \GlossaryHyperRef{system}{systems} or \GlossaryHyperRef{data}.
	}

	\GlossaryNote{Note 2}{
		An artifact stops be a resource when it is perceived as having no value, at which point it may be destroyed, recycled or sold to someone that does perceive it as a resource, for example.
	}

\GlossaryEntry{resource-local}{Resource, Local}
A \GlossaryHyperRef{resource}{resource} whose value is inextricably tied to a physical \GlossaryHyperRef{property}{property}.

	\GlossaryNote{Note 1}{
		Examples of local resources could be raw materials, drills, pumps, power stations, or drones.
	}

\GlossaryEntry{resource-virtual}{Resource, Virtual}
A \GlossaryHyperRef{resource}{resource} whose value is not derived from any physical \GlossaryHyperRef{property}{property}.

	\GlossaryNote{Note 1}{
		Examples of virtual resources could be compute, storage, or software-defined network utilities.
	}

\GlossaryEntry{role}{Role}
See \GlossaryNameRef{role-stakeholder}.

\GlossaryEntry{router}{Router}
See \GlossaryNameRef{router-message}.

\GlossaryEntry{router-message}{Router, Message}
A \GlossaryHyperRef{component-hardware}{hardware component} or \GlossaryHyperRef{procedure-software}{software procedure} that receives and passes on \GlossaryHyperRef{message}{messages} toward their intended end \GlossaryHyperRef{entity}{entities}.

\GlossaryEntry{role-stakeholder}{Role, Stakeholder}
An assignment, objective, or other responsibility, that makes a person or organization into a \GlossaryHyperRef{stakeholder}{stakeholder}. 

\GlossaryEntry{routing-message}{Routing, Message}
The act of forwarding a \GlossaryHyperRef{message}{message} towards the \GlossaryHyperRef{function}{function} it is meant to \GlossaryHyperRef{invocation-function}{invoke}.

\GlossaryEntry{service}{Service}
A set of \GlossaryHyperRef{function-service}{functions} that can be \GlossaryHyperRef{provider-service}{provided} by a \GlossaryHyperRef{system}{system} via one or more \GlossaryHyperRef{interface-service}{service interfaces}.
See Section \ref{sec:reference-model:service}.

	\GlossaryNote{RAMI4.0}{
		defines a service as ``separate scope of functions offered by an entity or organization via interfaces''.
		Our definition restricts service provision to systems.
	}

	\GlossaryNote{SOA-RM}{
		defines a service as ``the means by which the needs of a consumer are brought together with the capabilities of a provider''.
		Our definition is more specific about how the \GlossaryHyperRef{capability-system}{capabilities} of a service are made available.
	}

	\GlossaryNote{IoTA:AF}{
		defines a service as ``what [is] used to exchange information from a providing system to a consuming system''.
		It further adds that ``in a service, capabilities are grouped together if they share the same context''.
		The definition presented here should be interpreted as being compatible but more specific about how information is exchanged and capabilities are \GlossaryHyperRef{invocation-function}{invoked}.
	}

\GlossaryEntry{service-quality-of}{Service, Quality of (QoS)}
The degree of performance at which a given \GlossaryHyperRef{service}{service} is \GlossaryHyperRef{provider-service}{provided}. 

\GlossaryEntry{soa}{SOA}
See \GlossaryNameRef{architecture-service-oriented}.

\GlossaryEntry{software}{Software}
A set of sequences of instructions that can be executed by a \GlossaryHyperRef{unit-compute}{compute unit}.

\GlossaryEntry{solc}{SoLC}
See \GlossaryNameRef{system-of-local-clouds}.

\GlossaryEntry{sos}{SoS}
See \GlossaryNameRef{system-of-systems}.

\GlossaryEntry{stack-extensible-protocol}{Stack, Extensible Protocol}
A \GlossaryHyperRef{stack-protocol}{protocol stack} whose topmost \GlossaryHyperRef{protocol}{protocol} is \GlossaryHyperRef{protocol-extensible}{extensible}.

\GlossaryEntry{stack-protocol}{Stack, Protocol}
A stack with an \GlossaryHyperRef{protocol-extensible}{extensible protocol} as base and $n > 0$ \GlossaryHyperRef{subprotocol}{subprotocols} layered on top of it.

	\GlossaryNote{Note 1}{
		Every protocol part of a protocol stack, with the exception of the topmost, must be extensible.
	}

	\GlossaryNote{Note 2}{
		An example of a notable protocol stack is that of HTTP \cite{fielding2014hypertext}.
		It is defined as an extension of the TCP protocol, which in turn extends the IP protocol, which can work as a subprotocol of several other lower-level protocols.
		HTTP is an \GlossaryHyperRef{stack-extensible-protocol}{extensible protocol stack}, which allows for an engineer to define an application-specific protocol on top of its stack.
	}

\GlossaryEntry{stake}{Stake}
Any type of engagement or commitment.

\GlossaryEntry{stakeholder}{Stakeholder}
A person or \GlossaryHyperRef{organization}{organization} with \GlossaryHyperRef{stake}{stake} in certain \GlossaryHyperRef{entity}{entities} or enterprises.
See Section \ref{sec:reference-model:stakeholder}.

\GlossaryEntry[(noun)]{state}{State}
One out of all possible sequences of values that could be expressed by the \GlossaryHyperRef{datum}{datums} of some \GlossaryHyperRef{data}{data}.

	\GlossaryNote{Note 1}{
		If the data would consist of a sequence of bits, each of which can only have the values 0 and 1, a state becomes a pattern of zeroes and ones those bits can record.
		Given four bits, possible states could, for example, be \texttt{0010} or \texttt{1001}. 
	}

	\GlossaryNote{Note 2}{
		The term is often used as a wildcard for any kind of storage construct, including bit flags, state machines and graph databases.
	}

\GlossaryEntry{state-protocol}{State, Protocol}
The \GlossaryHyperRef{state}{state} of a \GlossaryHyperRef{protocol}{protocol} in active use, determining what \GlossaryHyperRef{message}{messages} it currently deems valid.
See Section \ref{sec:reference-model:protocol}.

\GlossaryEntry{state-software}{State, Software}
The \GlossaryHyperRef{state}{state} of a \GlossaryHyperRef{instance-software}{software instance}, determining its current activities and its reactions to any future \GlossaryHyperRef{procedure-software}{procedure} calls.

\GlossaryEntry{subprotocol}{Subprotocol}
A \GlossaryHyperRef{protocol}{protocol} that is realized as an \GlossaryHyperRef{protocol-extensible}{extension} of another protocol.

\GlossaryEntry{subsystem}{Subsystem}
A \GlossaryHyperRef{system}{system} or \GlossaryHyperRef{system-of-systems}{system-of-systems} being a constituent of a larger system-of-systems.

\GlossaryEntry{superprotocol}{Superprotocol}
A \GlossaryHyperRef{protocol}{protocol} that is \GlossaryHyperRef{protocol-extensible}{extended} by another protocol.

\GlossaryEntry{system}{System}
An \GlossaryHyperRef{entity}{entity} capable of \GlossaryHyperRef{provider-service}{providing services}, \GlossaryHyperRef{consumer-service}{consuming services}, or both.

	\GlossaryNote{Note 1}{
		The word ``system'' is more generally understood to be very inclusive, expressing the larger idea of connected \GlossaryHyperRef{component}{components} facilitating one or more \GlossaryHyperRef{capability-system}{capabilities}.
		From the perspective of Arrowhead, however, capabilities can only be \GlossaryHyperRef{invocation-function}{invoked} through \GlossaryHyperRef{service}{services}, which means that a system unable to provide or consume services can only be described as a component of another system.
	}

	\GlossaryNote{Note 2}{
		A system is practically distinct from a \GlossaryHyperRef{system-of-systems}{system-of-systems} by being represented only by a single \GlossaryHyperRef{identity}{identity}.
		In contrast, a system-of-systems does either not have its own identity, or has both its own identity and another identity for each of its \GlossaryHyperRef{subsystem}{subsystems}.
	}

	\GlossaryNote{IoTA:AF}{
		defines a system as ``what is providing and/or consuming services''.
		It further adds that ``a system can be the service provider of one or more services and at the same time the service consumer of one or more services''.
		The definition presented here should be interpreted as equivalent.
	}

\GlossaryEntry{system-isolated}{System, Isolated}
A \GlossaryHyperRef{system}{system} that is unable to either \GlossaryHyperRef{provider-service}{provide} or \GlossaryHyperRef{consumer-service}{consume} \GlossaryHyperRef{service}{services}.

\GlossaryEntry{system-supervisory}{System, Supervisory}
A \GlossaryHyperRef{system}{system} that is tasked with managing one or more \GlossaryHyperRef{resource}{resources} beyond its direct control.

	\GlossaryNote{Note 1}{
		All systems are managing the resources provided to them by their hosting \GlossaryHyperRef{device}{devices}, such as primary memory, compute time, and so on.
		This term is meant to capture the systems that are engaged in overseeing and/or managing resources beyond those directly provided.
		Examples of such scenarios could be a single system being responsible for provisioning other devices, or a system using its robot device to collect and handle raw materials.
	}

\GlossaryEntry{system-type}{System, Type}
A set of \GlossaryHyperRef{type-data}{data types} that can be used together, often in the context of an \GlossaryHyperRef{encoding}{encoding} or programming language.

\GlossaryEntry{system-of-local-clouds}{System-of-Local-Clouds (SoLC)}
A set of \GlossaryHyperRef{cloud-local}{local clouds} that \GlossaryHyperRef{consumer-service}{consume} each other's \GlossaryHyperRef{service}{services} in order to facilitate a \GlossaryHyperRef{capability-system}{capability} none of the constituent local clouds could \GlossaryHyperRef{provider-service}{provide} on its own.
See Section \ref{sec:reference-model:system-of-systems:system-of-local-clouds}.

\GlossaryEntry{system-of-systems}{System-of-Systems (SoS)}
A set of \GlossaryHyperRef{system}{systems} that \GlossaryHyperRef{consumer-service}{consume} each other's \GlossaryHyperRef{service}{services} in order to facilitate a \GlossaryHyperRef{capability-system}{capability} none of the constituent systems could \GlossaryHyperRef{provider-service}{provide} on its own.
See Section \ref{sec:reference-model:system-of-systems}.

	\GlossaryNote{IoTA:AF}{
		defines a system-of-systems as ``a set of system, which [...] exchange information by means of services''.
		It further adds that ``when Arrowhead compliant systems collaborate, they become a System of Systems in the Arrowhead Framework's definition''.
		While we clarify here that the desired outcome of collaboration is the facilitation of new capabilities, the definitions should be interpreted as being equivalent.
	}

\GlossaryEntry{type-data}{Type, Data}
A \GlossaryHyperRef{description}{description} of how datums are to be arranged to \GlossaryHyperRef{code}{code} certain facts.
See also \GlossaryNameRef{data}.

	\GlossaryNote{Note 1}{
		While this definition may seem foreign, it does capture how integer types, classes, enumerators and other general data type are used in the context of a programming language or \GlossaryHyperRef{encoding}{encoding}.
		In the end, all data are bits or other symbols.
		From our perspective, types serve to group those symbols and assign them meaning.
	}

	\GlossaryNote{Note 2}{
		A data type provides only syntactic, or structural, information about data.
		While knowing the data type used to code some data is required for its interpretation, contextual knowledge is also needed.
		For example, a data type may specify a \texttt{name}, but it will not indicate when or why that name is useful.
		That information would have to be provided via documentation or some other means.
	}

\GlossaryEntry{unit-compute}{Unit, Compute}
A \GlossaryHyperRef{component-hardware}{hardware component} able to execute \GlossaryHyperRef{software}{software} adhering to its instruction set.

\GlossaryEntry{unit-memory}{Unit, Memory}
A \GlossaryHyperRef{component-hardware}{hardware component} maintaining a set of changeable \GlossaryHyperRef{datum}{datums}.

\GlossaryEntry{user}{User}
A \GlossaryHyperRef{stakeholder}{stakeholder} involved in the usage of certain \GlossaryHyperRef{entity}{entities}.
See Section \ref{sec:reference-model:stakeholder}.

	\GlossaryNote{Note 1}{
		The activity of \textit{using} an entity is not related to its coming into existence, maintenance, decommissioning, or any other peripheral activity.
		When a user engages in an entity it produces whatever value it was designed to produce.
	}

\GlossaryEntry{validation}{Validation}
The process through which it is determined if a \GlossaryHyperRef{model}{model} satisfies a \GlossaryHyperRef{constraint}{constraint}.

}