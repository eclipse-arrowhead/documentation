% Copyright (c) 2021-10-07 Eclipse Arrowhead Project
%
% This program and the accompanying materials are made available under the
% terms of the Eclipse Public License 2.0 which is available at
% http://www.eclipse.org/legal/epl-2.0.
%
% SPDX-License-Identifier: EPL-2.0

\newcommand{\GlossaryEntry}[2]{\paragraph{#1}\label{sec:glossary:#2}}
\newcommand{\GlossaryHyperRef}[2]{{\color{ArrowheadDarkBlue}\hyperref[sec:glossary:#2]{#1}}}
\newcommand{\GlossaryNameRef}[1]{{\color{ArrowheadDarkBlue}\nameref{sec:glossary:#1}}}
\newcommand{\GlossarySourceNote}[2]{\begin{minipage}[b]{\dimexpr\linewidth-0.5cm\relax}\vspace*{0.33cm}\footnotesize{\textbf{#1}\ #2}\end{minipage}}

\GlossaryEntry{Administration}{administration}

\GlossaryEntry{Administrator}{administrator}

\GlossaryEntry{ANI}{ani} See \GlossaryNameRef{interface-application-network}.

\GlossaryEntry{API}{api} See \GlossaryNameRef{interface-application-programming}.

\GlossaryEntry{Application}{application}

\GlossaryEntry{Architecture}{architecture}
A \GlossaryHyperRef{model}{model} \GlossaryHyperRef{describing}{description} a \GlossaryHyperRef{system}{system}, such as a vehicle or \GlossaryHyperRef{software}{software} artifact, \GlossaryHyperRef{constrained}{constraint-model} either by \GlossaryHyperRef{abstract}{architecture-abstract} or \GlossaryHyperRef{concrete}{architecture-concrete} \GlossaryHyperRef{context}{context} \GlossaryHyperRef{descriptions}{description}.
See also Section \ref{sec:introduction:scope}.

	\GlossarySourceNote{RAMI4.0}{
		This definition replaces ``combinations of elements of a model'' with ``system'' and ``based on principles and rules'' with ``constrained ... by''.
		It also clarifies that an architecture can be either abstract or concrete.
	}

\GlossaryEntry{Architecture, Abstract}{architecture-abstract} A \GlossaryHyperRef{model}{model} of a \GlossaryHyperRef{system}{system} in an \textit{abstract} \GlossaryHyperRef{context}{context}.
An abstract \GlossaryHyperRef{architecture}{architecture} can serve as a component or \GlossaryHyperRef{reference}{architecture-reference} for other abstract and \GlossaryHyperRef{concrete}{architecture-concrete} architectures.

\GlossaryEntry{Architecture, Concrete}{architecture-concrete}
A \GlossaryHyperRef{model}{model} of a \GlossaryHyperRef{system}{system} \GlossaryHyperRef{constrained}{constraint-model} by a \textit{concrete} \GlossaryHyperRef{context}{context}, which may also be represented by an explicit model.
A concrete \GlossaryHyperRef{architecture}{architecture} can be realized as a concrete system, such as a physical car or an executable software artifact, or be used as a component or \GlossaryHyperRef{reference}{architecture-reference} of another concrete architecture.

\GlossaryEntry{Architecture, Reference}{architecture-reference}
An \GlossaryHyperRef{architecture}{architecture} recognized as being useful as reference, or foundation, for another architecture or work.
A reference architecture can be either \GlossaryHyperRef{abstract}{architecture-abstract} or \GlossaryHyperRef{concrete}{architecture-concrete}.

	\GlossarySourceNote{RAMI4.0}{This definition clarifies that an architecture can be either abstract or concrete.}

\GlossaryEntry{Arrowhead}{arrowhead} TODO

\GlossaryEntry{Asset}{asset}
An object, tangible or intangible, of any sort that is deemed to be of value to an owning \GlossaryHyperRef{organization}{organization}.

	\GlossarySourceNote{RAMI4.0}{This definition clarifies that an object can be either tangible or intangible.}

\GlossaryEntry{Authentication}{authentication} TODO

\GlossaryEntry{Authorization}{authorization} TODO

\GlossaryEntry{Capability}{capability}
An action that can be performed by an \GlossaryHyperRef{entity}{entity}.

	\GlossarySourceNote{OASIS-RM}{
		OASIS-RM defines a capability as ``a real-world effect that a service provider is able to provide to a service consumer''.
		As we fear that some could interpret ``real-world'' as ``physical-world'' and that our scope is wider, we adjusted their definition accordingly.
	}

\GlossaryEntry{Certificate}{certificate}

\GlossaryEntry{Cloud}{cloud}

\GlossaryEntry{Cloud, Compute}{cloud-compute}

\GlossaryEntry{Cloud, Local}{cloud-local}

\GlossaryEntry{Cloud, Storage}{cloud-storage}

\GlossaryEntry{Cloud, Virtual}{cloud-virtual}

\GlossaryEntry{Codec}{codec} TODO

\GlossaryEntry{Coding}{coding} TODO

\GlossaryEntry{Component}{component}
A part or subsystem of a larger \GlossaryHyperRef{system}{system}.

\GlossaryEntry{Configuration}{configuration}

\GlossaryEntry{Constraint, Model}{constraint-model}
A relation between two \GlossaryHyperRef{entities}{entity} of a \GlossaryHyperRef{model}{model} where one entity imposes constraints, or limits, on the other.
The existence of model constraints enable \GlossaryHyperRef{model validation}{validation-model}.

\GlossaryEntry{Consumer, Service}{consumer-service} TODO

\GlossaryEntry{Context}{context}
A \GlossaryHyperRef{model}{model} of an \GlossaryHyperRef{abstract}{context-abstract} or \GlossaryHyperRef{concrete}{context-concrete} environment in which an \GlossaryHyperRef{entity}{entity} can be situated.
The entity in question is \GlossaryHyperRef{constrained}{constraint-model} by, or limited to, the \GlossaryHyperRef{data}{data} and other resources the context provides.

\GlossaryEntry{Context, Abstract}{context-abstract}

\GlossaryEntry{Context, Concrete}{context-concrete}

\GlossaryEntry{Data}{data} TODO

\GlossaryEntry{Decode}{decode} TODO

\GlossaryEntry{Decoding}{decoding} TODO

\GlossaryEntry{Description}{description}
A set of details about an \GlossaryHyperRef{entity}{entity}, concretely represented by a \GlossaryHyperRef{model}{model} or a human-readable text.

\GlossaryEntry{Design}{design}

\GlossaryEntry{Device}{device}

\GlossaryEntry{Device, Human Interface (HID)}{device-human-interface}

\GlossaryEntry{Encode}{encode} TODO

\GlossaryEntry{Encoding}{encoding} TODO

\GlossaryEntry{Entity}{entity}

\GlossaryEntry{Enumerator}{enumerator} See \GlossaryNameRef{type-enumerating}.

\GlossaryEntry{Framework}{framework}

\GlossaryEntry{Function}{function}
An \GlossaryHyperRef{invocable}{invocation-function} subroutine, exercising one or more \GlossaryHyperRef{capabilities}{capability} of a \GlossaryHyperRef{providing}{provider-service} \GlossaryHyperRef{system}{system}.
Every function \textit{must} be made available via a \GlossaryHyperRef{service}{service} and comply to a \GlossaryHyperRef{function signature}{signature-function}, which specifies the \GlossaryHyperRef{type}{type-data} of \GlossaryHyperRef{message}{message} the function expects when invoked, as well as the types of every message the function could return in response.

\GlossaryEntry{Identifier}{identifier}

\GlossaryEntry{Identity}{identity}

\GlossaryEntry{Interface}{interface}

\GlossaryEntry{Interface, Administrative}{interface-administrative}

\GlossaryEntry{Interface, Application Network (ANI)}{interface-application-network}

\GlossaryEntry{Interface, Application Programming (API)}{interface-application-programming}

\GlossaryEntry{Interface, Management}{interface-management}

\GlossaryEntry{Interface, Network}{interface-network}

\GlossaryEntry{Interface, Operator}{interface-operator}

\GlossaryEntry{Invocation, Function}{invocation-function}
The attempt to trigger one or more \GlossaryHyperRef{capabilities}{capability} associated with a \GlossaryHyperRef{function}{function} by sending a \GlossaryHyperRef{message}{message} to the \GlossaryHyperRef{service}{service} through which it is \GlossaryHyperRef{provided}{provider-service}.
A function invocation \textit{must} cause its provider to return a response message if mandated by the \GlossaryHyperRef{signature}{signature-function} of the function.

\GlossaryEntry{Manager}{manager}

\GlossaryEntry{Management}{management}

\GlossaryEntry{Message}{message} TODO

\GlossaryEntry{Metadata}{metadata} TODO

\GlossaryEntry{Model}{model} TODO

\GlossaryEntry{Model, Reference}{model-reference} TODO

\GlossaryEntry{Operator}{operator}

\GlossaryEntry{Organization}{organization}

\GlossaryEntry{Policy}{policy}

\GlossaryEntry{Policy, Service Consumption}{policy-service-consumption}

\GlossaryEntry{Process, Application}{process-application}

\GlossaryEntry{Profile}{profile}

\GlossaryEntry{Protocol}{protocol}

\GlossaryEntry{Provider, Service}{provider-service}
A \GlossaryHyperRef{system}{system} making the \GlossaryHyperRef{functions}{function} of a \GlossaryHyperRef{service}{service} available to be \GlossaryHyperRef{invoked}{invocation-function} by a set of \GlossaryHyperRef{service consumers}{consumer-service}, which are made eligible by conforming to whatever \GlossaryHyperRef{service consumption policies}{policy-service-consumption} are set up by the provider.

	\GlossarySourceNote{OASIS-RM}{
		OASIS-RM defines a service provider as ``an entity (person or organization) that offers the use of capabilities by means of a service''.
		While we require that the one being offered the service be a \GlossaryHyperRef{system}{system}, that system could be seen as representing the interests of a person or organization.
		We consider the definitions to be compatible, even though our definition is more specific.
	}

\GlossaryEntry{Service}{service}
A set of \GlossaryHyperRef{functions}{function} \GlossaryHyperRef{provided}{provider-service} by a \GlossaryHyperRef{system}{system} via an \GlossaryHyperRef{interface}{interface}.
A service can be provided via an \GlossaryHyperRef{ANI}{interface-application-network}, \GlossaryHyperRef{API}{interface-application-programming}, or any other kind of interface.

	\GlossarySourceNote{RAMI4.0}{This definition clarifies that a service could be provided via an \GlossaryHyperRef{ANI}{interface-application-network} or \GlossaryHyperRef{API}{interface-application-programming}.}

	\GlossarySourceNote{OASIS-RM}{As noted in RAMI4.0, this definition is \textit{not compatible} with that of OASIS-RM, which defines a service as ``the means by which the needs of a consumer are brought together with the capabilities of a provider''.}

\GlossaryEntry{Session}{session}

\GlossaryEntry{Shell, Administrative}{shell-administrative}

\GlossaryEntry{Signature, Function}{signature-function}
A \GlossaryHyperRef{model}{model} specifying the \GlossaryHyperRef{type}{type-data} of the \GlossaryHyperRef{message}{message} a given \GlossaryHyperRef{function}{function} accepts when \GlossaryHyperRef{invoked}{invocation-function}, as well as the types of any messages that it could return in response.

\GlossaryEntry{Software}{software}

\GlossaryEntry{SoLC}{solc} See \GlossaryNameRef{system-of-local-clouds}.

\GlossaryEntry{SoS}{sos} See \GlossaryNameRef{system-of-systems}.

\GlossaryEntry{Specification}{specification}

\GlossaryEntry{Stack, Protocol}{stack-protocol} See \GlossaryNameRef{protocol}.

\GlossaryEntry{Standard}{standard}

\GlossaryEntry{Stakeholder}{stakeholder}

\GlossaryEntry{Subsystem}{subsystem} See \GlossaryNameRef{component}.

\GlossaryEntry{System}{system} TODO

\GlossaryEntry{System, Arrowhead}{system-arrowhead} TODO

\GlossaryEntry{System-of-Local-Clouds (SoLC)}{system-of-local-clouds}

\GlossaryEntry{System-of-Systems (SoS)}{system-of-systems}

\GlossaryEntry{Token}{token} See \GlossaryNameRef{token-authentication}.

\GlossaryEntry{Token, Authentication}{token-authentication}

\GlossaryEntry{Type, Data}{type-data} TODO

\GlossaryEntry{Type, Enumerating}{type-enumerating} TODO

\GlossaryEntry{Type, Primitive}{type-primitive} TODO

\GlossaryEntry{Type, Structured}{type-structured} TODO

\GlossaryEntry{User}{user}

\GlossaryEntry{Validation, Model}{validation-model}
The process through which it is determined if a \GlossaryHyperRef{model}{model} satisfies all of its \GlossaryHyperRef{constraints}{constraint-model}.