The Eclipse Arrowhead project is an effort at realizing the \textit{Industry 4.0} vision by providing the concrete software required to facilitate the degree of interoperation and dynamicity it makes relevant.
The effective use of any engineering artifacts depends critically on the ability of decision-makers and engineers to communicate and understand relevant concepts.
This reference model is meant to establish a foundational set of such concepts for Eclipse Arrowhead, both by providing human-readable descriptions and formal models.
It builds on \textit{Reference Architecture Model Industrie 4.0} (RAMI4.0) \cite{adolphs2016reference}, the \textit{Reference Model for Service Oriented Architecture} (SOA) \cite{mackenzie2006reference} and other relevant documents.

\subsection{Purpose}
\label{sec:introduction:purpose}

%TODO: What is a reference model? See SOA document.

\subsection{Audience}
\label{sec:introduction:audience}

This document is being written and maintained with the below audiences in mind.
It may, of course, be of advantage also to others not included in the list.

\begin{itemize}
\item \textit{System architects, integrators and developers} designing, integrating or developing systems relying on Arrowhead.
\item \textit{Decision makers, users and other stakeholders} requiring to understand the more fundamental concepts of Arrowhead.
\end{itemize}

Audiences that do not seek a highly rigorous description of Arrowhead may want to focus their reading on Section \ref{sec:arrowhead}, while others are adviced to read all sections carefully.

\subsection{Usage}
\label{sec:introduction:usage}

\subsection{Conventions}
\label{sec:introduction:conventions}

\subsubsection{Requirements}

Use of the words \textit{must}, \textit{must not}, \textit{required}, \textit{should}, \textit{should not}, \textit{recommended}, \textit{may}, and \textit{optional} are to be interpreted as follows when used in this document: \textit{must} and \textit{required} denote absolute requirements that must be adhered to for a described entity to be considered as compliant to this reference model; \textit{must not} denotes an absolute prohibition; \textit{should}, \textit{should not} and \textit{recommended} denote recommendations that should be deviated from only if special circumstances make it relevant; and, finally, \textit{may} and \textit{optional} denote something being truly optional.
These word definitions are derived from and are meant to capture what is outlined in RFC 2119 \cite{bradner1997keywords}.

\subsubsection{Diagrams}

\subsection{Relationships to Other Documents}
\label{sec:introduction:relationships}

\subsection{Section Overview}
\label{sec:introduction:sections}

\begin{itemize}[leftmargin=3cm,rightmargin=0pt,labelwidth=2cm,labelsep=0pt,itemindent=0pt,parsep=0.25cm,topsep=0.25cm,align=left]

\item[Section \ref{sec:introduction}]
This section.

\item[Section \ref{sec:arrowhead}]
An informal overview of Arrowhead, serving both to provide a workable summary of the framework and to prepare readers for better understanding Section \ref{sec:model}.

\item[Section \ref{sec:model}]
The formal and normative description of Arrowhead.

\item[Section \ref{sec:conformance}]
A brief list of requirements, meant to help determine whether or not a given system is conforming to this document.

\item[Section \ref{sec:references}]
Lists references to publications referred to in this document.

\item[Section \ref{sec:revision}]
Records the history of changes made to this document.

\end{itemize}
