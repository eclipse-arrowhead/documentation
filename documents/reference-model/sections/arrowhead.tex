% Copyright (c) 2021 Eclipse Arrowhead Project
%
% This program and the accompanying materials are made available under the
% terms of the Eclipse Public License 2.0 which is available at
% http://www.eclipse.org/legal/epl-2.0.
%
% SPDX-License-Identifier: EPL-2.0

The \GlossaryHyperRef{framework-arrowhead}{Arrowhead framework} is two things.
Firstly, it is a framework of assumptions, concepts, values and practices that frame the problem domain of \textit{dynamic device coordination in industrial contexts}.
Secondly, it is a set of software specifications, \GlossaryHyperRef{implementation-software}{implementations} and other \GlossaryHyperRef{artifact}{artifacts} meant to help address that problem domain.
In this section, we provide an overview of the primary \textit{concepts} of the Arrowhead framework.
While \textit{assumptions} and \textit{values} may be possible derive from this overview, neither of these, nor the other parts of the framework, are considered directly.

\paragraph{Stakeholders and artifacts.}
The are two kinds of citizens in the world of Arrowhead, (1) \GlossaryHyperRef{stakeholder}{stakeholders} and (2) \GlossaryHyperRef{artifact}{artifacts}.
The former denotes a person or organization with \GlossaryHyperRef{stake}{stake} in an Arrowhead enterprise, while the latter is any thing or object, tangible or intangible, that could be relevant to consider as part of such an enterprise.
Stakeholders \GlossaryHyperRef{owner}{own}, \GlossaryHyperRef{designer}{design}, \GlossaryHyperRef{developer}{develop}, \GlossaryHyperRef{operator}{operate}, and \GlossaryHyperRef{user}{use} artifacts, among many other possible activities.
It is their business needs and ambitions that dictate what and how Arrowhead artifacts will be employed.

\paragraph{Devices, systems and services.}
The most essential types of Arrowhead artifacts are (1) \GlossaryHyperRef{device}{devices}, (2) \GlossaryHyperRef{system}{systems} and (3) \GlossaryHyperRef{service}{services}.
\textit{Devices} constitute the physical machines that make up the industrial complexes, vehicles, tools, and other things that could be made operational via Arrowhead.
Each device hosts one or more \textit{systems}, which are \GlossaryHyperRef{communication}{communicating} \GlossaryHyperRef{instance-software}{software instances} that make their devices work toward whatever goals are set for them.
Finally, a \textit{service} is a set of related tasks that a system can make its device do for a person or another system.
Each task is concretely represented by a \GlossaryHyperRef{function-service}{function}.
Services can be concerned with vehicle repairs, information backups, analysis, manufacturing, or any other activity a system can perform via its device.
The service is the means whereby systems coordinate to fulfill their assignments.

\paragraph{Service provision and consumption.}
Communication between systems is formulated in terms of the \GlossaryHyperRef{provider-service}{provision} and \GlossaryHyperRef{consumer-service}{consumption} of services.
When a system \textit{provides} a service, it makes it available to other systems through certain \GlossaryHyperRef{interface-service}{service interfaces}.
Other systems can \textit{consume} the services of other systems by sending \GlossaryHyperRef{message}{messages} to the service interfaces of those services.


\paragraph{System-of-systems.}

